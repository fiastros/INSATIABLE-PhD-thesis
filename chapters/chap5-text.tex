% \chapter*{\Huge Big Title Here}
% \addcontentsline{toc}{chapter}{Big Title Here}  % Add to TOC if needed

\chapter{General discussion} % Main chapter title

% 


%----------------------------------------------------------------------------------------
%	SECTION 
%----------------------------------------------------------------------------------------
\section{Main contributions}
%-----------%-----------
%	SOUS-SECTION 
%-----------%-----------


\paragraph{Integration of sensor-based and mechanistic models} Central methodological contribution of this thesis is the integration of deep learning diagnostic tools with mechanistic epidemiological models to improve disease management, specifically for bovine respiratory disease (BRD). This hybrid AI-epidemiological approach explicitly tackles the challenge of reconciling short-term, sensor-driven diagnostic accuracy with long-term, model-based epidemiological forecasting. We drew inspiration from the Mixture of Experts paradigm by loosely coupling two specialized components – one for diagnosis and one for prognosis – and assigning each to its domain of expertise. This design allowed deep learning to focus on immediate, data-driven classification of disease status, while the mechanistic model provided reliable long-term projections based on epidemiological principles. The approach demonstrated two key strengths. First, the deep learning module excelled at rapid, accurate diagnosis from noisy sensor data (lung ultrasound imagery), even with limited training examples. For instance, the model achieved about 72\% classification accuracy using fewer than 30 training ultrasound video samples, highlighting its practicality in data-scarce veterinary settings. Notably, when we enhanced this diagnostic model with a Bayesian architecture to quantify prediction confidence, the effective accuracy rose to roughly 88\% by filtering out highly uncertain cases – an improvement that underscores the value of uncertainty-aware AI in automating complex clinical assessments \cite{gal2016dropout}
. Second, the mechanistic epidemiological module excelled at prognosis: it extended the short-term diagnostic insights into accurate long-term forecasts of disease dynamics. After parametrising the model with empirical BRD outbreak observations, it produced robust epidemic trajectory predictions with a forecast error (RMSE) below 10\%. This level of accuracy demonstrates the mechanistic model’s value for strategic decision-making over longer time horizons, complementing the immediacy of the deep learning diagnoses. By systematically separating diagnostic and prognostic tasks, our hybrid framework capitalizes on the respective strengths of data-driven AI and mechanistic modelling, an approach aligned with recent calls to combine these paradigms for epidemic prediction Overall, this integration of sensor-based AI and mechanistic modelling is an innovative step toward decision-support tools that operate across temporal scales of disease management \cite{chen2024hybrid}.



\paragraph{Explicit handling of uncertainty}  A second major contribution of this work is the explicit quantification and propagation of uncertainty within the hybrid diagnostic-prognostic pipeline. We addressed the question of how to manage the significant uncertainty inherent in BRD sensor data (noisy ultrasound observations of pathological lung changes) through a Bayesian Deep Mechanistic approach. By employing Monte Carlo Dropout for variational inference, the deep learning model generated probabilistic predictions along with measures of confidence \cite{gal_dropout_2016}. We then filtered out the most uncertain diagnoses – essentially having the system “know what it doesn’t know” – which yielded a notable reduction in diagnostic error rates (from an initial relative RMSE of 39\% down to 32\%). These uncertainty-filtered predictions were subsequently used in the mechanistic model’s calibration via a weighted approximate Bayesian computation scheme, so that less certain inputs were given diminished influence. Propagating the diagnostic uncertainty in this manner improved the reliability of long-term forecasts: the hybrid model’s projection error dropped to a relative RMSE of 27.2\%, approaching the 23\% error of a baseline model informed by expert veterinary diagnoses. In other words, our framework better matched expert-driven forecasts by embracing uncertainty rather than ignoring it. This result is significant because it shows that a principled treatment of uncertainty can enhance both immediate and future predictions in disease monitoring systems. The outcome supports recent observations that accounting for prediction confidence improves epidemiological forecasts. Importantly, this contribution tackles real-world complexity: livestock health data are often limited or imperfect, and by embedding uncertainty into the decision pipeline, our approach maintained robust performance even when data quality was compromised. This methodological advance – integrating Bayesian deep learning with epidemiological simulation – directly confronts the need for reliability in AI-driven agriculture, ensuring the system remains cautious and reliable under data ambiguity.

% [remark on Explicit integration and propagation of uncertainty: à développer, notamment en expliquant plus en détail "pourquoi ça marche" et l'articulation entre prédiction ML et prédiction méca". ]

\paragraph{Pathogen identification via model-based distinguishability} Another key innovation of this thesis is a method to identify the likely causative pathogen of BRD from clinical observations, using mechanistic models and simulation-based inference. We introduced a pathogen-specific modeling framework in which distinct mechanistic epidemic models were formulated for different BRD etiological agents – specifically, Orthopneumovirus (BRSV), Mannheimia haemolytica, and Mycoplasmopsis bovis. Using approximate Bayesian computation combined with multinomial logistic regression as a model selection tool, we were able to discriminate among these candidate pathogen models based solely on early symptomatic trajectories. This approach achieved high identification accuracy (93\% on average, with individual pathogen identification rates of about 87–96\%). Such performance demonstrates that even when different pathogens cause clinically overlapping respiratory syndromes, their dynamical “fingerprints” in outbreak data can be teased apart with the right analytical approach. This contribution is particularly noteworthy because coinfections and similar clinical presentations are common in BRD \cite{Gaudino2022}, making targeted interventions difficult in practice. By focusing on model distinguishability – ensuring each pathogen’s model produces sufficiently unique patterns – our method provides a data-driven way to infer the likely infection cause. This is an important step toward pathogen-specific decision support. Unlike traditional diagnostic tests which might require lab work or specific assays for each pathogen, our approach uses routine observational data (e.g. clinical scores over time) to probabilistically identify the pathogen. This opens the door for earlier and more tailored treatments. In summary, we demonstrated a novel use of simulation-based inference for epidemiological model selection in an animal health context, aligning with emerging applications of ABC in infectious disease modelling \cite{beaumont2019abc}.

\paragraph{Coupling with bio-economic modelling and decision support} We extended our framework beyond biological predictions by integrating economic analysis, thereby linking epidemiological outcomes to tangible farm management metrics. Specifically, we coupled the outputs of our mechanistic BRD models (such as predicted number of cases under different interventions) with a farm profitability model that accounts for treatment costs, animal performance, and other economic factors. Through this integration, we evaluated the real-world impact of using pathogen-informed strategies versus conventional blanket treatments. The results indicated that tailoring interventions to the identified pathogen could substantially reduce antimicrobial usage – by approximately 44\% in our simulations – without sacrificing economic performance. In fact, the optimized, information-driven strategy slightly increased net profit (~1\% higher) compared to traditional empirical treatment regimens. These findings carry practical significance. They suggest that better diagnostic-prognostic information can enable win-win scenarios in livestock health: improving animal welfare and public health (through judicious antibiotic use) while maintaining or even enhancing farm profitability. This bio-economic coupling illustrates the concrete benefits of our hybrid methodology. It moves the contribution from a purely methodological realm into one that resonates with industry and societal goals, like combating antimicrobial resistance (AMR) in agriculture \cite{lhermie2019antibiotic}. By quantitatively showing that informed decisions can reduce antibiotic use with minimal economic penalty, our work provides evidence in favour of precision medicine approaches in veterinary practice. Moreover, incorporating economic considerations forces the model to focus on outcomes that matter to farmers and stakeholders, enhancing the relevance of our research for real-world adoption. This interdisciplinary integration of epidemiology and economics is still uncommon; thus, our thesis contributes a template for how to merge disease modelling with cost-benefit analysis to guide actionable recommendations.

\paragraph{Structured modularity and methodological scalability} Finally, an essential contribution of this thesis is the modular design of the hybrid modelling framework, which emphasizes clear separation between components and hence greater interpretability and flexibility. We deliberately maintained independent modules for (i) sensor-based diagnosis via deep learning, (ii) disease progression and prognosis via mechanistic models, and (iii) outcome evaluation via economic modelling. This weak modularity (weakly coupled) means that each module can be developed, fine-tuned, and validated by domain experts relatively independently – for example, veterinarians and epidemiologists can focus on improving the mechanistic model or its parameters, while computer scientists can refine the deep learning model, without constantly retraining a monolithic system. This is in contrast to end-to-end integrated approaches like EAAMs, which entangle data-driven and mechanistic components into a single architecture. The modular design also eases adaptation to new contexts or updates. For instance, if a new diagnostic sensor becomes available or a new pathogen emerges, one can update or swap out the relevant module (diagnostic or mechanistic) without overhauling the entire system. This feature is especially attractive in agriculture settings where conditions vary widely: the framework could be reconfigured for a different species or management system by exchanging modules while preserving the overall architecture. This emphasis on modular, plug-and-play components aligns with software engineering best practices and is conducive to multi-disciplinary collaboration. Different teams (data scientists, veterinarians, economists) can work in parallel on their piece of the puzzle, which is crucial in an interdisciplinary project. In summary, the thesis not only delivered specific models and results, but also a methodological template for hybrid modelling that is interpretable, extensible, and generalizable. This approach could help bridge the gap between experimental AI models and practical decision-support tools in agriculture, where stakeholder trust and adaptability are paramount \cite{wolfert2017big}.

% [remark: explain precisely how our approach supports interpretability],
% [remark: give examples of adaptation to diverse contexts. Answer: conduite en bande porcine (réf à vianney sicard et expliquer rapidement ses travaux de modélisation multi-agent multi-echelle en epidemiologie: abstract de sa thèse ?)]

% \subsection{Practical contributions}

%-----------
%	SOUS-SOUS-SECTION 
%------------
% \paragraph{Data collection}. The collection of data... 




\section{Limitations}

Despite the above contributions, several limitations of our current work must be acknowledged. These limitations point to areas where further research and development are needed, and they temper the interpretation of our results. We discuss the main bottlenecks in turn, focusing on data inputs, model coupling, validation, and practical deployment challenges. It should also be noted that some of these limitations arise inherently from the interdisciplinary nature of the project – spanning animal health, machine learning, and farm management – which requires balancing competing considerations.

\paragraph{Data input limitations (sensor and observations} A fundamental limitation lies in the reliance on thoracic ultrasound (TUS) as the primary sensor input for the diagnostic module. While lung ultrasound imaging was chosen for its practical relevance (it provides a direct non-invasive view of lung lesions and consolidation in BRD cases), it only captures one aspect of the animal’s health state. Pulmonary ultrasound, as valuable as it is, offers an incomplete picture of respiratory disease. For example, severe lung lesions can sometimes be missed if they do not contact the pleura or if they occur in lung regions not accessible to ultrasound scanning. Studies have shown that traditional clinical examinations like auscultation often fail to detect such lesions altogether, and TUS is much more sensitive in that regard \cite{buczinski_comparison_2014}. In feedlot cattle, 
Timsit \cite{timsit_association_2019} demonstrated that the maximal depth and area of lung consolidation visible on ultrasound at the time of diagnosis are significantly associated with increased risk of BRD relapse and with reduced weight gain. This evidence underpins our use of ultrasound as a prognostic indicator. Likewise, on-farm studies \cite{ollivett_-farm_2016} have advocated TUS as a useful tool to identify poor prognostic signs such as extensive lung lobe consolidation or abscessation in calves, which can guide culling or intensified treatment decisions. However, focusing on ultrasound alone means our diagnostic system could overlook clinical signals of BRD that manifest in other modalities (e.g. fever, coughing, nasal discharge, or behavioral changes). For instance, a Scottish study in sheep found that relying on auscultation alone missed many cases of pneumonia that ultrasound or necropsy would catch, indicating that each modality has blind spots. In our case, the limitation is that using only ultrasound-based features might lead to false negatives (disease not detected if lesions are not visible on the pleural surface) or false positives (lesions due to past infection or other causes). This in turn would affect the accuracy of both the diagnosis and the downstream prognostic recommendations. In a real veterinary setting, a clinician examines the animal holistically – looking at physical demeanour, nasal/ocular secretions, listening for coughs or abnormal lung sounds, measuring temperature, etc. Our current approach does not yet incorporate these additional data streams. Therefore, a more multimodal sensing strategy is warranted. Combining multiple sensors (visual, acoustic, thermal, etc.) could give a more complete view of the disease state and mitigate the reliance on any single observation type. The need for multimodal data integration is underscored by evidence that certain BRD cases present predominantly with behavioural changes (e.g. feed intake reduction) or audible symptoms (frequent coughing) that might precede ultrasound-detectable lesions \cite{carpentier2018automatic}. In summary, the limitation is not the ultrasound modality per se – which is in fact quite informative – but the exclusivity of its use. Expanding the observational input to include, for example, automatic cough monitors, could improve detection sensitivity and specificity. Our current dataset was also relatively limited in size and scope (few farms and conditions), which might limit the generalizability of the trained diagnostic model; larger and more diverse datasets are needed to ensure the model’s robustness across different herd management conditions.

\paragraph{Proxy-based coupling and uncertainty modelling} Another methodological limitation concerns how the deep learning outputs are integrated into the mechanistic model – what we termed a proxy-based hybrid approach. We used the deep learning model to generate a proxy indicator (the probability of infection in the group) which then feeds into the mechanistic simulation. While intuitive, this coupling can be fragile. It assumes that the learned proxy is a reliable summary of the complex infection state, and any error or bias in the proxy will propagate to the prognosis. We partially addressed this by incorporating the model’s uncertainty (via variational Bayesian methods) into the coupling: uncertain predictions were down-weighted during mechanistic calibration. However, the uncertainty quantification method itself has limitations. We relied on Monte Carlo dropout to approximate Bayesian uncertainty in the deep network \cite{gal_dropout_2016}. This method provides an estimate of model uncertainty and has the advantage of easy implementation, but it optimizes for average-case performance and does not guarantee calibrated uncertainty intervals \cite{matiz2020conformal}. In practice, we observed that our Bayesian neural network sometimes remained over-confident or under-confident in certain scenarios. For example, some lung ultrasound videos classified as positive (diseased) with high confidence turned out to be false alarms, partly because the ground-truth labeling by veterinarians is subjective and can be ambiguous (there is no perfect gold standard test for subclinical BRD). The Bayesian neural network’s predictive intervals did not always capture these ambiguities – a limitation because it means that simply having a high model confidence isn’t a foolproof indicator of correctness \cite{kendall2017uncertainties}. Moreover, the Monte Carlo dropout approach can become computationally expensive and may not scale well to more complex deep learning architectures. Recent advances like Transformers \cite{vaswani2017attention} have shown superior performance in many pattern recognition tasks, including medical imaging, but applying Monte Carlo sampling to such large models would be costly and may still yield poorly calibrated uncertainties. In summary, while our incorporation of uncertainty is a strength of the thesis, it is not the final word on the matter. The limitation is that our current uncertainty modelling may not fully guarantee that the “right” decisions (e.g., whether to trust a particular model prediction) are always made. In future iterations, alternative uncertainty quantification techniques (such as conformal prediction to generate guaranteed coverage prediction sets, or Bayesian neural networks with better priors) should be explored to overcome this limitation. Additionally, our hybrid model currently treats the deep learning output as a static proxy; a tighter integration (for instance, a joint inference over parameters of both models) could potentially improve coherence between diagnosis and prognosis, though this comes at the cost of a much more complex inference procedure.

\paragraph{Validation using simulated vs real field data} Some of our findings, particularly those related to optimal control strategies and pathogen-specific interventions, were derived from simulated outbreak scenarios rather than extensive field trials. This reliance on simulation is a practical necessity – it would be infeasible to experimentally trial different pathogen-specific interventions on real farms within the PhD timeline – but it constitutes a limitation in terms of validation. We showed theoretically (in Chapter 3) that the mechanistic model, when given early infection data, can differentiate between pathogens and inform decisions like whether to use a virus-specific treatment or a bacterium-specific antibiotic. These simulations included realistic stochastic variability and indicated significant potential benefits (less antibiotic use, maintained performance). However, real-world BRD outbreaks can be more complex. Multiple pathogens often circulate simultaneously or sequentially in the same group of animals \cite{Gaudino2022}, and subclinical infections can go undetected. The interactions between co-infecting agents (viral and bacterial) may alter disease dynamics in ways not fully captured by our set of discrete pathogen-specific models. For instance, concurrent infections could lead to atypical progression or different treatment responses that the model, which assumes a single dominant pathogen at a time, might not predict. Furthermore, farmer interventions (such as metaphylactic antibiotic treatment or vaccination) in real settings are not as controlled as in our simulations, potentially introducing deviations from model assumptions. In short, there is a gap between simulated performance and field performance of the system. This thesis did not include a longitudinal field trial to empirically confirm that using our hybrid diagnosis-prognosis system leads to better outcomes than status quo decisions. The lack of field validation means that conclusions about management benefits should be interpreted cautiously. For example, the predicted 44\% reduction in antibiotic use assumes perfect adherence to model recommendations and accurate pathogen identification by the model. In practice, there may be cases where the model’s recommendation is not followed or is misinformed by unusual data, which could reduce the realized benefit. This limitation points to the need for future empirical studies: deploying the system on farms to measure its impact on decision-making, disease outcomes, and economic returns. Until such validation is done, our results remain promising indicators rather than proven outcomes. We have taken steps to ensure realism in simulations (e.g., including variability and noise), but empirical calibration against real outbreak data is needed to fine-tune model parameters and to build confidence in the system’s recommendations under practical conditions.

\paragraph{Interdisciplinary and deployment challenges} A further set of limitations arises from the interdisciplinary scope of the project, which brings together expertise from veterinary science, computer science, and agricultural engineering. One issue was the definition of ground truth for training and evaluation. In the absence of a single definitive diagnostic test for BRD, we relied on veterinary clinical assessments (symptom scoring, etc.) as proxies for ground truth labels (infected vs. healthy). However, even experienced veterinarians can disagree on borderline cases, and as noted, BRD is a syndrome with no unique biomarker. This label noise likely impacted the training of our diagnostic model – a limitation common in medical AI applications where labels are imperfect \cite{roy2019weak}). Disagreements over what constitutes “disease presence” introduced uncertainty not only in the model but also among the team members interpreting results. Better approaches to handle ambiguous labels (such as probabilistic labels or consensus labelling) were not fully implemented in this thesis. Another challenge was meeting the diverse expectations of stakeholders. Farmers ideally want a tool that is easy to use, provides clear recommendations, and improves their bottom line. Veterinarians want the tool to be trustworthy, aligning with their clinical intuition and not missing critical cases. Industry partners (e.g., ag-tech companies) are concerned with feasibility: is the system fast and reliable enough, cost-effective, and integrable into farm workflows? Our prototype system, while scientifically promising, is still a proof-of-concept. Usability and integration limitations include the need for continuous data connectivity (ultrasound data had to be uploaded to a server for analysis), the time taken to run analyses (which currently may not be real-time in field conditions), and the requirement for relatively sophisticated hardware (ultrasound machines, GPU servers for the AI model, etc.). In a practical deployment, decisions such as on-device (edge) vs. cloud computing must be addressed to ensure timely feedback to the farmer. We partially addressed this by setting up a basic communication pipeline: for example, audio recordings were transmitted from on-farm sensors to a central server for processing. However, this is only an initial step. We did not fully optimize the system for latency or energy consumption – important factors if devices are battery-powered or connectivity is intermittent. Additionally, the current system would require a technician or vet to perform ultrasounds on calves, which is an extra labour step. Automating or simplifying data collection (perhaps using fixed sensors or self-service kiosks for animals) is another practical hurdle. In summary, the limitation here is that significant work remains to turn the research prototype into a deployable product. This includes improving the user interface, ensuring the analysis can run with minimal user intervention, establishing reliability and fail-safes (what if a sensor fails or gives implausible data?), and conducting training for end-users. The interdisciplinary nature of the project, while a strength, also meant we had to navigate different terminologies and priorities, which occasionally slowed progress or led to compromises in design. These challenges emphasize that technical innovation alone is not sufficient – understanding the context of use is crucial. The thesis lays the groundwork, but a concerted effort with input from farmers, veterinarians, and engineers will be needed to refine the system. Only by doing so can we overcome the last-mile limitations and ensure the tool is accepted and effective in real decision-making scenarios.



% Phil cott PR, Collie D, McGorum B, Sargison N. Relationship between thoracic auscultation and lung pathology detected by ultrasonography in sheep. Vet J 2010;186:53–57
%-----------------------------------
%	SECTION 
%-----------------------------------
\section{Perspectives}

Building on the contributions and acknowledging the limitations discussed, several avenues for future work emerge. These perspectives encompass methodological enhancements and broader explorations to increase the impact and applicability of our hybrid AI-epidemiological framework. We outline key future directions in terms of model improvements, uncertainty handling, inference techniques, decision-making integration, comprehensive validation, and domain transfer. Each of these is aimed at addressing current limitations and pushing the boundary of what such hybrid systems can achieve in animal health management.


\paragraph{Integration of multimodal deep Bayesian mechanistic models} A natural extension of this work is to incorporate multiple sensor modalities (e.g. visual and auditory data) into a unified diagnostic-prognostic framework. In current practice, veterinarians assess BRD using a combination of visual cues (signs of fatigue, nasal discharge, posture) and auditory cues (frequency and nature of coughing, lung sounds) alongside ultrasound findings. Our system could be expanded to emulate this holistic assessment by fusing data from cameras and microphones in addition to ultrasound. Recent developments in deep learning provide methods for audio-visual learning that could be leveraged to this end \cite{zhu2020deep}. For example, an audio analysis model could continuously monitor cough sounds in the barn, which are a strong indicator of respiratory distress \cite{10.1371/journal.pone.0123111}. By aligning cough event data with visual health indicators and ultrasound results, we might improve early detection sensitivity – catching cases that ultrasound-alone diagnostics could miss. However, integrating heterogeneous data streams poses significant challenges: synchronization of signals (timing coughs to specific animal observations), dealing with noise (barn acoustics can be poor, and visuals can be affected by lighting or occlusion), and learning an effective joint representation. Advanced techniques like audio-visual attention mechanisms or representation learning could be applied so that the model learns cross-modal features (e.g., linking an increase in cough frequency with subtle changes in animal posture or ultrasound anomalies). Audio-visual separation methods might help isolate meaningful sounds (coughs vs. background noise) in realistic farm environments 
\cite{carpentier2018automatic}. Future research may implement a multimodal Bayesian mechanistic model, wherein each modality contributes to an overall belief about the herd’s health state, and uncertainties from each sensor are combined. Importantly, incorporating new modalities will require new data – potentially a large labeled dataset of concurrent audio, video, and ultrasound recordings of calves. Obtaining such data is non-trivial, and data annotation becomes a bottleneck. Here, techniques like weakly-supervised learning and active learning could prove invaluable. Rather than exhaustively labeling every instance (which is labor-intensive and prone to human error, one could use active learning to have the model query a human expert for labels on only the most informative or uncertain cases \cite{gal2017deep}. Semi-supervised learning could further allow the model to learn from the abundance of unlabeled sensor data available on farms (e.g., long audio recordings) combined with the limited labeled examples. By coupling these strategies, future systems might build robust multimodal classifiers with far less manual labeling than traditionally required. In summary, expanding to a multimodal, sensor-fusion approach is promising for enhancing diagnostic accuracy and early detection of BRD. This research direction requires methodological advances in multi-sensor data fusion and pragmatic solutions for data collection and annotation on farms, but the payoff would be a more sensitive and veterinarian-like AI system that captures the full spectrum of disease indicators.

\paragraph{Advanced uncertainty quantification and explainability} This thesis took a first step toward uncertainty-aware AI in agriculture using variational Bayesian methods. A future direction is to explore complementary or alternative approaches for uncertainty estimation that provide stronger guarantees and interpretability. One such approach is conformal prediction, a framework that can wrap around any model to produce prediction sets with a guaranteed coverage probability \cite{Angelopoulos2021}. Unlike Monte Carlo dropout, conformal prediction does not rely on Bayesian assumptions; instead, it uses past prediction errors to determine confidence sets for new predictions with rigorous statistical coverage (e.g., “with 90\% probability, the true outcome lies in this set”). Integrating conformal prediction into our deep learning diagnostic could yield more actionable uncertainty estimates – for instance, instead of outputting a single label, the system might output a set of likely diagnoses or a range for the number of infected animals, with an associated confidence level. Matiz and Barner \cite{matiz2020conformal} highlighted that conformal methods can complement Bayesian neural networks by providing calibrated uncertainty measures alongside the model’s point estimates. For our hybrid model, a potential research avenue is a hybrid uncertainty approach: use Bayesian methods (like dropout) to maintain high average accuracy and sharpness of predictions, but apply conformal wrapping to ensure the uncertainty intervals are reliable (e.g., capturing the true outbreak size 95\% of the time). This could result in, for example, prediction intervals for the future number of BRD cases that farmers and vets can trust to a specified probability. The practical challenge will be computational: conformal prediction typically requires an additional calibration step and may need plenty of past data for validation. Moreover, applying conformal prediction in a streaming data context (where the model is used continuously on new farms or new seasons) is an open area of research. Nonetheless, the benefit would be decision-theoretic robustness – users of the system could be presented with worst-case and best-case scenarios within a confidence bound, which might encourage more cautious and risk-aware decisions. Another aspect for future work is explainability of the model’s predictions. In high-stakes domains like animal health, users are more likely to trust and adopt AI if it can explain its reasoning. Techniques such as feature attribution (e.g., highlighting which part of an ultrasound image led to a positive diagnosis) or case-based reasoning (e.g., “this farm’s data closely resembles past outbreak X”) could be integrated so that the system not only predicts but also justifies its predictions. The Bayesian nature of our approach could be leveraged to produce explanations like “the model is only 50\% confident because the inputs are unlike anything seen before,” which itself is useful information. In summary, future research should aim to enhance the trustworthiness of the hybrid model through better uncertainty quantification (possibly combining Bayesian and conformal methods) and improved explainability. This will ensure that as the model’s capabilities grow (e.g., multimodal input), its outputs remain transparent and calibrated – qualities that are essential for real-world deployment and user acceptance.

\paragraph{Robust simulation-based inference and model parametrisation} We identified that our use of approximate Bayesian computation (ABC) for pathogen model selection is promising, but there is room to improve the efficiency and robustness of the inference. One path is to employ Sequential Monte Carlo ABC (ABC-SMC) algorithms or other advanced simulation-based inference techniques. ABC-SMC iteratively focuses simulation effort on parameter regions with higher posterior likelihood, which can greatly improve efficiency over the basic ABC rejection approach we used. By adopting an ABC-SMC approach, we could better explore complex parameter spaces, especially if we integrate more parameters or more complex mechanistic models (for example, models capturing coinfection dynamics). More robust inference could lead to finer discrimination between similar pathogen models or more precise parameter estimates for each model, thus improving the fidelity of forecasts. Additionally, recent developments like using machine learning surrogates within ABC (e.g., regression adjustments or neural density estimators) could be leveraged. For instance, replacing the simple multinomial logistic regression post-ABC with a trained classifier or using distance-learning approaches
\cite{jagalur2021abc} might improve the power to distinguish models using high-dimensional summary data. Another consideration is joint parameter and model inference. In our work, we first identified the most likely pathogen model and then used that model’s best-fit parameters for forecasting. A more rigorous Bayesian approach would be to treat the pathogen identity as just another parameter to infer – effectively averaging predictions over all possible models weighted by their posterior probability (a form of Bayesian model averaging). This could potentially account for uncertainty in pathogen identification in the forecasts (e.g., if two pathogens are similarly likely, the forecast might combine both possibilities). The downside is computational complexity, but ABC-SMC methods are well-suited to approximate this kind of joint inference \cite{beaumont2019abc}. We also note that alternative inference paradigms, like synthetic likelihood or Hamiltonian Monte Carlo for simulator-based models, are emerging and could be tested on our problem to see if they offer gains in speed or accuracy. In sum, the perspective here is to stress-test and refine the inference engine of our hybrid model. By exploring more advanced ABC variants or other likelihood-free inference techniques, future work can ensure that the model calibration and pathogen identification remain robust even as model complexity grows or as we move to more challenging datasets. Such improvements would strengthen the foundation of the entire hybrid approach, since accurate inference is critical to everything from generating trustworthy predictions to learning from new data.

\paragraph{Coupling to robust decision-making frameworks} While we incorporated a basic economic analysis, future research can deepen the integration between epidemiological predictions and decision optimization. For example, instead of outputting a single “optimal” intervention strategy based on average outcomes, the system could use the uncertainty in its predictions to suggest strategies that are robust to worst-case scenarios. This aligns with concepts in decision theory where one seeks solutions that perform acceptably under a range of possible futures, not just the most likely future. In practice, this could mean using the posterior distribution of the mechanistic model parameters (or the predictive distribution of future cases) to evaluate interventions: e.g., choosing a treatment plan that maximizes expected profit and minimizes the risk of catastrophic loss in a bad outbreak. Techniques like Value of Information analysis could also be employed to determine if gathering more data (say, doing an extra diagnostic test) is worth the effort before making a treatment decision. Our current analysis already hinted at interventions (like selective antibiotic metaphylaxis or enhanced biosecurity) and their outcomes, but an explicit decision model would allow one to simulate policies over an entire season or production cycle. Importantly, any such decision-support extension should be evaluated not just on model outputs but on how it impacts real objectives (antibiotic use, cost, and animal welfare). Future collaboration with economists and ethicists might also consider incorporating externality costs (e.g., the societal cost of antibiotic resistance) into the decision-making objective, potentially guiding farmers towards choices that are globally optimal, not just farm-optimal. In summary, the perspective is to evolve our system from a predictive tool into a prescriptive tool – one that can recommend actions under uncertainty. Doing so will likely involve robust optimization techniques and further interdisciplinary work, but it directly addresses the end-goal of this research: not only to predict disease, but to improve disease control outcomes in practice.

\paragraph{Closing the diagnostic-prognostic loop with real-world trials} As noted in the limitations, a critical next step is to validate and refine the integrated system through field studies and deployment pilots. One future research avenue is to implement the full pipeline – from sensor data acquisition to diagnosis to pathogen identification to recommended intervention – on a set of commercial farms, in close collaboration with veterinarians, and monitor outcomes. This would effectively test the “deep mechanistic model with pathogen-specific expert selection” in a real-world setting. Concretely, we envision using the lung ultrasound video dataset (and potentially other sensors) collected in Chapter 2 as the foundation to develop a system that, for each new batch of calves, automatically analyzes incoming sensor data, produces a probabilistic diagnosis for each calf (with uncertainty), and then uses that to infer the most likely pathogen-specific scenario via numerical solvers (e.g., ABC-SMC as discussed). This inference could trigger specific control recommendations (for example, “outbreak likely viral – consider anti-viral and avoid antibiotics unless secondary infection signs appear”). Validation of this approach would involve biological ground truthing: for instance, collecting nasal swabs or blood samples from calves to identify the actual pathogen(s) via PCR or culture, and comparing those to the model’s inferred pathogen. Additionally, one would track metrics like antibiotic usage, illness recurrence, and weight gain in groups managed with model support versus control groups managed by standard practice. Key performance indicators would be whether the model-informed groups use significantly less medication while maintaining health and performance. Any discrepancies or failures observed during such trials would provide invaluable feedback to improve the model (e.g., if the model systematically misses a particular scenario, that model structure might need extension). We should also explore the system’s user experience during these trials: how easily can farm staff and vets interact with it? do they trust the recommendations? By iterating with user feedback, the model and interface can be adjusted (perhaps simplifying outputs or adding explanation features as discussed). Ultimately, such applied research will help transition our framework from a concept to a tangible tool. A successful field demonstration would not only prove out the efficacy of our approach but also possibly reveal new research questions (for example, how to rapidly adapt the model to a farm experiencing an atypical outbreak, or how to incorporate farmer intuition into the AI feedback loop). This “last mile” research is often where interdisciplinary projects either flourish or flounder, so careful experimental design and stakeholder engagement will be paramount. The knowledge gained from these real-world deployments will also inform any necessary regulatory approvals or guidelines for AI in veterinary practice, an emerging area that we have not yet touched but will be important for widespread adoption.

\paragraph{Adaptation to other domains and scalability} Finally, the modular and interdisciplinary nature of our methodology lends itself to transfer and generalization to other infectious disease management problems in agriculture. Future work could test the adaptability of the hybrid model in different contexts, thereby evaluating its generality. For example, one could apply a similar deep learning + mechanistic modelling approach to swine respiratory disease in farrow-to-finish pig operations. Efforts have already been made in modelling porcine infectious diseases with multi-scale agent-based models \cite{Sicard2022},and integrating sensor data (such as cough monitors for pigs or thermal cameras for fever detection) with those models could improve early outbreak detection in swine just as we aimed to do in cattle. Another potential application is in dairy herd health monitoring beyond BRD – for instance, combining sensor-based lameness detection with a mechanistic model of disease spread in a barn to forecast and control a foot-and-mouth disease outbreak. The crop farming sector might also benefit: one could envision using imaging sensors (drones or satellites detecting crop stress) feeding into mechanistic models of pest or disease spread in fields, thereby informing integrated pest management strategies. Coupling mechanistic models in crop protection with ML that interprets sensor images of crop canopies could parallel our work in the plant domain. The challenge in transferring the methodology will lie in customizing each module to the new domain while preserving the overall architecture. The diagnostic AI would need retraining on the new sensor data, the mechanistic model would need to capture the relevant epidemiology (or pest ecology), and the economic module would change to whatever metrics matter (e.g., crop yield or market value). However, none of these require fundamentally new algorithmic development – they are matters of implementation and training, which speaks to the scalability of the approach. We anticipate that as long as the disease system has (1) some form of sensor that provides early indicators, and (2) a mechanistic understanding that can be modelled, our hybrid approach can be applied. One lesson from our work that will be valuable in other domains is the importance of modularity: keeping the components decoupled means a new team of experts can replace or modify one part (say, the pig disease model) without needing to rewrite the entire pipeline. In pursuing these new applications, collaboration with domain experts (swine veterinarians, plant pathologists, etc.) will be crucial to ensure the models are biologically sound. Additionally, computing infrastructure and data management need to scale – a successful deployment in one sector could mean data coming from hundreds of farms, requiring robust cloud support and perhaps automated model updating as more data flows in. These are engineering challenges but foreseeable ones. 
In conclusion, by validating and refining our approach in other livestock or agricultural health contexts, future research can test the universality of the hybrid AI-epidemiological modelling paradigm. If successful, it would mark a significant advance in digital agriculture, providing a general blueprint for smart disease surveillance and control across different farming systems. This would amplify the impact of our initial research, contributing not only to cattle health management but broadly to the sustainability and efficiency of animal and crop health interventions in the era of precision agriculture.


% pour faire références à la thèse d'Hassan, on peut également regarder tout les questions autours du comment disposer les capteurs pour optimiser la collecte d'informations riche et miniser la consommation d'energie (regarder la problèmatique de Hassane). 

% \section{Conclusion}

% Il reste à constuire une bdd unifiés pour faciliter la publication et l'utilisation des données multimodales

% In conclusion, the methodological progress of this thesis demonstrates how to integrating deep learning with mechanistic epidemiological models—grounded in explicit uncertainty quantification—significantly improves both diagnostic accuracy and epidemiological prognosis from sensors observations. 


% By explicitly addressing the complexities and uncertainties inherent in sensor-based disease diagnostics, our approach provides practical, actionable solutions for more effective and sustainable livestock disease management strategies.

% remarks [le but est d'avoir construit de nouvelles connaissances et ouvert de nouvelles questions, pas juste d'avoir une sorte de réponse binaire]

\newpage\thispagestyle{empty}