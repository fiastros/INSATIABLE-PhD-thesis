\section{[In french] Résumé grand public}

La gestion des maladies infectieuses en élevage bovin s’inscrit dans un système d’une grande complexité, en raison de facteurs intrinsèques (par exemple la virulence des pathogènes et la sensibilité des animaux) et extrinsèques (pratiques d’élevage, conditions environnementales) en constante évolution. Face à cette complexité, l’intelligence artificielle (IA) émerge comme une approche prometteuse pour modéliser les dynamiques épidémiques et anticiper leur évolution via des simulations, fournissant ainsi des outils d’aide à la décision aux éleveurs, vétérinaires et autres acteurs. Parallèlement, l’essor de l’agriculture de précision se traduit par le déploiement de capteurs capables de surveiller en continu des variables physiologiques individuelles et des conditions environnementales, et de générer des alertes rapides (en quelques heures ou jours) sur des événements critiques tels que le vêlage, les chaleurs, le bien-être ou la santé des animaux.

Néanmoins, s’agissant des maladies infectieuses en élevage, ces alertes issues de capteurs souffrent d’une faible spécificité et génèrent un taux élevé de faux positifs. Ces fausses alarmes imposent une charge mentale importante aux éleveurs, qui finissent soit par les ignorer faute de pertinence, soit par réaliser des interventions inutiles et coûteuses. Ainsi, concevoir des méthodologies innovantes capables de transformer ces signaux peu spécifiques en recommandations précises et actionnables sur des horizons de temps plus longs (plusieurs jours à plusieurs semaines) reste un défi de recherche ouvert. C’est précisément ce défi que cette thèse entreprend de relever, en posant la question centrale: comment exploiter efficacement les observations issues de capteurs pour étudier les maladies infectieuses et appuyer des décisions éclairées en élevage bovin ?

Pour y répondre, cette thèse postule qu’il est optimal d’intégrer des approches d’IA complémentaires, en l’occurrence les réseaux de neurones profonds et les modèles épidémiologiques mécanistes. La stratégie proposée capitalise d’une part sur la capacité des modèles mécanistes à représenter explicitement les processus épidémiques aux différentes échelles de temps en mobilisant les connaissances vétérinaires et biologiques, et d’autre part sur la puissance du deep learning pour extraire automatiquement des descripteurs pertinents à partir de données massives et hétérogènes issues des capteurs (images, sons, etc.). Le couplage de ces deux approches doit permettre une meilleure intégration des observations réelles (issues des capteurs, mais aussi des retours d’éleveurs et de vétérinaires) dans les prédictions épidémiques à court terme comme à moyen et long terme. 

La problématique choisie pour appliquer cette méthodologie est celle des maladies respiratoires bovines (en anglais *Bovine Respiratory Disease*, BRD) chez les jeunes bovins de boucherie. La BRD constitue en effet le principal problème de santé dans les ateliers d’engraissement, avec des conséquences sanitaires et économiques majeures. Elle ralentit la croissance et la productivité des animaux, induit des frais vétérinaires et médicamenteux importants, et provoque une mortalité non négligeable (environ 3\% en moyenne). C’est également la première cause d’usage d’antibiotiques en élevage bovin, avec près de 20\% des bovins à l’engraissement recevant un traitement contre la BRD. 

L’étiologie de la BRD est multifactorielle, résultant d’interactions complexes entre de nombreux facteurs. Côté animal, la susceptibilité dépend de la race, de l’état immunitaire et de la co-infection par divers agents pathogènes (notamment les bactéries *Mannheimia haemolytica* et *Pasteurella multocida*, ou des virus comme le virus respiratoire syncytial bovin), dont les interactions restent encore mal élucidées. Côté élevage et environnement, des facteurs de risque tels que le stress du transport, la densité des animaux, la gestion de l’alimentation, les conditions de logement, les protocoles de biosécurité ou le climat influencent fortement l’apparition et la gravité de la maladie. La conjonction de ces facteurs rend la prédiction et le contrôle de la BRD particulièrement hasardeux sur le terrain et complique sa modélisation épidémiologique. 

En outre, la détection précoce de la BRD s’avère particulièrement ardue. Ses manifestations cliniques initiales — toux, écoulement nasal, fièvre, anorexie, léthargie, retard de croissance — sont peu spécifiques et peuvent facilement passer inaperçues, d’autant que les bovins ont tendance à dissimuler les signes de maladie aux premiers stades (comportement de proie). Les méthodes traditionnelles de surveillance visuelle présentent ainsi une sensibilité et une spécificité limitées (de l’ordre de 60 à 65\% seulement), ce qui conduit à de fréquentes erreurs de diagnostic: des cas infectés peuvent ne pas être détectés à temps (faux négatifs), et inversement des animaux sains sont parfois traités à tort (faux positifs). Ces difficultés sont exacerbées par la pénurie de vétérinaires en zones rurales, qui restreint la possibilité d’une surveillance rapprochée et régulière des troupeaux.

La détection de la BRD pourrait néanmoins bénéficier des avancées récentes en élevage de précision, grâce au suivi continu de la santé des animaux par divers capteurs. Des accéléromètres, microphones, thermomètres connectés ou caméras permettent de mesurer des signaux physiologiques et comportementaux associés à la maladie, tels que la température corporelle, les patterns d’activité ou les sons respiratoires. Par exemple, environ 73\% des épisodes de fièvre (hyperthermie) chez des veaux à l’engraissement coïncident avec une BRD, ce qui suggère qu’une surveillance de la température peut être un indicateur utile. Plus récemment, un modèle statistique (régression logistique) exploitant des données de collier d’activité, de podomètre et de bolus intra-ruminal a atteint environ 75\% de sensibilité et 76\% de spécificité pour prédire l’apparition de signes cliniques de BRD jusqu’à 24 heures à l’avance. De même, des capteurs accélérométriques fixés sur les oreilles ont permis de détecter des changements de comportement (activité, rumination) distinguant clairement des veaux malades et sains, illustrant le potentiel des capteurs pour une détection plus précoce des maladies respiratoires bovines.

Pourtant, ces approches purement basées sur les capteurs demeurent limitées par l’utilisation de modèles d’apprentissage automatique traditionnels, peu aptes à extrapoler au-delà des situations déjà observées. Leur capacité à prévoir l’évolution d’une épidémie dans des contextes différents ou sur le long terme est réduite, ce qui limite leur utilité pour orienter les décisions de gestion sanitaire à moyen ou long échéance. Par ailleurs, il est éthiquement et pratiquement impossible de rassembler des données exhaustives couvrant tous les scénarios de BRD (en particulier les cas sévères) dans les conditions réelles d’élevage, ce qui freine inévitablement les méthodes purement empiriques. 

En complément des capteurs, la modélisation épidémiologique mécaniste apporte une solution prometteuse pour dépasser ces limites observationnelles. Par exemple, une étude *in silico* récente a identifié des stratégies optimales de gestion de la BRD fondées sur des alertes capteurs. Cependant, les modèles mécanistes employés étaient calibrés à partir de données de la littérature vétérinaire plutôt que de données empiriques issues du terrain, introduisant des incertitudes quant à leur validité pratique. De plus, certaines solutions actuelles de détection reposent sur des dispositifs invasifs ou coûteux, d’où l’importance d’explorer des alternatives non invasives et abordables (analyse automatique d’images, d’enregistrements audio, etc.) pour une adoption à large échelle. 

La stratégie de la thèse s’inscrit dans cette perspective et mise sur une forte interdisciplinarité en couplant apprentissage profond et modélisation mécaniste pour améliorer la détection et la prédiction de la BRD. Ce travail a été mené dans le cadre d’une convention CIFRE, en partenariat étroit entre la société Adventiel et l’organisme de recherche INRAE, ce qui a favorisé le lien entre recherche académique et application industrielle. Adventiel, entreprise française spécialisée dans les solutions numériques pour l’agriculture, a apporté son expertise en intelligence artificielle appliquée (vision par ordinateur, analyse de signaux) ainsi que son infrastructure technologique (serveurs de calcul, stockage de données) pour la collecte et le traitement des observations issues des capteurs. 

Du côté de l’INRAE, l’unité de recherche BIOEPAR (équipe DYNAMO) a fourni le cadre de modélisation mécaniste avec l’outil EMULSION — une plateforme à base de systèmes multi-agents et d’un langage dédié facilitant le développement de modèles épidémiologiques — et a partagé un premier modèle mécaniste de la BRD servant de base à cette étude. L’expertise vétérinaire et épidémiologique de BIOEPAR sur la BRD (connaissances cliniques, immunologiques et socio-économiques) a largement orienté la conception du modèle et l’interprétation des données. En parallèle, l’équipe Démécologie (unité IGEPP, INRAE) a contribué des méthodes statistiques avancées pour l’estimation des paramètres, la quantification des incertitudes et l’inférence bayésienne, afin d’aborder les défis liés à l’ajustement des modèles sur les données réelles. La synergie de ces compétences variées – modélisation, deep learning, statistique, expertise vétérinaire de terrain – confère à cette recherche un caractère original et illustre la convergence de multiples expertises autour d’un même objectif. 

D’un point de vue empirique, une composante importante de la thèse a été la constitution d’un jeu de données multimodal inédit sur la BRD en conditions d’élevage réel. Cette collecte a eu lieu dans le cadre du projet collaboratif SEPTIME (Carnot «France Futur Élevage»), impliquant l’INRAE (BIOEPAR) et l’Institut de l’Élevage (Idele). Elle s’est déroulée sur neuf exploitations d’engraissement bovin réparties dans différentes régions, lors de deux campagnes correspondant aux arrivées typiques de jeunes bovins: de janvier à juin 2023, puis d’octobre 2023 à janvier 2024. Ces périodes ont été choisies car les premières semaines suivant l’introduction d’animaux dans un nouveau troupeau sont connues pour présenter un risque élevé de BRD. Sur chaque élevage, un à trois lots de 5 à 12 bovins ont été suivis pendant 30 jours dès leur arrivée. Environ 78\% des animaux étaient de race Charolaise, ce choix facilitant l’observation visuelle de certains symptômes grâce à la robe claire de ces bovins.

Le protocole de suivi mis en place combinait plusieurs capteurs et des examens vétérinaires réguliers. Chaque ferme était équipée d’une caméra vidéo fixe enregistrant un extrait de 5 minutes chaque heure en journée (de 9h à 18sh), d’un microphone synchronisé capturant les sons en parallèle de la vidéo (notamment la toux) et d’un capteur environnemental mesurant en continu la température ambiante, l’humidité, le $CO_2$ et l’ammoniac ($NH_3$). Parallèlement, les animaux ont été examinés par un vétérinaire environ tous les deux jours durant le mois suivant leur arrivée. Lors de ces visites, un examen clinique était réalisé (observation du comportement, détection de signes respiratoires tels que fatigue, écoulements oculaires ou nasaux, prise de la température rectale) et des prélèvements biologiques étaient effectués: analyses sanguines et écouvillonnages nasaux pour détecter les pathogènes respiratoires par PCR. En complément, des échographies pulmonaires ont été pratiquées à des jours prédéfinis (le jour de l’arrivée, puis les jours 5, 14, 21 et 28) afin d’évaluer visuellement l’état des poumons et de confirmer d’éventuelles lésions. L’ensemble des données collectées (vidéos, audio, mesures environnementales, observations cliniques, résultats de laboratoire et imagerie médicale) a été automatiquement transmis et stocké de façon sécurisée sur les serveurs d’Adventiel, constituant une base empirique riche et synchronisée pour les analyses ultérieures. Ce jeu de données unique, alliant signaux de capteurs et diagnostics vétérinaires détaillés, fournit un fondement solide pour répondre aux questions scientifiques tout en restant ancré dans les besoins concrets de l’élevage.

La démarche méthodologique de la thèse se décline en trois étapes complémentaires, correspondant aux chapitres principaux, afin d’apporter successivement des éléments de réponse à la question posée. (1) Tout d’abord (Chapitre 2), chaque approche a été évaluée séparément pour en établir la faisabilité et les limites propres. D’un côté, nous explorons dans quelle mesure un modèle d’apprentissage profond peut automatiser le diagnostic de la BRD à court terme à partir de données de capteurs limitées et spécifiques (en particulier l’analyse de vidéos d’échographie pulmonaire). De l’autre, nous examinons si des observations vétérinaires de terrain peuvent servir à paramétrer un modèle épidémiologique mécaniste afin de fournir un pronostic de la maladie sur le long terme. Les résultats obtenus confirment l’intérêt de chaque approche. Un réseau de neurones profond entraîné sur les séquences d’échographies pulmonaires parvient à détecter des lésions de BRD avec environ 72\% de précision, malgré les conditions d’acquisition variées sur le terrain. Parallèlement, un modèle épidémiologique mécaniste calibré à partir des observations cliniques réussit à reproduire les tendances de l’infection sur plusieurs semaines, et ce malgré le caractère parcellaire des données disponibles. Ce double constat fournit un socle empirique pour l’analyse intégrée et s’accompagne de contributions notables, telle que la constitution d’un jeu de données original d’échographies pulmonaires annotées pour la BRD – un atout précieux pour de futurs travaux en diagnostic vétérinaire assisté par IA. Il met en lumière la complémentarité des approches: les tâches de diagnostic instantané sont confiées au deep learning, excellent pour extraire des caractéristiques complexes d’images ou de sons, tandis que le pronostic à long terme est dévolu au modèle mécaniste, qui s’appuie sur les bases théoriques épidémiologiques pour simuler l’évolution de la maladie dans le temps.

(2)La deuxième étape (Chapitre 3) aborde la variabilité des agents pathogènes pouvant être en cause dans la BRD et l’impact de cette variabilité sur le pronostic. Plusieurs modèles mécanistes spécifiques peuvent en effet être envisagés selon le pathogène prédominant (par exemple un modèle calibré pour le virus BRSV, et d’autres pour les bactéries *M. haemolytica* ou *P. multocida*). Il devient alors crucial de déterminer, à partir des seuls symptômes observés, quel agent prédomine afin de choisir le modèle de prévision adéquat, et de vérifier si cette identification améliore les décisions sanitaires. Par analogie, tout comme un médecin généraliste oriente un patient vers un spécialiste approprié en fonction de ses symptômes, notre système doit pouvoir sélectionner le «modèle spécialiste» (viral ou bactérien) correspondant à la situation réelle afin d’affiner ses prédictions. 

Pour ce faire, nous avons mis en œuvre une approche bayésienne originale combinant une méthode d’inférence par ABC (*Approximate Bayesian Computation*) et une régression logistique multinomiale. Cette méthode permet de différencier avec environ 93\% d’exactitude entre plusieurs scénarios simulés de BRD, en identifiant lequel des pathogènes principaux (virus BRSV, *M. haemolytica*, *M. bovis*, etc.) correspond le mieux à la trajectoire de symptômes observée. Surtout, le fait d’intégrer la reconnaissance de l’agent causal dans le modèle conduit à des recommandations de gestion plus efficaces. À l’aide d’un modèle bio-économique simulant le fonctionnement de l’élevage, nous montrons qu’en adaptant les interventions au pathogène identifié, il est possible de réduire d’environ 44\% le recours aux traitements antibiotiques, tout en améliorant légèrement la performance économique de l’atelier d’engraissement. Ce résultat illustre concrètement l’intérêt de lier étroitement la modélisation épidémiologique aux décisions de terrain: un pronostic mieux ciblé permet des actions plus pertinentes, bénéfiques à la fois pour la santé des animaux (moins de traitements inutiles) et pour la rentabilité de l’élevage.

(3) La troisième et dernière étape (Chapitre 4) réalise l’intégration effective des deux approches (diagnostic automatique et modélisation mécaniste) dans un cadre unifié, tout en gérant explicitement les incertitudes inhérentes aux données de capteurs. L’enjeu est double : utiliser les diagnostics automatisés à court terme issus des capteurs pour informer le modèle mécaniste en vue d’un pronostic à long terme, et tenir compte de l’incertitude de ces diagnostics pour garantir la fiabilité des prévisions. Pour cela, nous avons développé une approche hybride nommée *Bayesian Deep Mechanistic* (BDM), qui intègre les prédictions d’un modèle de deep learning (appliqué notamment aux vidéos d’échographie pulmonaire) au sein d’un modèle épidémiologique de manière probabiliste. Concrètement, chaque prédiction de l’IA est associée à un degré de confiance, estimé par la technique du *Monte Carlo dropout* afin de quantifier l’incertitude du réseau de neurones. Ces diagnostics «probabilisés» alimentent ensuite le modèle mécaniste de deux manières: soit en filtrant ou pondérant les observations en fonction de leur niveau d’incertitude (de façon à ne conserver que les informations jugées fiables), soit en intégrant directement cette incertitude dans l’inférence des paramètres du modèle. Un tel dispositif améliore nettement la précision et la robustesse du système global. Par exemple, l’erreur de prévision à long terme est réduite d’environ un tiers (de 39\% à 27\% d’erreur quadratique moyenne relative) lorsque l’on tient compte de l’incertitude des données capteurs dans le modèle. Ainsi, le cadre BDM rapproche les performances d’un pronostic automatisé de celles d’un expert humain, en combinant les atouts du deep learning et des modèles mécanistes tout en atténuant leurs faiblesses respectives grâce à une gestion rigoureuse des incertitudes. Il en résulte un outil d’aide à la décision fiable, transparent et adaptable pour le suivi des maladies infectieuses en élevage.

En synthèse, les travaux menés dans cette thèse démontrent l’intérêt d’approches d’IA hybrides pour l’étude et le contrôle des maladies infectieuses en élevage, en particulier dans le contexte de l’élevage de précision. L’étude de cas sur la BRD illustre comment le croisement du deep learning et de la modélisation mécaniste permet de dépasser les limites actuelles des capteurs en santé animale : on passe de simples alertes ponctuelles et peu spécifiques à un système intégré de diagnostic automatisé, de pronostic robuste à long terme et d’aide à la décision, le tout étayé par une quantification explicite de l’incertitude. Les résultats obtenus se positionnent par rapport à la littérature actuelle en apportant tout à la fois des avancées méthodologiques (intégration bayésienne innovante, différenciation de modèles en fonction des pathogènes) et des bénéfices concrets pour l’élevage (réduction de l’usage inutile d’antibiotiques, amélioration du bien-être animal et optimisation économique). Bien que développée sur la BRD, l’approche proposée revêt un caractère générique et pourrait être transposée à d’autres maladies infectieuses en élevage (voire en santé des plantes), dès lors que des données de capteurs sont disponibles pour alimenter les modèles. Ce travail, fruit d’une collaboration étroite entre acteurs académiques et industriels, ouvre ainsi de nouvelles perspectives pour des systèmes de santé prédictive en élevage, combinant intelligence artificielle et expertise métier afin d’aider les éleveurs et les vétérinaires à prendre des décisions éclairées.
