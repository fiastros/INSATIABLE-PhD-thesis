% \chapter*{\Huge Big Title Here}
% \addcontentsline{toc}{chapter}{Big Title Here}  % Add to TOC if needed

% \chapter{Components integration - a deep mechanistic approach} % propositions de titre
\chapter{A deep mechanistic model: Grounded mechanistic model for adaptive knowledge} % Main chapter title
Identifiability and individual performances, underdetermination
pourqoui on  a pas fait la fusion ici : Ground truth qui a pu servir pour le méca mais pas suffisant pour tackler la "undertermination" of the deep learning model. 
validation des deux modèlès

%----------------------------------------------------------------------------------------
%	SECTION 
%----------------------------------------------------------------------------------------
\section{Introduction}
%-----------%-----------
%	SOUS-SECTION 
%-----------%-----------
\subsection{Elements of context}
% \subsection{Chain-of-paradigms: proxy robustness}

In our previous work, we couldn't couple the automated diagnosis from deep learning and the long-term predictions of mechanistic models.The low performance of the deep learning model (72 percent accuracy) would yield over 25 percent errors rate if used in reality, which is too much variations to be integrated within the parameterisation of the mechanistic model. We hypothesized that the errors could stem from:
\begin{itemize}
    % \item the ground truth used for training the diagnosis expert relied solely on clinical examinations, however we know there is no gold standard on clinical signs indicative of a symptomatic animal. thresholds varies in literature. 
    \item Lung Ultrasounds Videos (LUS) are really hard to analyse, even for veterinarians, they are noisy (inherently and sometimes induced for instance if animal is moving). Some data points are either bring too much noise too be exploitable or contains a part of uncertainty that should be considered if it is to be used as the basis of decision-making. Following the principle of "garbage in, garbage out" in modelling, one should create a model that is able to robustly handle garbage as they multiple interferences in real-world observations. 

\end{itemize}

%-----------
%	SOUS-SOUS-SECTION 
%------------

CP however has been proven statistically to be robust against ground truth ambiguouty. 




%-----------%-----------
%	SOUS-SECTION 
%-----------%-----------
\subsection{Article originality}

%-----------
%	SOUS-SOUS-SECTION 
%------------
The main bottleneck of integrating these expert methods (deep learning for diagnostics and epidemiological for prognostics) is the effective linkage of short-term predictions with long-term forecasting. In the next chapter, we propose three methodological frameworks for coupling deep learning diagnostic outputs with an epidemiological mechanistic models for BRD. We address precisely this challenge, providing an integrated decision-making approach that leverages the strengths of both methodologies, paving the way for reliable, timely, and economically optimal BRD control strategies.

\begin{itemize}
    \item a deep mechanistic model: we sketched a first pipeline where a deep learning architecture to diagnose high level semantic feature descriptors at several temporal points. We then re-use these features to specify an epidemiological mechanistic model to give insights at a bigger temporal resolution  
    \item uncertainty-aware modelling: we propose a way to measure the uncertainty in sensor observations and use it to improve the overall accuracy of the diagnosis and prognosis of BRD.
\end{itemize}



\subsubsection{Titre de la sous-sous-section}

%-----------%-----------
%	SOUS-SECTION 
%-----------%-----------
\subsection{Main contributions and perspectives}

The contributions of this chapter relatively to the objectives of this thesis are:

\begin{itemize}
    \item We proposed here a three coupling methodologies:
        \begin{itemize}
            \item Punctual diagnosis: We trained a deep neural network (CNN-RNN based) to predict the clinical states (symptomatic or asymptomatic) at several temporal points. These automated diagnosis are used through bayesian framework (ABC) to infer the parameters of stochastic mechanistic model of BRD.   
            \item filtered diagnosis: we employ a bayesian deep learning architecture (monte carlo dropout on our previous archictecture) to predict punctual estimates alongside confidence intervals of the clinical states. Highly noisy observations (Out-of-distribution) are filtered-out (could serve for posterior analysis by a veterinarian), thus improving the overall diagnosis and prognosis
            \item uncertainty-aware prognosis: instead of filtering-out outlier observations, we use it inversely at the weights when fitting the mechanistic model through the automated diagnosis. Resulting in an improved prognosis of BRD.
        \end{itemize}
\end{itemize}

 
\subsection{[In French] Résumé grand public}



%-----------------------------------
%	SECTION 
%-----------------------------------
\section{Article published in Preventive Veterinary Medicine (Elsevier), 2024}


    % \input{chapters/chap1-article} # pas besoin de mettre un file appart sauf si j'ai des choses spécifiques à rajouter pour cette partie
    \includepdf[pages=-]{articles/PVM.pdf}  % Replace with your actual filename

