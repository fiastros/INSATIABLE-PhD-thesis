% \chapter*{\Huge Big Title Here}
% \addcontentsline{toc}{chapter}{Big Title Here}  % Add to TOC if needed

\chapter{General discussion} % Main chapter title


%----------------------------------------------------------------------------------------
%	SECTION 
%----------------------------------------------------------------------------------------
\section{Main contributions}
%-----------%-----------
%	SOUS-SECTION 
%-----------%-----------
\subsection{Methodological progress}

Ce document est composé de trois partie proposant une progression en trois étapes dans l'exploration de des complémentarités entre un modèle deep et un modèle epidemiologique stochastique dans l'objectif de coupler des observations capteurs avec la connaissances theoriques afin de d'étudier des phénomènes complexes, à l'occurences les BRD dans le cadre de cette thèse et prendre des décisions-informées.


%-----------
%	SOUS-SOUS-SECTION 
%------------

 \begin{itemize}
     \item A partir d'observations capteurs non-structurée et en quantité limités, peut-on extraire des descripteurs de l'état de santé global des animaux ? 
     Dans un premiers temps, nous évalué les capacités de diagnostique d'un modèle deep learning à partir d'une quantité limité d'observations capteurs. Cela revient à évaluer les capacités d'un modèle de deep à extraire (pattern-matching) automatiquement des high-level semantic features descriptors à partir d'une observations capteurs non-structurées et dont on sait limités en quantité. 
     Sans surprise et malgré le peu de quantité d'observations disponible, le deep pourrait permettre d'automiser l'expertise du diagnostique. Avec une 30 de vidéos d'échographie pulmonaire, nous avons pu entrainer un modèle de deep à prédire à l'instant t (vs prédire une serie temporelle) l'état de santé (symptomatique ou non) des animaux. 
     \item A partir de diagnostiques réalisés occasionnellement, peut-on faire le lien avec nos connaissances biologiques afin d'expliciter les mécanismes d'infections et de propagation à une plus grande échelle temporelle ?  
     Nous avons également évalué les capacités de prognostique d'un modèle epidemiologique mécaniste à partir d'une quantité limité de diagnostiques vétérinaires supposé être ce qui se rapproche le plus de la vérité. Nous avons pour ainsi dire, évaluer les capacités d'un modèle de pathogène moyen de BRD à fitter (par inférence bayésienne de paramètre) des diagnostiques vétérinaires prises à différentes dates. pour une période de 30 jours, si le diagnostique véto est réalisé une fois par semaine, alors on peut être capable de fitter un modèle de pathogène moyen avec une RRMSE en dessous de 10percent. Cela permet d'obtenir explicitement les dynamiques des BRDs sur une plus grande résolution temporelle.
     \item A partir uniquement de diagnostiques (issue d'examination de signes cliniques) éffectué à une certaine fréquence, peut-on  distinguer et expliciter le modèle expert qui saura le mieux expliciter les dynamiques des BRD à une plus grande résolution temporelle ? We studied the numerical distinguishability of three BRD pathogen specific models using 5 days of observations of symptomatic animals. Results prove that we can with achieve a 95percent accuracy in identifying the infectious agents and this also that taking pathogen-informed model significantly reduces antimicrobial usage by 44percent across different batch configuration for a single batch of 12 animals.
     \item How can the diagnosis () automated by the deep learning on sensor observation serve to grasp a deeper understanding of the underlying complex mechanisms ? We sketched a hybrid pipeline where the high-level semantic feature descriptors are used to specify the theoretical knowledge embedded in a epidemiological mechanistic model. We 
     
 \end{itemize}




%-----------%-----------
%	SOUS-SECTION 
%-----------%-----------
\subsection{Practical contributions}

%-----------
%	SOUS-SOUS-SECTION 
%------------
\subsubsection{Titre de la sous-sous-section}

\section{Implications}

je peux également parler de ma contribution pour l'installation du matériel et le dév des softwares pour le transfer des datas. 
Il reste à constuire une bdd unifiés pour faciliter la publication et l'utilisation des données multimodales
c'est ici que je peux parler par exemple du projet de stétoscope dans lequel j'ai contribué
Je peux égalemnt parler

%-----------------------------------
%	SECTION 
%-----------------------------------
\section{Pending questions and perspectives}




%-----------------------------------
%	SECTION 
%-----------------------------------
\section{Conclusion}



\newpage\thispagestyle{empty}