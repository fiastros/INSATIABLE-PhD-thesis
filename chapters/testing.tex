La deuxième et troisième \textbf{citation} : \cite{Faulkes2013,Faulkes2021}.

\paragraph{Titre du paragraphe.} 

\lipsum[1]

\begin{figure}[h]
\centering
\includegraphics[scale=0.2]{figures/chap1/SiegeUP_1920-1.jpg} %pour fixer l'échelle.
\caption[Titre de la figure qui va apparaître dans la table]{Titre de la figure qui va apparaître sous la figure dans le document. Extrait de } %\cite{Smith2021-wv}. }
\label{fig:siegeUP}
\end{figure}

\textbf{Référence à une figure} dans le texte : (figure \ref{fig:siegeUP}). 
Exemples des \textbf{lettres grecques} : $\alpha$, $\beta$, $\gamma$, $\delta$





\begin{table}[tb]%[!ht]
    \raggedleft
    \caption{Liste des 5 génomes officiels d'\textit{Heterocephalus glaber} selon le NCBI.}
    \begin{tabular}{llllp{4cm}}
    \hline \hline
        \textbf{Assembly Name} & \textbf{Size (Mb)} & \textbf{Level} & \textbf{Date} & \textbf{Annotation Name} \\ \hline
        HetGla\_female\_1.0 & 2,618 & Scaffold & Mar, 2012 & NCBI RefSeq \\ \hline
       HetGla\_1.0 & 2,644 & Scaffold & Nov, 2011 & Annotation submitted by Beijing Genomics Institute \\ \hline
      Heter\_glaber.v1.7\_hic\_pac & 3,042 & Scaffold & Sep, 2020 & ~ \\ \hline
        Naked mole-rat maternal & 2,500 & Chromosome & Aug, 2022 & ~ \\ \hline
       Naked mole-rat paternal & 2,499 & Chromosome & Aug, 2022 & ~ \\ \hline \hline
    \end{tabular}
\end{table}


\subsubsection{Titre de la sous-sous-section : tables, tableaux et tableaux sur 2 pages}
Le tableau \ref{tab-dom-et} présente xxx.

\lipsum[60]

%\small
\begin{center}
\footnotesize
\begin{longtable}{p{3cm}p{10.5cm}}
\caption{Titre de la table longue (sur 2 pages).} \label{tab-dom-et} \\
\hline \hline \multicolumn{1}{l}{\textbf{Domaine d’étude}} & \multicolumn{1}{p{10cm}}{\textbf{Principales caractéristiques d’intérêt pour ces domaines d’études}} \\ \hline 
\endfirsthead

\multicolumn{2}{l}%
{{\bfseries \tablename\ \thetable{} -- suite de la page précédente}} \\
\hline \multicolumn{1}{l}{\textbf{Domaine d’étude}} & \multicolumn{1}{p{10cm}}{\textbf{Principales caractéristiques d’intérêt pour ces domaines d’études}} \\ \hline 
\endhead

\hline \multicolumn{1}{c}{{Suite à la page suivante}} \\ \hline
\endfoot

\hline \hline
\endlastfoot

Coopération sociale             &   Eusocialité \cite{Ruppell1842, Faulkes2013}, rôle insaisissable de la prolactine \cite{Faulkes2013}. \\
Endocrinologie et \newline Reproduction   &   Régulation phosphocalcique et de la glycémie singulière, utilisation privilégiée du cortisol, inhibition de l’axe gonadotrope, report de la puberté, pas de ménopause \cite{Faulkes2013}.   \\
Oncologie                       & Résistance aux cancers : plusieurs formes de cancers spontanés ou expérimentalement induits, caractéristiques génomiques uniques et des adaptations moléculaires compatibles avec la résistance au cancer \cite{Faulkes2013}.      \\
Hypoxie et \newline hypercapnie          &   Tolérance accrue à l'hypoxie, survie pendant des heures à 3\% d'O2 \cite{Faulkes2013}, survie de 18 minutes en anoxie \cite{Faulkes2013}, évolution dans un environnement confiné avec une forte densité animale \cite{Faulkes2013}. Réorganisation du métabolisme pour tolérer un environnement faible en oxygène \cite{Faulkes2013}. Niveau élevé d'activation du facteur 1$\alpha$ inductible de l'hypoxie (HIF-1$\alpha$), d'expression du facteur A de croissance endothéliale vasculaire (VEGFA), du NF$\kappa$B et de plusieurs protéines impliquées dans les fonctions neuroprotectrices pendant les conditions hypoxiques \cite{Faulkes2013}. Réponses ventilatoires hypoxiques et hypercapniques diminuées, pas d'oedème pulmonaire \cite{Faulkes2013}. \\
Somatosensoriel                 & Organe voméro-nasal peu développé, vision rudimentaire, rôle des vibrisses et poils sensoriels dans son orientation spatiale, organisation corticale somatosensorielle, sensibilité auditive originale \cite{Faulkes2013}.     \\ 
Somatosensoriel                 & Organe voméro-nasal peu développé, vision rudimentaire, rôle des vibrisses et poils sensoriels dans son orientation spatiale, organisation corticale somatosensorielle, sensibilité auditive originale \cite{Faulkes2013}.     \\ 
Somatosensoriel                 & Organe voméro-nasal peu développé, vision rudimentaire, rôle des vibrisses et poils sensoriels dans son orientation spatiale, organisation corticale somatosensorielle, sensibilité auditive originale \cite{Faulkes2013}.     \\ 
Somatosensoriel                 & Organe voméro-nasal peu développé, vision rudimentaire, rôle des vibrisses et poils sensoriels dans son orientation spatiale, organisation corticale somatosensorielle, sensibilité auditive originale \cite{Faulkes2013}.     \\ 
Système X & Système et détails \cite{Faulkes2013}.   \\
\end{longtable}
\end{center}



