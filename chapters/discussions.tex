% \chapter*{\Huge Big Title Here}
% \addcontentsline{toc}{chapter}{Big Title Here}  % Add to TOC if needed

\chapter{General discussion} % Main chapter title

% 


%----------------------------------------------------------------------------------------
%	SECTION 
%----------------------------------------------------------------------------------------
\section{Main contributions}
%-----------%-----------
%	SOUS-SECTION 
%-----------%-----------

% \subsection{Methodological progress}



\paragraph{Integration of sensor-based and mechanistic models} A central methodological advancement of this thesis lies in the integration of deep learning diagnostic tools and mechanistic epidemiological models for improved disease management, specifically applied to the Bovine Respiratory Disease (BRD) context. This thesis explicitly addressed the complexities inherent in integrating disparate model types, reconciling sensor-driven, short-term diagnostic accuracy with long-term epidemiological prognosis reliability. Inspired by the "Mixture of Experts" (MoE) concept, our methodology systematically separates diagnostic and prognostic tasks, assigning each model type to its respective domain of expertise. This loose coupling has two main methodological strengths:

\begin{enumerate}
    \item Diagnostic specialization: Deep learning excels at extracting timely and accurate diagnostic insights from sensor observations, even under conditions of limited and noisy data (72\% accuracy with less than 30 lung ultrasound videos). This demonstrated capability supports practical, short-term veterinary decision-making by automating complex clinical assessments that otherwise require significant veterinary expertise and resources [est-ce grave de n'avoir "que" 72\% de précision ? -> limitation à discuter. answer: if raw performance is used for decision-making this would be critical. [find scientific refs], moreover it has been showned that classical DL models can be uncertain in its predictions even with a high softmax output (yarin gal et al., Dropout as bayesian approximation,2016). In the works of chapter 4, when we upgrade our classical deep learning model to a bayesian achicture, the diagnosis accuracy improve to 88\% thanks to the filtering process implemented, helping the DL model set aside uncertain TUS videos.]
    
    \item Prognostic specialization: Mechanistic epidemiological models reliably extend diagnostic insights into accurate, long-term disease forecasts. Calibrating these models using empirical veterinary observations resulted in robust epidemiological predictions (forecast accuracy RMSE < 10\%), demonstrating their value for long-term disease management and strategic decision-making.
\end{enumerate}

\paragraph{Explicit integration and propagation of uncertainty} One of the essential questions addressed in this thesis concerns uncertainty management: How to explicitly handle uncertainty inherent in sensor-based observational data within hybrid diagnostic-prognostic models? This question is especially relevant for BRD diagnostics, given the inherent complexity and ambiguity of symptoms, compounded by noisy and limited ultrasound data acquisition. To explicitly manage uncertainty, we introduced a Bayesian Deep Mechanistic approach where uncertainty quantification via Monte Carlo Dropout (MCD) played a pivotal role. We quantified prediction uncertainty at the diagnostic level, filtering out the most uncertain predictions and consequently reducing diagnostic error rates from an initial RRMSE of 39\% down to 32\%. Additionally, propagating these uncertainties into mechanistic model calibration, via weighted Approximate Bayesian Computation, further improved the prognosis reliability, achieving a forecast RRMSE of 27.2\%, nearly matching the veterinarian-informed baseline (23\% RRMSE).

Our proposed solution is interesting because it directly confronts the real-world complexities of sensor-driven observations, integrating probabilistic quantification of uncertainty into the predictive decision-making pipeline. This ensures robust diagnosis and prognosis even in scenarios where data quality is inherently compromised or limited, a frequent reality in livestock management practices.

[remark on Explicit integration and propagation of uncertainty: à développer, notamment en expliquant plus en détail "pourquoi ça marche" et l'articulation entre prédiction ML et prédiction méca". ]

\paragraph{Pathogen identification through mechanistic model distinguishability} A significant methodological innovation introduced in this thesis was the pathogen-specific mechanistic model identification using Approximate Bayesian Computation (ABC) combined with multinomial logistic regression. By employing symptomatic trajectory data, we successfully distinguished between multiple candidate mechanistic models tailored for different BRD pathogens—Orthopneumovirus bovis (BRSV), Mannheimia haemolytica (Mh), and Mycoplasmopsis bovis (Mb)—achieving an average identification accuracy of approximately 93\% (BRSV=96\%, Mh=90\%, Mb=87\%).

This methodological step addresses a fundamental challenge: how to discriminate between overlapping symptomatic presentations of distinct pathogens. Our approach is innovative due to its explicit focus on model distinguishability, a providing clear methodology for a data-driven identification that can inform targeted pathogen-specific interventions.

\paragraph{Integration with bioeconomic models and practical implications} Our methodological framework explicitly extends beyond biological diagnostics by integrating bioeconomic considerations. By coupling mechanistic model outcomes with economic evaluations (expected profits, antibiotic usage metrics, and treatment costs), our approach offers a practical, real-world impact measure that assesses pathogen-informed management decisions. This integration is particularly valuable because it tangibly demonstrates that pathogen-informed mechanistic interventions not only significantly reduce antimicrobial usage (by approximately 44\%) but simultaneously maintain or slightly improve economic outcomes (+1\% net profit) compared to conventional empirical treatments. These bioeconomic insights further illustrate the practical relevance of our methodological integration for improving livestock management, reducing antibiotic misuse, and enhancing economic sustainability.

\paragraph{Structured modularity and methodological scalability} An essential and innovative methodological dimension of this thesis is its emphasis on modularity. Our hybrid modelling framework maintains clear separations between diagnostic (deep learning), prognostic (mechanistic models), and economic evaluation modules, facilitating independent model development, calibration, and validation by different disciplinary experts (epidemiologists, veterinarians, and deep learning specialists). Unlike tightly integrated architectures (e.g., Neural Differential Equations), our loosely coupled, modular approach supports interpretability  ease of model maintenance, retraining, adaptation to diverse contexts (give examples). This is particularly attractive for practical agricultural implementation, where multiple stakeholders must collaborate on disease management.

[remark: explain precisely how our approach supports interpretability],
[remark: give examples of adaptation to diverse contexts. Answer: conduite en bande porcine (réf à vianney sicard et expliquer rapidement ses travaux de modélisation multi-agent multi-echelle en epidemiologie: abstract de sa thèse ?)]

% \subsection{Practical contributions}

%-----------
%	SOUS-SOUS-SECTION 
%------------
\paragraph{Data collection}. The collection of data... 




\section{Limitations}
This subsection reviews the limitations of our current work. Future questions and research perspective is discussed in the next subsection.

There are several bottlenecks to consider in the hybrid approach we sketched throughout this thesis.

\paragraph{Ultrasound as input sensor}First the input from the sensors which are real-observations. We use pulmonary ultrasound videos to assesses our methodology as they provide a direct view of the lung tissue, making them highly relevant for assessing the severity of respiratory diseases like BRD.  In a feedlot cattle, Timsit et al. (2019) [184] demonstrated that the maximal depth and area of lung consolidation measured at the time of bronchopneumonia diagnosis using TUS were significantly associated with an increased risk of BRD disease relapse and negatively impacted animal growth performance. This is also supported by veterinarian researchers (Sébastien buczinski, 2014, comparison of thoracic ausculation) "Ultrasonographic assessment of the thorax could be a useful tool to assess BRD detection efficiency on dairy farms." and also supported by Ollivet et al. (on-farm use of ultrosnography for BRD) "TUS can be used to identify poor prognostic indicators such as caudal lung lobe consolidation, lung abscessation, and lung necrosis, and can aid culling and purchasing decisions". However this only give insights on part of infection. A Scottish study showed that in sheep, when compared with both ultrasonography and necropsy, severe lung lesions can be missed easily using thoracic auscultation.11  this study, the gold standard used (clinical signs) may occur later in the pathophysiologic process of lung infection or can be because of lung atelectasis without lung infection per se. This may have decreased the apparent sensitivity of thoracic auscultation and also may decrease specificity (ie, abnormal sounds in nonconsolidated calves) (Ollivet et al., on-farm use of ultrosnography for BRD). This negatively impacts both the performance of diagnosis will could be used to make short-term alerts by causing too many false positive and it also affects the accuracy of prognostic for relevant recommendation.
Just like veterinarians in reality doing various visual and auditive clinical examination, a more appropriate method would be to simultaneously use various modality of observation to grasp a overview from different point of vues of the disease state.

\paragraph{Proxy-base coupling} Another bottleneck to consider in our methodology is the integration of outputs of deep learning models into mechanistic model. We presented a proxy-based hybrid model which relies on Dl to extract semantic feature descriptors to inform an epidemiological mechanistic model. Is it critical to evaluate incorporate uncertainty quantification in the diagnosis and prognosis process however we relied here variationnal methods to extract DL posteriors, notably monte carlo dropout (Yarin Gal et al., 2016). this Bayesian framework is about quantifying uncertainty inherent in DL models, parameters, and consequently, the predictions. The typical guarantee we get from the Bayesian framework is that the estimates/predictions achieve the smallest average loss over the prior. That means Bayesian estimators are optimal estimators in terms of average performance, meaning they will be accurate, but do not guarantee coverage [find references] . This could be one of the reasons why our diagnosis Bayesian posteriors in chapter 4 do not always cover veterinarian ground truth. Ground truth itself contains ambiguity so it is not to be regarded as 100\% safe anyway. Another issue to consider with MCD techniques is that is hardly scales to more recent architectures proven more powerful at pattern matching than the CNN-RNN models we used, we are referring here to transformer based models (ref attention is all you need). A bayesian framework is still crucial to optimize the overall diagnosis accuracy however a statistical framework that creates coverage just like confidence interval would be required for robust prognosis as these are used as weights when fitting the mechanistic model.

\paragraph{Decision-making} Another point of consideration would be the assessment of control measures and return of investment. Through chapter 3, we showed how we could theoretically design a method to recommend relevant control measures that could reduce antimicrobial usage while improving the economical benefits. While our findings demonstrate the theoretical ability of mechanistic models to differentiate BRD pathogens based on early symptomatic data, it is crucial to acknowledge that these results are derived from simulated outbreaks. The inherent stochastic variability enhances realism, yet the models have not been validated against real-world outbreak data. In practice, multiple pathogens often co-circulate, and their interactions can complicate diagnosis and treatment strategies. Future research should focus on empirical validation by testing these models against field data to refine their predictive power and assess their robustness in real epidemiological conditions. Such validation would provide essential insights into the applicability of these models as decision-support tools for veterinarians and farmers, ensuring their relevance beyond controlled simulation environments.

\paragraph{interdisciplinarity} Other consideration for working in such interdisciplinary project is understanding and communication between different domain experts: farmers, veterinarians and modellers (mechanistic, deep learning, statistical inference) and also industrial. Just like a pareto front, they all have their expectations from this work and the difficulty is in finding the perfect ground that satisfies everyone. [detail the expectations of everyone. farmer want a tool that could they can trust and can be easily usable to make better decisions. There were challenges with veterinarians in agreeing with ground truth labels (since there is no gold standard defined yet as it is still an active area of research there is ambiguity introduced inside the labels, noise that has to be also considered. During covid-19 clinical signs to determine a infectious patient where also and are still ambiguous). For industrials there is also the question of are we time and accuracy of the methodology, a important question for economically viable solution as it actually can be scientifically translated, is the sytem design optimal ? are the chosen architecture optimal ? are the training and inference process optimal ? There are also questions of what kind hardware (small iot objects, gpu servers,) and where (edge devices vs cloud). ] This are crucial points for going towards tools that could be accepted by decision-makers. In this work, we setup-up a communication protocol between the sensors and the storage server for the audio recordings. The communication was designed to account for network shortage that sometimes occur in farms, by chunking the audio files and reconstructing the signal on the other end and assuring the package has stayed intact. There a still other issues to consider.

Notamment (pas limitatif : reprendre chaque contribution et chaque question posée dans l'introduction générale)
\begin{itemize}
    \item conditions concrètes de mise en oeuvre en élevage, obstacles pratiques, empiriques et théoriques au fonctionnement des ces méthodes, à leur adoption + discuter le rôle des vétérinaires (seuls habilités à prescrire)
    \item spécificités possibles des BRD / du système de production par rapport à la méthodologie (capacités d'extrapolation ?)
\end{itemize}


Phil cott PR, Collie D, McGorum B, Sargison N. Relationship between thoracic auscultation and lung pathology detected by ultrasonography in sheep. Vet J 2010;186:53–57
%-----------------------------------
%	SECTION 
%-----------------------------------
\section{Pending questions and perspectives}

% chercher ce qui se fait de nouveaux: multimodal deep bayesian mechanistic model
% relancer l'administration et remerciant le fait que ça a avancé.


\paragraph{Integration of multimodal deep Bayesian mechanistic models} A relevant and natural extension of this work involves developing multimodal Bayesian mechanistic models capable of simultaneously exploiting audio and visual diagnostic data. Indeed, veterinarians, farmers, and experts typically evaluate animal health through a holistic combination of visual cues—such as observing fatigue, nasal discharge, or behavioural alterations—and auditory assessments, including the detection of coughing, sneezing, or abnormal respiratory sounds. Although our methodological advancements have demonstrated strong diagnostic performance through visual sensor data alone, incorporating auditory signals offers potential for improved sensitivity to early or subtle disease symptoms. However, integrating these heterogeneous data streams poses significant methodological challenges, particularly concerning alignment, temporal synchronization, and effective joint representation learning. Inspired by Zhu et al. (2020), future research may benefit from employing deep audio-visual learning methods such as audio-visual localization, separation, and representation learning. For instance, audio-visual separation methods could assist in isolating critical respiratory sound signals from noisy environmental conditions, while representation learning techniques could provide meaningful joint embeddings for robust multimodal diagnostic classification. Nonetheless, this perspective requires careful validation given the complexity inherent in combining multiple sensory inputs in practical livestock conditions.
 
[parler egalement de comment on peut envisager l'annotation de données avec du weakly-supervised learning, coupler avec du unsupeervised ]
[un papier en plus: "Efficient Audiovisual Fusion for Active Speaker Detection"]
[une strategie pourrait être d'annnoter les vidéos avec de l'active learning, weakly-supervised/weakly-labelled using semi- or unsupervised model ]

[this could actually be used to help annotate the video or audio data, getting weakly labelled dataset; from Parmida atighechain "Bayesian active leanring ofr prduction, a systemic study and reusbal library" he said "in a real-world setup, the data is often not cleaned nor balanced. In particular, studies have shown that humans are far from perfect when labelling and the problem is even worse when using crowd-sourcing (Ipeirotis et al., 2010; Allahbakhsh et al., 2013)". We could couple that with semi-supervised learning techniques to actually combined the labelled and unlablled observations (active learning vs weakly supervised learning, semi-supervised learning, unsupervised learning)]
["This is in comparison to 2.40\% test error of DGN (Kingma et al., 2014) or 1.5\% test error of the Ladder Network model (Rasmus et al., 2015), both semi-supervised learning techniques which additionally use the entire unlabelled training set."]

[faire un peu de biblio sur les methode de modelisation en weakly labelled: TTA, semi-supervised: "In semi-supervised learning a model is given a fixed set of labelled data, and a fixed set of unlabelled data. The model can use the unlabelled data to learn about the distribution of the inputs, in the hopes that this information will aid in learning from the small labelled set as well." Yarin Gal dans Deep Bayesian Active Learning with Image Data]


\paragraph{Explicit uncertainty estimation: Conformal prediction versus variational inference} Our thesis leveraged variational inference (Monte Carlo Dropout) for uncertainty quantification, effectively reducing diagnostic error by explicitly modeling uncertainty in sensor-based predictions. However, recent developments in Explainable AI (XAI) suggest that integrating conformal prediction approaches could complement Bayesian methods by providing statistical coverage guarantees. As Matiz \& Barner (2020) highlighted, conformal prediction offers explicit statistical assurances regarding uncertainty coverage, a property not inherently guaranteed by variational inference alone. While Bayesian methods optimize for average predictive accuracy, conformal prediction provides robust and calibrated uncertainty intervals without assumptions about the underlying data distribution. Future research should therefore investigate integrating conformal prediction into our deep mechanistic framework, potentially resulting in hybrid models combining the accuracy of Bayesian approaches with conformal statistical guarantees. However, practical integration will necessitate careful methodological consideration to balance computational complexity against reliability gains, especially in realistic diagnostic scenarios characterized by noisy or sparse data.



\paragraph{Inference robustness} In mechanistic epidemiological modeling, inference robustness—ensuring consistency of predictions across plausible model variations—is a critical component of model reliability for decision-making. Throughout this thesis, we employed Approximate Bayesian Computation (ABC) with multinomial logistic regression to identify and discriminate among pathogen-specific models, achieving strong identification performance. Yet, alternative simulation-based inference methods such as ABC Sequential Monte Carlo (ABC-SMC) might further enhance robustness by efficiently exploring the parameter space, thereby potentially improving calibration accuracy and parameter estimation stability. Future work should systematically evaluate such alternative numerical solvers, explicitly comparing inference robustness across ABC variants or recently developed approaches such as ABC-SMC. Such assessments, including comparisons to recent methods demonstrated by Beaunée (BBRWE ? son packge à gael) ou sinon (Francesco Pinotti, de l'UMR EPIA, Simulation-based inference with complex data and simulators).

\paragraph{Robust decision-making} Although our thesis has effectively integrated bioeconomic models to assess real-world implications, further strengthening of the coupling between mechanistic predictions and economic outcomes is desirable. Particularly, integrating Bayesian posterior distributions into economic models may allow construction of uncertainty sets, ensuring that economic decisions remain robust against prediction uncertainty. For instance, leveraging Bayesian posterior distributions could help construct decision strategies that uniformly encompass plausible epidemiological outcomes, providing safer economic recommendations under uncertainty. Such methodologies have demonstrated practical effectiveness in ensuring safety under uncertain predictions (Eyango et al., 2024). Future research could thus aim to explicitly integrate posterior predictive uncertainty within bioeconomic decision-making, enhancing both theoretical coherence and real-world utility of pathogen-informed economic decisions.

\paragraph{A deep mechanistic model with prognosis expert selection} An important future research avenue involves integrating the developed deep mechanistic diagnostic frameworks (chapter 4) with the pathogen-specific prognosis expert selection mechanisms validated in chapter 3, leveraging the pulmonary ultrasound video datasets collected (chapter 2). The envisioned approach would utilize deep learning predictions from ultrasound videos to infer clinical states at discrete assessment points, subsequently employing numerical solvers (e.g., ABC methods) to distinguish the most likely pathogen-specific mechanistic models. This approach could be validated by comparison with ground-truth pathogen identification via biological examination (blood samples). Through such validation, future research could rigorously quantify the direct benefits of pathogen-informed mechanistic interventions in terms of antimicrobial usage reduction and net profit optimization, thus bridging diagnostic precision, epidemiological forecasting accuracy, and economic viability in livestock management. 

\paragraph{Application to related domains - transferability} Finally, the modular methodological framework proposed in this thesis inherently facilitates adaptation and scalability across diverse agricultural contexts. Future work could explore transferring and assessing the coupling methodology to related agricultural domains, such as pig batch management (Sicard vianney thèse) or plant disease diagnostics ( N parisey articles). Such applications would test and potentially confirm the generalizability and scalability of our structured modular approach, further extending its methodological relevance and practical impact across broader agricultural management practices.


pour faire références à la thèse d'Hassan, on peut également regarder tout les questions autours du comment disposer les capteurs pour optimiser la collecte d'informations riche et miniser la consommation d'energie (regarder la problèmatique de Hassane). 

% \section{Conclusion}

% Il reste à constuire une bdd unifiés pour faciliter la publication et l'utilisation des données multimodales

% In conclusion, the methodological progress of this thesis demonstrates how to integrating deep learning with mechanistic epidemiological models—grounded in explicit uncertainty quantification—significantly improves both diagnostic accuracy and epidemiological prognosis from sensors observations. 


% By explicitly addressing the complexities and uncertainties inherent in sensor-based disease diagnostics, our approach provides practical, actionable solutions for more effective and sustainable livestock disease management strategies.

% remarks [le but est d'avoir construit de nouvelles connaissances et ouvert de nouvelles questions, pas juste d'avoir une sorte de réponse binaire]

\newpage\thispagestyle{empty}