\chapter*{Remerciements}
\addcontentsline{toc}{chapter}{Remerciements}
Cette thèse n’aurait pu voir le jour sans l’accompagnement, le soutien et l’expertise de nombreuses personnes, que je tiens à remercier ici chaleureusement. \vspace{0.5\baselineskip}

Je souhaite tout d’abord exprimer ma profonde gratitude à mes encadrants académiques, \textbf{Sébastien PICAULT} et \textbf{Nicolas PARISEY}, pour leur engagement sans faille tout au long de ce travail. Leur patience, leur polyvalence, ainsi que la richesse de leurs connaissances et de leur expertise ont constitué un cadre intellectuel stimulant, exigeant et toujours bienveillant. Leur regard croisé a permis à ce projet de s’épanouir à l’intersection de plusieurs disciplines, en lui donnant à la fois rigueur scientifique et portée opérationnelle. \vspace{0.5\baselineskip}

Je remercie également mes encadrants en entreprise, dans le cadre de la convention CIFRE, au sein d’\textbf{Adventiel}. Leur disponibilité, leur écoute, et leur capacité à prendre du recul sur les cas d’usage concrets ont grandement contribué à l’ancrage et à la pertinence de ce travail. Leur bienveillance, leur patience et le temps qu’ils m’ont consacré m’ont permis d’évoluer sereinement dans un environnement professionnel exigeant mais toujours ouvert. Je n'oublie évidemment pas non plus \textbf{Jean DU PUYTISON}, qui a su me faire confiance et a soutenu le projet.\vspace{0.5\baselineskip}

Je tiens à adresser mes remerciements sincères à l’ensemble de l’équipe DATA  d’\textbf{ADVENTIEL}, en particulier \textbf{Léane}, \textbf{Bastien}, \textbf{Tim} et tout spécialement \textbf{Leslie}, pour leur soutien constant, leur temps précieux et leurs relectures attentives du manuscrit, qui m’ont grandement aidé dans les dernières phases de rédaction. Un grand merci à \textbf{Lila}, sans qui je n'aurais jamais réussi toutes les démarches administratives ainsi qu'à \textbf{Maxime} et \textbf{Nathan} pour leur bonne humeur. \vspace{0.5\baselineskip}

Mes remerciements s’adressent également à l’équipe \textbf{DYNAMO de BIOEPAR} : \textbf{Gaëlle}, \textbf{Alifa}, \textbf{Pauline}, \textbf{Baptiste}, \textbf{Guita} et \textbf{Vianney}, pour leurs relectures précieuses, leurs retours constructifs et la qualité de leurs apports scientifiques spécifiques, qui ont enrichi le projet à bien des égards. \vspace{0.5\baselineskip}

Un grand merci à l’équipe d’\textbf{ONIRIS}, et en particulier à \textbf{Sébastien ASSIE} pour son soutien, ainsi qu’à \textbf{Maud}, doctorante, pour le temps considérable qu’elle a consacré à la relecture des articles, à son expertise vétérinaire et à la collecte et l’annotation des données dans des conditions parfois difficiles. Son implication concrète a été essentielle à la robustesse des résultats présentés. \vspace{0.5\baselineskip}

Je n’oublie pas \textbf{Melen}, de l’\textbf{IGEPP}, pour son soutien lors du démarrage de la thèse. Son partage d’expérience et ses conseils avisés sur les attendus du travail doctoral m’ont été d’une aide précieuse dans les premiers mois de cette aventure. \vspace{0.5\baselineskip}

Enfin, je tiens à remercier mes \textbf{parents}, pour leur soutien indéfectible et leur présence, même à distance. Leur appui familial, discret mais constant, a été une source de force et de sérénité tout au long de ce parcours. \vspace{0.5\baselineskip}

À toutes ces personnes, et à celles qui ont croisé mon chemin de près ou de loin durant cette thèse, je vous adresse ma reconnaissance la plus sincère.
