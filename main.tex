%----------------------------------------------------------------------------------------
%	PREAMBULE // PACKAGES AND OTHER DOCUME2NT CONFIGURATIONS
%----------------------------------------------------------------------------------------
\documentclass[a4paper,12pt,twoside,english]{book}
\usepackage[utf8]{inputenc}
\usepackage[T1]{fontenc}
\usepackage{babel}
% \usepackage[backend=biber]{biblatex}

%	MARGIN SETTINGS
\usepackage[a4paper,left=2cm,right=2cm,top=2cm,bottom=2cm, twoside]{geometry}

\usepackage{xcolor}
\usepackage{stmaryrd}
\usepackage{amssymb}
\usepackage{amsmath}
\usepackage{libertine}



%\usepackage{cite}
\usepackage{hyperref} % for references (\ref, \label), url
\hypersetup{
    colorlinks=true,
    linkcolor=black,
    filecolor=black,      
    urlcolor=black,
    citecolor=black,
    pdfpagemode=FullScreen,
    }
\usepackage[
    % backend=biber, 
    natbib=true,
    style=numeric,
    sorting=none
]{biblatex} %Imports biblatex package

\addbibresource{biblio.bib} %Import the bibliography file

\usepackage{graphicx} % to include images
%\usepackage[pdftex]{graphicx}
\graphicspath{ {figures/} }
\usepackage{caption}
\usepackage{subcaption}
\usepackage{float}
\usepackage{multirow}
\usepackage{array, multirow, tabularx}
\usepackage{booktabs}
\usepackage{longtable}
\usepackage{apalike}
\usepackage{minitoc}
\usepackage{tikz}
\usepackage{fancyhdr}
\pagestyle{fancy}
\usepackage[strict]{changepage}
\usepackage{siunitx}
\selectlanguage{english}%

\usepackage{enumitem}
\usepackage{epigraph}
%\usetikzlibrary{shadows.blur}
\usepackage{lscape}
\usepackage{titletoc}
\usepackage{mdframed}

\usepackage{tabu}
\usepackage{pdfpages}

\usepackage{arydshln}
%\usepackage[nonumberlist]{glossaries}
\usepackage{calc}
\usepackage[]{titlesec} 
\definecolor{linkColor}{HTML}{32a852}
%\usepackage[colorlinks=true,citecolor=linkColor,linkcolor=black]{hyperref}

\addto\captionsfrench{%
  \renewcommand{\listfigurename}{List of figures}%
}



%%%%%%%%%%%%%%%%%%%%%%%%%%%%%%%%%%%%%%%%%%%%%%%%%%%%%%%%%%%%%%%%%%%%%%%%%%
%%%%%%%%%%%%%%%%%%%%%%%%%%%%%%%%%%%%%%%%%%%%%%%%%%%%%%%%%%%%%%%%%%%%%%%%%%

\usepackage{sectsty}
%\sectionfont{\color{CentraleRed}}  % sets colour of sections
%\subsectionfont{\color{uclablue}}  % sets colour of subsections, ou gray, teal 
\definecolor{darkseagreen}{rgb}{0.56, 0.74, 0.56}
\usepackage{lipsum}

\title{INSATIABLE V2.0.}
\author{EYANGO TABI Loic}
\date{March 2025}


%----------------------------------------------------------------------------------------
%	HEADERS AND FOOTERS
%----------------------------------------------------------------------------------------

%\fancyhead[RO]{\scshape \leftmark}
%\fancyhead[L]{\scshape \righttmark}
\fancyhead{} % clear all header fields
\fancyfoot{} % clear all footer fields
%\fancyhead[LE,RO]{\slshape \rightmark}
\fancyfoot[C]{--~\thepage~--}

\usepackage{emptypage}

%\leftmark : adds name and number of the current top-level structure (for example, Chapter for reports and books classes; Section for articles ) in uppercase letters.
% \rightmark : adds name and number of the current next to top-level structure (Section for reports and books; Subsection for articles) in uppercase letters.

%%%%%%Glossaire
%\usepackage{glossaries}
\usepackage[acronym,xindy,toc]{glossaries} % nomain, if you define glossaries in a file, and you use \include{INP-00-glossary}
%[nomain,acronym,xindy,toc]
\makeglossaries

\usepackage[xindy]{imakeidx}
\makeindex
%\input{auxilliaires/glossaire}
%%%%%%%%%%%%%%%%%%%%%%%%%%%%%%%%%%%%%%%%%%%%%%%%%%%%%%%%%%%%%%%%%%%%%%%%%%
%%%%%%%%%%%%%%%%%%%%%%%%%%%%%%%%%%%%%%%%%%%%%%%%%%%%%%%%%%%%%%%%%%%%%%%%%%
%----------------------------------------------------------------------------------------
%	DOCUMENT
%----------------------------------------------------------------------------------------

\begin{document}

    \begin{titlepage}
        %\maketitle
        %----------------------------------------------------------------------------------------
        %	TITLE PAGE
        %----------------------------------------------------------------------------------------
        \includepdf[pages=1]{figures/ED_requirements/page_de_garde.pdf}  % Replace with your actual filename
        \newpage %passer à une nouvelle page
        \thispagestyle{empty} %page de garde vide
    \end{titlepage}
    
%\renewcommand{\chaptermark}[1]{\markboth{\textsc{#1}}{}}
\renewcommand{\chaptermark}[1]{\markboth{#1}{}}
\frontmatter % numérote les pages en chiffres romains jusqu'a la commande mainmatter

%----------------------------------------------------------------------------------------
%	RESUME ET ABSTRACT PAGE
%----------------------------------------------------------------------------------------
 

%----------------------------------------------------------------------------------------
%	QUOTATION PAGE ou DEDICACE
%----------------------------------------------------------------------------------------
\clearpage
\vspace*{0.2\textheight}


\begin{quote}
\textbf{
"La connaissance s'acquiert par l'expérience, tout le reste n'est que de l'information".
}
\end{quote} \bigbreak

\hfill 
- Albert Einstein (n.d.)



% \mbox{}
% \newpage
%----------------------------------------------------------------------------------------


%----------------------------------------------------------------------------------------
%	ACKNOWLEDGEMENTS
%----------------------------------------------------------------------------------------
\chapter*{Remerciements}
\addcontentsline{toc}{chapter}{Remerciements}
Cette thèse n’aurait pu voir le jour sans l’accompagnement, le soutien et l’expertise de nombreuses personnes, que je tiens à remercier ici chaleureusement. \vspace{0.5\baselineskip}

Je souhaite tout d’abord exprimer ma profonde gratitude à mes encadrants académiques, \textbf{Sébastien PICAULT} et \textbf{Nicolas PARISEY}, pour leur engagement sans faille tout au long de ce travail. Leur patience, leur polyvalence, ainsi que la richesse de leurs connaissances et de leur expertise ont constitué un cadre intellectuel stimulant, exigeant et toujours bienveillant. Leur regard croisé a permis à ce projet de s’épanouir à l’intersection de plusieurs disciplines, en lui donnant à la fois rigueur scientifique et portée opérationnelle. \vspace{0.5\baselineskip}

Je remercie également mes encadrants en entreprise, dans le cadre de la convention CIFRE, au sein d’\textbf{Adventiel}. Leur disponibilité, leur écoute, et leur capacité à prendre du recul sur les cas d’usage concrets ont grandement contribué à l’ancrage et à la pertinence de ce travail. Leur bienveillance, leur patience et le temps qu’ils m’ont consacré m’ont permis d’évoluer sereinement dans un environnement professionnel exigeant mais toujours ouvert. Je n'oublie évidemment pas non plus \textbf{Jean DU PUYTISON}, qui a su me faire confiance et a soutenu le projet.\vspace{0.5\baselineskip}

Je tiens à adresser mes remerciements sincères à l’ensemble de l’équipe DATA  d’\textbf{ADVENTIEL}, en particulier \textbf{Léane}, \textbf{Bastien}, \textbf{Tim} et tout spécialement \textbf{Leslie}, pour leur soutien constant, leur temps précieux et leurs relectures attentives du manuscrit, qui m’ont grandement aidé dans les dernières phases de rédaction. Un grand merci à \textbf{Lila}, sans qui je n'aurais jamais réussi toutes les démarches administratives ainsi qu'à \textbf{Maxime} et \textbf{Nathan} pour leur bonne humeur. \vspace{0.5\baselineskip}

Mes remerciements s’adressent également à l’équipe \textbf{DYNAMO de BIOEPAR} : \textbf{Gaëlle}, \textbf{Alifa}, \textbf{Pauline}, \textbf{Baptiste}, \textbf{Guita} et \textbf{Vianney}, pour leurs relectures précieuses, leurs retours constructifs et la qualité de leurs apports scientifiques spécifiques, qui ont enrichi le projet à bien des égards. \vspace{0.5\baselineskip}

Un grand merci à l’équipe d’\textbf{ONIRIS}, et en particulier à \textbf{Sébastien ASSIE} pour son soutien, ainsi qu’à \textbf{Maud}, doctorante, pour le temps considérable qu’elle a consacré à la relecture des articles, à son expertise vétérinaire et à la collecte et l’annotation des données dans des conditions parfois difficiles. Son implication concrète a été essentielle à la robustesse des résultats présentés. \vspace{0.5\baselineskip}

Je n’oublie pas \textbf{Melen}, de l’\textbf{IGEPP}, pour son soutien lors du démarrage de la thèse. Son partage d’expérience et ses conseils avisés sur les attendus du travail doctoral m’ont été d’une aide précieuse dans les premiers mois de cette aventure. \vspace{0.5\baselineskip}

Enfin, je tiens à remercier mes \textbf{parents}, pour leur soutien indéfectible et leur présence, même à distance. Leur appui familial, discret mais constant, a été une source de force et de sérénité tout au long de ce parcours. \vspace{0.5\baselineskip}

À toutes ces personnes, et à celles qui ont croisé mon chemin de près ou de loin durant cette thèse, je vous adresse ma reconnaissance la plus sincère.



%----------------------------------------------------------------------------------------
%	LIST OF CONTENTS/FIGURES/TABLES PAGES
%--------------------------------------------------‡--------------------------------------
\setcounter{secnumdepth}{3} % organisational level that receives a numbers
\setcounter{tocdepth}{3}
\tableofcontents % Prints the main table of contents
\listoffigures % Prints the list of figures
\addcontentsline{toc}{chapter}{List of figures}
% \listoftables % Prints the list of tables
% \addcontentsline{toc}{chapter}{List of tables}

%----------------------------------------------------------------------------------------
%	Liste des abréviations-
%----------------------------------------------------------------------------------------
\input{auxilliaires/abreviations}


% --------------------------------------------------------------
%                         Début du corps
% --------------------------------------------------------------

\mainmatter
\setlength{\parskip}{.7em}

%\titlespacing*{\section}{0pt}{.9em}{.8em}
\renewcommand{\baselinestretch}{1.1}

\fancyhead[RO]{\leftmark}
\fancyhead[LE]{\textsc{\chaptername~\thechapter}}

%\chapter*{Introduction}
% \fancyhead{} % clear all header fields
% \fancyhead[OL]{\textsc{Foreword}}
% \chapter*{Foreword - à enlever !} % Main chapter title
\addcontentsline{toc}{chapter}{Foreword}  

%Le but de l'avant-propos est d'améliorer les conditions dans lesquelles les membres d’un jury, vont pouvoir l’apprécier.
%\textbf{C’est une partie facultative du travail de recherche dans laquelle l’auteur peut expliquer les raisons qui l’ont incité à étudier le sujet en question, tout en exposant le but poursuivi et les difficultés rencontrées en cours de recherche.}

The conjoining of dynamical systems and deep learning has become a
topic of great interest. In particular, neural differential equations (NDEs)
demonstrate that neural networks and differential equation are two sides
of the same coin. Traditional parametrised differential equations are a
special case. Many popular neural network architectures, such as residual
networks and recurrent networks, are discretisation.
NDEs are suitable for tackling generative problems, dynamical systems,
and time series (particularly in physics, finance, . . . ) and are thus of
interest to both modern machine learning and traditional mathematical
modelling. NDEs offer high-capacity function approximation, strong priors
on model space, the ability to handle irregular data, memory eficiency,
and a wealth of available theory on both sides.
This doctoral thesis provides an in-depth survey of the field.
Topics include: neural ordinary differential equations (e.g. for hybrid
neural/mechanistic modelling of physical systems); neural controlled differential
equations (e.g. for learning functions of irregular time series);
and neural stochastic differential equations (e.g. to produce generative
models capable of representing complex stochastic dynamics, or sampling
from complex high-dimensional distributions).
Further topics include: numerical methods for NDEs (e.g. reversible
differential equations solvers, backpropagation through diferential equations,
Brownian reconstruction); symbolic regression for dynamical systems
(e.g. via regularised evolution); and deep implicit models (e.g. deep
equilibrium models, differentiable optimisation).
We anticipate this thesis will be of interest to anyone interested in the
marriage of deep learning with dynamical systems, and hope it will provide
a useful reference for the current state of the art.



\subsubsection*{Motivation}

We have two goals in writing this document. One: to satisfy the requirements of a
PhD, by writing a thesis describing our original research. Two: to give an accessible
survey of the new, rapidly developing, and in our opinion very exciting field of neural
differential equations. To the best of our knowledge this is the first survey to have
been written on the topic.
We hope this will prove useful to the interested reader! Along the way we shall cover a
wide variety of applications, both to classical mathematical modelling, and to typical
machine learning problems.


\subsubsection*{Getting started}
We will assume throughout that the reader is familiar with the basics
of ODEs and with the basics of modern deep learning, but we will not assume an
in-depth knowledge of either. On the basis that many of our readers may come from a
traditional applied mathematics background without much exposure to deep learning,
then Appendix A also provides a summary of the relevant deep learning concepts we
shall assume. It also provides references for learning more about deep learning.
The material on neural SDEs will assume familiarity with SDEs.
Beyond these (relatively weak) assumptions, we will introduce concepts as we need
them. Various parts of the text will touch on topics such as rough path theory,
or numerical methods for differential equations. In each case we assume little-tono
familiarity on the part of the reader, and where necessary provide references for
learning more about them.
The next chapter (on neural ODEs) makes an effort to explicitly spell out even `elementary'
details such as the existence of solutions to ordinary differential equations,or the use of cross entropy as a loss function. Later chapters assume increasing levels
of sophistication; it is recommended to read them in sequential order.

Code The reader interested in applying these techniques is strongly encouraged to
write some example code.
Each chapter contains a few numerical examples usually on toy datasets for ease
of understanding. The corresponding code is both available and well-documented:
they can be found as the examples of the Diffrax software library [Kid21a], which is
written for the JAX framework [Bra+18].
Indeed standard software libraries for solving and differentiating differential equations
make working with NDEs essentially easy. These are discussed in Section 5.6
(including both Diffrax and other options for other frameworks). These libraries are
again well-documented and contain numerous examples.


Experiments The material here focuses on presenting the theory of NDEs; correspondingly
our numerical examples will tend to be on toy datasets chosen for ease of
understanding. Real world (and possibly very large scale) applications of these techniques
may be found in the original papers, which are referenced in the text alongside
each individual topic.


\textcolor{red}{je dois rajouter une partie pour les notations scientifiques/mathématiques ? comme la thèse "On neural differential equations". partie appélé "Notations"}
 %avant-propos

\fancyhead{} % clear all header fields

\fancyhead[RO]{\leftmark}
\fancyhead[LE]{\textsc{\chaptername~\thechapter}}

\chapter{Introduction} 

\section{Livestock farming, animal health and infectious diseases}

\subsection{Socio-economical context and the stakes}

Livestock farming is a cornerstone of agricultural systems, essential for food production and the livelihood of many communities. However, modern farming faces a dual challenge: ensuring high levels of animal health and well-being while transitioning toward sustainable, self-sufficient practices. In this section, I will first give a general context on livestock farming , ...
I will outline 
%the socio-economic impacts our work could have.


\subsubsection{Livestock Farming, animal health well-being}

First, we define the core concepts:
\begin{itemize}
    \item Livestock Farming: The organized breeding and raising of animals for food, fiber, and other products.
    \item Animal Health and Well-Being: The physical and psychological state of animals, which directly impacts productivity and the quality of the products.
    \item Sustainable/Self-Sufficient Livestock Farming: Farming systems that are designed to maintain productivity and animal health over the long term with minimal reliance on external inputs, promoting environmental, economic, and social resilience.
\end{itemize}


Next, we examine why sustainable livestock farming is critical in today’s global context:

\begin{itemize}
    \item Food Security: As the global population grows, the demand for high-quality animal products increases. Ensuring that farming practices are sustainable helps guarantee that future generations will have access to sufficient, nutritious food. This involves not only meeting quantity demands but also maintaining quality standards.
    \item Animal Welfare and public Health: The safety of the food supply is paramount. Sustainable practices reduce the risks of diseases that can be transmitted between animals and humans. Maintaining robust animal health also minimizes the use of interventions, such as antibiotics, which can lead to issues like antibiotic resistance.
    \item Environmental and Ecological Management: Livestock farming significantly contributes to environmental challenges such as greenhouse gas emissions, pollution, and deforestation. Transitioning to sustainable systems can mitigate these impacts, balancing food production with the need to preserve natural ecosystems.
    \item Resource Management: Efficient farming requires the optimal use of resources—land, water, energy, and labor. Sustainable systems are designed to reduce waste and improve efficiency. This is particularly important as both the availability of resources and the demand for skilled labor (veterinarians, farmers, technicians) face increasing pressures in many parts of the world.
\end{itemize}

By addressing these interrelated aspects, sustainable livestock farming not only tackles immediate socio-economic challenges but also contributes to long-term ecological balance and public health protection.

\textit{\textbf{Keywords:}} Livestock farming, animal health, animal-welfare, resource management, food security, ecology


\subsubsection{Infectious diseases - impact on livestock farming}

Infectious diseases are disorders caused by pathogenic microorganisms—such as bacteria, viruses, fungi, or parasites—that can spread directly or indirectly from one host to another. These diseases can manifest across different domains, affecting both humans and animals, and their transmission may occur at multiple scales—from within a single host to entire metapopulations.

\textbf{Types and Transmission Patterns:} Infectious diseases vary by type. For instance, respiratory infections (e.g., COVID-19 in humans and certain respiratory illnesses in animals) and sexually transmitted diseases (such as AIDS in humans) illustrate the diversity of these conditions. In the animal domain, examples like African swine fever and bovine viral diarrhea underscore how pathogens can devastate livestock. Some infections are zoonotic, meaning they can jump from wildlife to livestock and even to humans (e.g., Ebola, monkeypox), highlighting the interconnectedness of human, animal, and environmental health. Transmission can occur in several ways, including host-to-host contact, within-host dynamics, and across populations, which sometimes creates complex patterns that pose significant public health challenges.

The effects of infectious diseases are far-reaching:
\begin{itemize}
    \item Ecological Impact: Inappropriate or excessive use of antimicrobials in response to infections can lead to the emergence of resistant bacteria. This resistance not only renders certain treatments ineffective but also complicates recovery, thereby jeopardizing both animal and human health.
    \item Food Safety and Security: Outbreaks can cause major production losses in livestock farming by reducing both the quality and quantity of animal products. These losses threaten the broader goal of achieving food security, especially in regions where livestock is a primary source of nutrition and income
    \item Animal Health and Welfare: Infectious diseases directly compromise the health and well-being of animals, leading to suffering and significant economic losses for farmers.
    \item Resource Management: Managing infectious diseases demands frequent monitoring and diagnosis, along with substantial investments in vaccines, antibiotics, and veterinary expertise. This imposes a heavy financial and operational burden on farming systems, making sustainable practices even more challenging to implement.
\end{itemize}

By understanding these aspects, one recognizes that controlling infectious diseases is not only a matter of public health but also a critical component of creating sustainable and self-sufficient livestock farming systems.

\textit{\textbf{Keywords:}} infectious diseases, critical concern, animal welfare, animal health, antimicrobial resistance, economical loss, production loss, food safety, public health, epidemiology

% Did I miss key ideas that could help me demonstrate why infectious diseases in livestock farming are interresting topics in society currently ? 
% Also i think inn this subsection i need to give more example of infectious diseases in the livestok farming context. The other ones (for humans, or wild animals) are juste to give an idea or illustrate something. 

\subsection{Conventional control strategies of infectious diseases}
\label{Conventional control strategies} % ajout-du-label-pour-faire-reference-a-la-sous-section

\subsubsection{Diagnose, prognose and recommend control measures}

Effective management of infectious diseases in livestock relies on a systematic process that begins with accurate diagnosis, proceeds with reliable prognosis, and culminates in tailored recommendations for control measures. Each step is crucial for mitigating outbreaks and ensuring the long-term sustainability of farming systems.

\paragraph{Diagnosis.} The first step involves identifying the presence and nature of the disease. Diagnosis can be achieved through:
\begin{itemize}
    \item Clinical Appraisal: This is the most common method among farmers, where visual or tactile observations are used to detect clinical signs in animals. While this approach is quick and accessible, it is inherently subjective and may lead to inconsistent interpretations.
    \item Biological Examinations: In this approach, samples such as blood or mucus are collected for laboratory analysis (e.g., PCR testing). Although this method can provide a more precise identification of pathogens, it is often more invasive, time-consuming, and may delay decision-making during fast-moving outbreaks.
\end{itemize}

\paragraph{Prognosis.} Once the disease is diagnosed, the next step is to forecast its progression and potential impact. Prognosis typically relies on:
\begin{itemize}
    \item Expert Knowledge: Veterinarians draw on empirical observations and scientific experimentations (detail more how they build knowledge pieces by pieces) knowledge, to predict the disease trajectory.
    \item Historical Data: Farmer can also rely on past events and past decision outcomes to help inform the likely spread of the disease. 
    % though variability between farms or over time can challenge generalization and accuracy.
\end{itemize}

\paragraph{Recommendation of Control Measures.} Based on the diagnostic and prognostic assessments, appropriate control measures can be devised. These measures are intended to:
\begin{itemize}
    \item Reduce the spread of the disease within and between farms.
    \item Minimize production losses and safeguard food safety.
    \item Optimize the use of resources such as veterinary expertise, time, and financial investment.
    \item Balance short-term emergency responses with long-term sustainable practices
\end{itemize}

While these conventional methods have proven useful, they come with inherent limitations.

\textit{\textbf{Keywords:}} Diagnosis, clinical appraisal, biological examinations, prognosis, expert knowledge, empirical observations, scientific experimentations, recommendations, control measures.


\subsubsection{Challenges to tackle}

\paragraph{Limitations in Diagnosis:} 
\begin{itemize}
    \item Subjectivity and Inconsistency: Visual and manual examinations by farmers or experts, though common, are inherently subjective. Different observers may interpret clinical signs in various ways, leading to inconsistent diagnoses.
    \item Scalability Issues: Manual diagnosis becomes increasingly impractical as farm sizes grow or when frequent assessments are necessary. This limitation is critical as rising demand requires more rapid and consistent monitoring.
    \item Intrusiveness and Delay in Biological Testing: Biological examinations (e.g., blood sampling and PCR testing) offer greater precision but are invasive and time-consuming. The time lag in obtaining results can allow the disease to progress before control measures are implemented.
\end{itemize}

Challenges in Prognosis:
\begin{itemize}
    \item Limited Expertise and Resource Constraints: Relying on expert judgment for prognosis is not scalable. The shortage of veterinarians, especially in rural areas, exacerbates the challenge of providing timely and accurate prognoses across large populations.
\end{itemize}

Addressing these challenges is essential for enhancing the reliability and scalability of disease management strategies. 

\textit{\textbf{Keywords:}} subjectivity, scalability, shortage of expertise, costly, laborious, time-consuming, animal stress, repetitive

% il faudrait que je trouve un endroit où mettre les défifintiosn de certains mots: metapopulation par exemple ?  mais c'est mieux de l'insérer dans le texte si c'est facilemetn possible. 
% je peux jeter un coup d'oel sur l'article de pauline pour avoir d'autres argument
% et regrader de le docs de maud si on parle de méthode traditionnele de control des maladies infectieuse: éleveur comme vétos.
% je vais devoir chercher les chiffres pour argumenter certains faits. example: ethical concerns...

% my old notes
% - il faut bien préciser que c'est ce qui est le plus répandu car c'est le minimum et c'est à la portée de tous mais elle a également ces limites.
% - observation des eleveurs et des experts
% - estimation des états de santé ... 
% - doses des traitement non adaptés: phénomène de resistance des bactéries
% - la santé animale qui prend chère
% - méthode purement empiique qui mais qui peut couter chère
% - d'un environnement à l'autre ce n'est pas toujours les mêmes stratégies qui peuvent être appliquées
% - scalable ? update of the knowledge ? transferability to other issues ? 
% - manual job that is not consistant but it is the baseline that has always been done
% - le problème c'est que non seuelemnt les eleveurs seuls ne possèdent pas toute la connaissance véto et/ou de modélisateurs pour optimiser de manière rigoureuse et stable les .... (lire l'article de pualine pour prouver pourqoui cees méthodes ne peuvent pas faire long terme tel quels s'ils ne sont pas automatisé. (ils ne sont pas mauvais, mais iil faut une pratique standardisé)

% - conclure sur la self-suffisiancy ou susbtainability des filières par rapport aux maladies respiratoires. 
% - Tacler tous ces problèmes nécessites la mise en place de méthodes qui sont repoductible, compréhensible/interpretable, facilement utilisable, et qui sont construit en impliquant la connaissances des tous les acteurs de la filière concernées. 

% - L'idéée est d'avoir des méthodes qui peuvent aider à aléger la tâche aux éleveurs afin qu'ils consacrent leur energies, argent, temps et efforts sur des problèmes à plus forte valeur ajoutés.  
% peuvent nous aider à mimer et faire mieux ou de manière fiable ce ce qu'on.

%----------------------------------------------------------------------------------------
%	SECTION 
%----------------------------------------------------------------------------------------
\clearpage 


\section{Artificial intelligence}

A note on history (un rappel de l'ia, les grands courants... ce qu'on voit tous les jours ne reflètent qu'une partie de l'iceberg imergé ??? bref

ou est-ce que nous on se situe par rapport aux DL, modèle epi et modèle hybride.
Ensuite je me suis intéressé à tel et tel partie (par rapoport aux choses faites dans les modèles hybrides)


Rajouter stéto dans les perspectives (discussions)


\subsection{Precision agriculture: hardware and sensors}

Precision agriculture represents an innovative approach to farming that focuses on optimizing inputs (such as feed, medication, and energy) while maximizing outputs, including net profit and environmental sustainability. This paradigm is rapidly being adopted in livestock farming to enhance decision-making and operational efficiency. At its core, precision agriculture is supported by a robust infrastructure consisting of both hardware and software components. In this subsection, we concentrate on the sensors that underpin data collection:


\begin{itemize}
    \item Behavioural Data: For instance, accelerometers can record movement patterns of individual animals.
    \item Positional Data: Electronic identification systems (such as Eartags) track animal locations and movement across the farm.
    \item Biological Data: Devices like intraruminal boluses can continuously monitor physiological metrics such as body temperature.
    \item Environmental Data: Connected sensors measure ambient conditions, such as temperature and humidity, which can influence animal health.
    \item Operational Data: Information regarding feeding schedules and management practices can also be recorded to provide a comprehensive picture of the farming environment.
\end{itemize}

One of the major benefits of these sensors is their configurability. They can be tailored to collect data at various spatial scales—from detailed individual observations to broader herd-level trends—and temporal resolutions, whether through continuous real-time monitoring or periodic measurements. This scalability ensures that both micro-level behaviours and macro-level patterns are captured effectively. 

By leveraging these advanced sensor systems, precision agriculture can provide a more objective, scalable, and comprehensive means of monitoring livestock health, thereby addressing some of the inherent limitations of conventional observational methods.

Although sensors deliver raw observational data that is closely linked to the emergence and spread of infectious diseases, a conceptual gap remains. The challenge lies in processing this unstructured, multi-dimensional data to extract meaningful insights and translate them into actionable recommendations.

\textit{\textbf{Keywords:}} sensors, automated data collection, multi-modal datasets, scalable, customizable.

% if you have more examples of sensors used in precision agriculture, then i shoudl insert it here. I don't want to mention anyy modelling strategies here nor do i want to mention any, just want to stay limited on the sensors and hardware so do you have any more ideas ? 

\subsection{Epidemiological modelling: Stochastic mechanistic models}

Epidemiological models serve as valuable tools for understanding, predicting, and controlling the spread of infectious diseases. By combining theoretical knowledge of disease transmission dynamics with empirical observations, these models provide insights that guide public health and veterinary interventions.

\paragraph{Fundamental Concepts and Historical Context.} Epidemiological modelling has deep roots in classical infectious disease research, with early examples focusing on simple compartmental models (e.g., SIR, SEIR). Over time, these models have evolved to capture more nuanced disease dynamics and real-world complexities. They explicitly represent the biological and ecological processes governing infection spread, such as contact rates, recovery patterns, and immunity mechanisms. By embedding domain knowledge, mechanistic models are interpretable and allow researchers to dissect how various factors influence disease trajectories.

\paragraph{Stochastic Mechanistic Models.} Stochastic approaches extend mechanistic models by incorporating random events—such as unpredictable contact opportunities, mutations, or interventions—thereby accounting for variability in disease spread. Key features include granular representation of uncertainty (Randomness in infection times and transmission events is captured, leading to a range of possible outcomes rather than a single deterministic trajectory). Stochastic models can better handle heterogeneous populations (e.g., different farms, varying animal susceptibility), especially when data are limited or events are sporadic. Scenario Exploration : Researchers can run multiple simulations under different assumptions or interventions, providing valuable insights into potential disease outcomes without requiring large-scale, real-world experimentation.

Interpretability and Biological Realism: By embedding expert knowledge and explicit mechanisms, these models help highlight the interactions and feedback loops that drive disease progression.

Hypothesis Testing and Policy Guidance: Stochastic mechanistic models allow for the evaluation of “what-if” scenarios, enabling more informed decisions on control measures—such as vaccination strategies or biosecurity protocols.

Although powerful, stochastic mechanistic models can become mathematically complex.

\textit{\textbf{Keywords:}} epidemiological stochastic mechanistic models, explicit representation of underlying mechanisms, incorporation of as much knowledge as we want, explainable predictions and insights, evidence-based recommendations, extrapolation


\subsubsection{Identifiability and observability}

Identifiability and observability are fundamental concepts in epidemiological modelling and statistical inference. They describe our ability to determine meaningful information about a system based solely on available observations (relire l'article de nik cunniffe et frederic hamelin)

Identifiability: Refers to whether the parameters of a model can be uniquely and accurately estimated from the available observations. In epidemiological modelling, a model is identifiable if different parameter values always lead to distinct observational outcomes. A lack of identifiability implies multiple plausible explanations exist for the same observed data, making reliable inference difficult.

Observability: Closely related to identifiability, observability refers specifically to the possibility of inferring the internal state of a system from available observations. An observable epidemiological model allows researchers to infer hidden states (e.g., the number of infected or susceptible animals at a given time) from external observations (e.g., clinical signs, ultrasound imaging, sensor data).

The challenges of identifiability and observability are central when modelling livestock diseases, as they directly affect the reliability of predictions, diagnoses, and recommended interventions. When parameters in a mechanistic epidemiological model cannot be reliably identified, the predictions become uncertain, undermining effective disease management.

Challenges of using epidemiological mechanistic models in Livestock farming:

\begin{itemize}
    \item Parameter inference, conversely, aims to estimate model parameters directly from context-specific observational data. It typically provides more precise results tailored to the studied population or outbreak but requires sufficient data quantity and quality, as well as more computational resources and advanced inferential methods. Real-world epidemiological models often have many parameters. As model complexity increases, the parameter space grows exponentially, a phenomenon known as the "curse of dimensionality." This makes it increasingly difficult to accurately infer parameters from observational data, especially when the quantity or quality of data is limited. 
    \item Manual Processing and Scalability Issues: often, methods including visual analysis or expert-driven parameter selection, are time-consuming and hard to scale. Manual identification of observable patterns or features from raw data becomes impractical when dealing with extensive datasets, particularly data derived from videos, audio recordings, or continuous sensor measurements. Livestock observational data, such as ultrasound imaging or audio recordings of animal coughs, often contain high levels of noise and unstructured information. Such data are challenging to interpretate even by experts, severely limiting scalability and accuracy.
\end{itemize}

\textit{\textbf{Keywords:}} Parameter identification, fitting to non-structured observations, differentiating models, manual model fitting does not scale-up.

Even thought these modelling approach are arguably the best to study and control complex diseases, they still have a few but critical challenges to be tackled. Being able to exploit their knowledge on non-structure observations (which are very frequent in the real-world) would also make them more widely used. 


\subsection{Machine learning modelling: Deep learning models}
Machine learning is a subfield of artificial intelligence involving algorithms that allow computers to learn patterns or relationships directly from data. The defining characteristic of machine learning models is their ability to improve their predictive performance through exposure to increasing amounts of data, rather than through explicitly programmed instructions. Within machine learning, deep learning models represent a sophisticated subclass, specifically designed to handle complex, high-dimensional, and unstructured data. Deep learning employs artificial neural networks—computational models inspired by biological neural systems—which consist of interconnected processing units called neurons arranged in layered architectures. This hierarchical structure enables deep learning models to learn intricate data representations at multiple abstraction levels.

In this thesis, we give extra attention to deep learning architectures, for their high performance in data-mining, pattern extraction in unstructured data which are an efficient format for storing high dimensions observations in their closest form to the original state: "one image contains a thousand words". 
Deep learning models are exceptionally powerful in tasks requiring automatic pattern extraction, and they are frequently described as universal approximators, capable of capturing non-linear relationships and subtle features from large and varied datasets. Typical examples include:

\begin{itemize}
    \item Image classification: Identifying objects or diseases from medical or veterinary images (e.g., ultrasound images).
    \item Natural language processing (NLP): Understanding and generating human language, such as sentiment analysis and text classification.
    \item Speech recognition: Automatically transcribing spoken audio signals.
\end{itemize}

There are methods implemented to enhance the explainability of deep learning models (XAI):
\begin{itemize}
    \item Grad-Cam and LIME produce saliency maps. They provide visual explanations by highlighting important regions of an input image that contribute to a neural network's prediction. Grad-CAM is specific to convolutional neural networks (CNNs) and generates heatmaps by computing gradients of the predicted class score relative to the CNN's intermediate convolutional layers, emphasizing regions that strongly influence the prediction. In contrast, LIME is model-agnostic, working by creating local perturbations around the input data, fitting a simple interpretable model (such as a linear regression) to approximate the behaviour of the original model, and highlighting input regions that most significantly affect the prediction. They are several other methods, such as DeepLIFT (Deep Learning Important Features, SmoothGrad, occlusion sensitivity). 
    \item Uncertainty quantification such as Bayesian deep learning or conformal predictions. There is no rivalry. They are radically different frameworks achieving very different things. CP is about calibrating predictions, while the Bayesian framework is about quantifying uncertainty of various things from the model, parameters, and consequently, the predictions. The typical guarantee you get from the Bayesian framework is that your estimates/predictions achieve the smallest average loss over the prior. That means Bayesian estimators are optimal estimators in terms of average performance, meaning they will be accurate, but do not guarantee coverage. CP does nothing about the accuracy of your estimator but gives you coverage. You can combine both and have the best of both worlds. One can understand this from a PAC-Bayesian perspective: CP is related to the empirical risk, while BDL is related to the KL divergence between the posterior and prior. 
\end{itemize}

These examples illustrate the flexibility and adaptability of deep learning across multiple application domains, including precision agriculture and epidemiological diagnostics. Despite their remarkable capabilities, deep learning models present certain limitations critical in epidemiological and livestock farming contexts. 

\textit{\textbf{Keywords:}} Deep learning, automated pattern extraction, handles unstructured observation, most powerful function approximator, intrapolation, uncertainty quantification, XAI


\subsubsection{Limitations and Challenges of classical Deep Learning Models}

% Interpolative vs extrapolative knowledge

Deep learning methods typically assume the data to be independently and identically distributed (iid), meaning each data point is generated from the same probability distribution and independently from one another. Deviations from this assumption can significantly degrade model performance and generalization capabilities.

% \begin{itemize}
%     \item Lack of Interpretability and Biophysical Meaning: Deep learning models are often described as "black-box" models due to their complexity and limited interpretability. The learned parameters (weights and biases) generally do not possess direct biological or mechanistic meaning, making it difficult to understand why a model produces a certain prediction.
%     \item Overfitting and Generalization: With high capacity and flexibility comes the risk of overfitting, where the model learns dataset-specific noise or irrelevant patterns rather than meaningful generalizable trends. Consequently, achieving robust generalization performance beyond the training set can be challenging.
%     \item Requirement for Extensive and Representative Data: Deep learning models rely heavily on datasets that comprehensively represent the underlying phenomena of interest. Insufficient or biased training data can lead to suboptimal generalization, especially in disease diagnostics, where precise and representative datasets are rare, difficult to collect, or ethically challenging to generate (e.g., intentionally inducing disease outbreaks).
%     \item Sensitivity to Distribution Shifts (Out-of-Distribution Data): Deep learning models typically struggle when encountering data outside their training distribution (Out-of-Distribution, OOD). This limitation is particularly pertinent in epidemiology, where the appearance of previously unobserved disease variants or changes in environmental conditions frequently occur.
%     \item these methods can also quickly become very resource intensive particularly when dealing with complex task as often the architecture also get more complex, requiring bigger infrastructure (gpu, cpu) for training or inference, these also has an impact on environmental management.
% \end{itemize}

XAI methods are a way to measure and understand why predictions are made, They are usually are employed for outliers detection, data drift, ...  however they cannot be used to precisely forecast extreme scenarios. 

Each of these methodologies individually addresses certain specific aspects of infectious disease modelling effectively, but they also carry inherent limitations when applied alone:
\begin{itemize}
    \item Deep learning models excel at extracting intricate patterns from large-scale or high-dimensional observational data (e.g., sensor-derived data such as audio, video, or biological signals). Their data-driven nature allows them to effectively capture context-specific and complex relationships without explicit theoretical assumptions. Nevertheless, their main limitation lies in their lack of explicit biological or theoretical interpretability, limiting their capability to reliably extrapolate to unseen scenarios or out-of-distribution contexts.
    \item Stochastic mechanistic epidemiological models explicitly encode theoretical or biological mechanisms (infection transmission, recovery rates, or disease progression) using interpretable parameters with clearly defined biological meanings. These models can thus generate hypotheses, facilitate understanding of causal dynamics, and provide robust extrapolation to scenarios not explicitly observed in the data. However, their main challenge lies in effectively incorporating detailed, unstructured, and context-specific observational data (such as sensor data), limiting their practical utility and responsiveness to specific epidemiological contexts.
\end{itemize}

% Even thought these model are widely used to process complex datasets, they are hardly used to explore novel situations or predict task that are inherently has multiple complex relationships (so hard to gather a representative dataset). these challenges essentially stem from the lack of explicit knowledge incorporation (example hallucinations in LLMs). They are in their need of feature that could  enhance interpretability, improve generalization in OOD settings, and facilitate more accurate and meaningful insights by explicitly incorporating domain-specific knowledge into the learning process.


% do models like PINN require more data ? 
% otherwise the advantage of what we are proposing will be on the structure i guess. A modular method, that could require a bit less since we are training a a less complex task. at yet maybe as performant ? 

% prmireè formulation: a methodology that can be used to process non-structure datasets and still easily be used to extrapolate towards unknown complex scenarios.
% second thing incorporate more knowledge into the observation that are gathered because sometimes the observations are noisy and don't represent the global mechanisms. 

\textit{\textbf{Keywords:}} Generalization, forecast extreme scenarios, input dataset has to be iid, lack of biophysical meaning of parameters, black-box modelling, data intensive for complex tasks, resource intensive (GPU, electricity, materials...) for complex tasks, false positive, false negative, extrapolation


\subsection{Hybrid modelling: machine learning and epidemiological models}

[Raccourci cette partie pour garder le tableau, le plus important c'est de montrer les objectifs (summarize), les types de couplages (summarize), montrer les axes que nous mettons en avant]
En terme d'objectif voici les cases que nous cochons: model parameterizination, disease intervention assessemnt and optimization, restrospective epidemic course analysis, infectious disease forecasting (par contre préciser qu'ici en réalité nous avons n'avons pas fait de forecast dans le sens statistique du terme jusqu'à aller à l'évaluation des performances, transmission inference, outbreak detection]

[Résumé les objectifs en un paragraphe. Etendre par contre le type de d'intégration et comment c'est fait. mettre égalemenet le rôle que joue chaque partie ?] [the roles are long-term forecast (or prognose) or short-term predictions]

[numerotez les approches d'intégration, ça va m'être utile pour la suite]

This section serves as a concise narrative review of various methodologies for integrating machine learning models with epidemiological models, offering a critical perspective on their limitations. Addressing these limitations is central to the contributions of this thesis.

This section presumes that the reader possesses a foundational understanding of ensemble techniques in modelling, specifically bagging, boosting, and stacking. They are pivotal in enhancing the global performance by combining multiple models. These methods aim to reduce errors, improve accuracy, and increase the robustness of predictions. Stacking Combines outputs from several distinct models using an additional "meta-model," learning optimal combinations for improved predictions. Boosting refers to sequentially training multiple weak predictive models, each aiming to correct errors from previous models, thus progressively improving overall prediction accuracy. Bagging (or Bootstrap Aggregating) refers to parallel training of multiple unique models (while stacking used distinct models, bagging uses multiple instances but of the same model) on random subsets of data (with replacement), thereby reducing prediction variance and mitigating over-fitting.

In the scientific literature, hybrid modelling approaches integrating deep learning and mechanistic models have gained attention, especially during epidemiological crises such as the COVID-19 pandemic. Several prominent methodologies in this domain include:

En sciences, toute tentative de tirer des conclusions généralisables nécessite un
modèle, i.e. une version abstraite de la réalité \cite{McCallum2008}. Cette représentation
abstraite est dotée d’un objectif (a model is a "purposeful representation", 
\cite{Starfield1990}, qui requiert un certain degré de simplification.

The purposes of the hybrid models currently found in the litterature can be grouped into 6 mutually exclusive groups, indicating that a single integrated model serve multiple application areas \cite{Ye2025} :
\begin{itemize}
    \item Infectious disease forecasting: Predicting the future spread or trajectory of infectious diseases by combining AI's data-mining strengths with the explanatory power of mechanistic epidemiological models. \cite{Poirier2020, Fan2022, Garner2016, Kharazmi2021, Barmparis2022}
    \item Model Parameterization and Calibration: Determining and refining model parameters (e.g., disease transmissibility, contact rates) accurately and efficiently, often by extracting additional insights from diverse and complex datasets. \cite{Tuarob2015, SolaresHernandez2023, Reiker2021, Jorgensen2022, Raissi2019}
    \item Disease Intervention Assessment and Optimization: Evaluating and identifying optimal strategies for interventions (such as vaccinations, social distancing, or lockdowns) to minimize disease spread and impact, utilizing methods like reinforcement learning and game theory. \cite{Thiagarajan2022, Bertsimas2021, Aurell2022, Janko2023}
    \item Retrospective Epidemic Course Analysis: Understanding past epidemics by analyzing historical data, identifying critical factors influencing transmission dynamics, and informing future outbreak preparedness. \cite{Ruth2022, Santermans2015, Sanson2022, Weyant2023}
    \item Transmission Inference: Inferring hidden patterns, source origins, and transmission networks using AI techniques trained on data from mechanistic epidemiological simulations or real-world data. \cite{Ling2022, Colijn2014, Vilar2023}
    \item Outbreak Detection:Quickly identifying new outbreaks or unusual surges in cases, enabling timely public health responses through AI-enhanced analysis of varied data sources, such as medical reports or social media content.
\end{itemize}


Methodologicaly-wise, these models can also grouped into nine primary integration approach:
\begin{itemize}
    \item AI-augmented epidemiological models: These models directly replace (or augment) specific components (e.g., parameters or even entire sub-models) of the mechanistic framework with machine learning outputs. In many cases, the machine learning component is trained end‐to‐end alongside the mechanistic model or is integrated into the numerical solver (réfs heres). In an end-to-end configuration, the machine learning module’s weights are updated simultaneously with the mechanistic model’s parameters, creating a tightly coupled structure that learns both representation and dynamics jointly. 
    \item Epidemiological models with improved observational data: In these approaches, AI is applied to augment or “clean” raw data inputs before they feed into the mechanistic model. Techniques include using machine learning to extract features from social media, search trends, or satellite imagery(réfs heres). Machine learning operates as a pre-processor and the mechanistic model then uses the enhanced, higher-quality data for calibration and simulation. This is typically a weakly coupled integration because the machine learning module’s output is treated as a fixed input to the downstream mechanistic model rather than being jointly optimized with it.
    \item Physics-infromed neural networks (PINNs) and Epidemiology-aware AI models (EAAMs): These models embed the differential equations governing disease transmission directly into the loss function of the neural network (réfs heres). PINNs are effective when generalised to some PDEs, in particular nonlocal or high-dimensional PDEs, for which traditional solvers are computationally expensive. However, \cite{Kidger2021}  emphasises that this is a distinct notion to AI-augmented epidemiological models (AIEM). AIEMs use neural networks to specify differential equations. PINNs uses neural networks to obtain solutions to prespecified differential equations. This distinction is a common point of confusion, especially as the PDE equivalent of PINNs is sometimes referred to as a `neural partial differential equation'.
    \item Machine learning models incorporating epidemiological input features: In this configuration, traditional machine learning architectures (like recurrent neural networks or graph neural networks) are augmented by explicitly incorporating epidemiological features (e.g., contact rates, incubation periods) as additional inputs or via modified architectures. This represents a moderate or weak coupling since the epidemiological knowledge is “injected” as additional features rather than being deeply embedded in the model’s structure or loss function.
    \item Surrogate modelling/synthetically-trained ai models: Surrogate models use machine learning to approximate the outputs of computationally expensive mechanistic models. Typically, synthetic datasets are generated from simulations of the mechanistic model, and an machine learning model (often a neural network) is then trained on these data. The coupling here is often “loose” or weak because the AI surrogate is a stand-alone approximation that is later used in place of the full mechanistic model.
    \item Ensemble Learning Frameworks: Ensemble methods combine predictions from mechanistic models and machine learning models (or multiple ML models) using techniques like weighted averaging, stacking. The coupling is relatively loose since each model is usually developed independently and then combined at a later stage.
    \item Bayesian Neural Networks: These hybrid approaches integrate Bayesian inference with deep learning to quantify uncertainty in parameter estimation. In the epidemiological context, Bayesian neural networks are used to learn parameters while simultaneously providing credible intervals for model outputs.The coupling is inherent in the training process as the posterior distribution is approximated jointly using both data likelihood and prior mechanistic information.
    \item ML-Enhanced Optimization Frameworks: These methods use AI algorithms—such as reinforcement learning, optimal control, or Markov decision processes—to optimize intervention strategies based on outputs from mechanistic models.In an end-to-end framework, the AI agent interacts with the mechanistic simulation in a closed loop, receiving feedback and updating its strategy iteratively (tight coupling).Alternatively, the optimization can be performed in a decoupled fashion where the mechanistic model is used to generate scenarios and the AI module optimizes the intervention based on pre-computed results (weaker coupling).
    \item Cluster-Based Transmission Analysis Frameworks: This approach uses unsupervised learning methods (such as k-means clustering) to segment populations or regions with similar transmission characteristics. Once clusters are identified, mechanistic models are applied to each subgroup to refine forecasts or intervention strategies.
\end{itemize}

These findings can be summarized into a table \ref{tab:integration_methods}:

\begin{table}[htbp]
\centering
\resizebox{\textwidth}{!}{%
\begin{tabular}{|p{4cm}|p{5cm}|p{3cm}|p{4cm}|p{4cm}|p{4cm}|p{6cm}|}
\hline
\textbf{Integration approach} & \textbf{Objective} & \textbf{Coupling Type} & \textbf{ML Architecture} & \textbf{Epidemiological Architecture} & \textbf{Aggregation type} & \textbf{Full Reference} \\ \hline

AI-augmented epidemiological models &
Infectious disease forecasting, Model parametrisation and calibration &
Strong coupling &
Neural networks, regression, tree-based models replacing parts of the traditional models for parameter estimation &
Mechanistic epidemiological models with components replaced or augmented with by ML &
End-to-end training &
Arik et al. (2021) \\ \hline

Epidemiological models with improved observational data &
Outbreak detection, model parametrization and calibration &
Weak coupling &
methods like SVM , tree-based, NLP, MLP algorithms used to extract and refine features from non tradictional surveillance data (text) &
Bayesian epidemiological frameworks (or classical compartmental or mechanistic models updated using enhanced observational inputs derived from diverse data sources) &
Stacking, Bayesian model selection &
Tuarob et al. (2015), Rosato et al.(2023), Kandula et al.(2019) \\ \hline

PINNs (physics-informed neural networks) / EAAMs (Epidemiology-Aware AI models) &
Model parametrisation and calibration, Infectious Disease Forecasting &
Strong coupling &
Neural networks (DNN, CNN, RNN, GNN) that incorporate physics-based loss functions (combining data loss and residual loss enforcing differential equations) &
Differential equations-based mechanistic models(Compartmental, renewal equations e.g: SIR, etc) &
End-to-End training &
Cai, M., Karniadakis, G. E., \& Li, C. (2022). Fractional SEIR model and data-driven predictions of COVID-19 dynamics of Omicron variant. \textit{Chaos}, 32, 071101; 

Ghosh, S. et al. (2023). Understanding the implications of under-reporting, vaccine efficiency, and social behaviour on the post-pandemic spread using physics informed neural networks. \textit{PLoS One}, 18, e0290368. Barmparis \& Tsironis (2022), De rosa et al. (2023) \\ \hline

AI models incorporating epidemiological input features &
Infectious disease forecasting &
Weak coupling &
ML models (e.g: FFN, RNN, Tree-based models) that integrate epidemioloigcal indicators as part of their input features &
Mechanistic epidemiological models supplying key epidemiological metrics (e.g contact rates, parameters) as inputs to the AI model &
Feature-level integration &
- \\ \hline

Surrogate modelling / synthetically-trained AI models &
Model parametrisation and calibration, Disease intervention assessment and optimization, Retrospective epidemic course analysis &
Weak coupling &
Lightweight neural networks or 
Tree-based models trained on synthetic datasets &
simulation-based mechanistic models (compartmental or Agent-based models) for synthetic generation &
Surrogate modelling &
Anirudh, R. et al. (2022). Accurate Calibration of Agent-based Epidemiological Models with Neural Network Surrogates. \textit{Proceedings of the 1st Workshop on Healthcare AI and COVID-19, ICML 2022}, vol. 184, pp. 54–62, PMLR. \\ \hline

Ensemble learning frameworks &
Infectious disease forecasting &
Weak coupling &
combinations of models such as LTSM, stacking and boosting methods to merge prediction from different sources &
Traditional mechanistic models whose forecasts are combined with AI-based predictions &
Weighted averaging, stacking, boosting &
Kandula et al. (2018), Adiga et al. (2021), Nadler et al. (2020),
Maniamfu \& Kameyama (2023), Delli compagni et al. (2022)\\ \hline

Bayesian neural networks &
Model parameterisation and calibration &
Strong coupling &
Bayesian neural networks enabling uncertainty quantification in parameters estimation &
mechanistic model providing a probabilistic structure (e.g SIR models with uncertainty in parameters) Agent-based models, Ordinary differential equations &
Bayesian inference &
Kwok, W. M., Streftaris, G., \& Dass, S. C. (2023). Laplace based Bayesian inference for ordinary differential equation models using regularized artificial neural networks. \textit{Stat. Comput.}, 33, 124. \\ \hline

AI-enhanced optimization frameworks &
Disease intervention assessment and optimization &
Strong coupling &
Reinforcement learning, neural networks, optimal control neural networks, and game-theoretic models designed to learn optimal intervention strategies &
often agent-based epidemiological models that serve as environments for optimization and decision-making &
Reinforcement learning, optimal control &
Yien et al. (2023), Jian et al. (2023), Malloy \& Brandeau (2022), Zang et al. (2022), Janko et al. (2023) \\ \hline

Cluster-based transmission analysis frameworks &
Infectious disease forecasting, Transmission inference &
Weak coupling &
Clustering algorithm (k-means, Hierarchical clustering...) and network analysis methods (Graph neural networks) to identify and analyze clusters in transmission data &
Network-based or agent-based models that capture transmission dynamics and suppot the identification of key transmission clusters
Individual-based models &
Clustering & - \\ \hline
\end{tabular}%
}
\caption{Summary of nine primary integration method types.}
\label{tab:integration_methods}
\end{table}

% je pourrais rajouter deux colonnes pour les avantages et les limites ou les architectures utilisés pour le deep comme pour les modèles epidelio

\begin{table}[htbp]
\centering
\resizebox{\textwidth}{!}{%
\begin{tabular}{|p{4cm}|p{5cm}|p{3cm}|p{4cm}|p{4cm}|p{4cm}|p{6cm}|}
\hline
\textbf{Integration approach} & \textbf{Objective} & \textbf{Description} & \textbf{training process}  & \textbf{References} \\ \hline


PINNs (Physics-Informed Neural Networks) &
Model parametrisation and calibration, infectious disease forecasting &
A neural network is trained to approximate the state variables (such as susceptible, infected, and recovered populations) as functions of time. The key design feature is the incorporation of the governing differential equations (for example, those from a classical SIR model) directly into the loss function. This composite loss includes a data loss term (comparing network predictions to observed time-series data) and a residual loss term that forces the solution to satisfy the epidemiological equations &

Training Process: Gradient-based optimization (using methods like Adam or SGD) is applied to minimize the combined loss function, ensuring that both the data fit and the differential constraints are respected &

Original Citations: References [32, 59–66] \\ \hline

PINNs (Physics-Informed Neural Networks) &
Model parametrisation and calibration, infectious disease forecasting &
A neural network is trained to approximate the state variables (such as susceptible, infected, and recovered populations) as functions of time. The key design feature is the incorporation of the governing differential equations (for example, those from a classical SIR model) directly into the loss function. This composite loss includes a data loss term (comparing network predictions to observed time-series data) and a residual loss term that forces the solution to satisfy the epidemiological equations &

Training Process: Gradient-based optimization (using methods like Adam or SGD) is applied to minimize the combined loss function, ensuring that both the data fit and the differential constraints are respected &

Original Citations: References [32, 59–66] \\ \hline

\end{tabular}%
}
\caption{Summary of nine primary integration method types.}
\label{tab:integration_methods}
\end{table}







\begin{itemize}
    \item Physics-Informed Neural Networks (PINNs): PINNs integrate known governing equations (e.g., differential equations describing infectious dynamics) as soft constraints into neural networks during training. The neural network is guided by both observed data and theoretical knowledge, ensuring physically plausible predictions and improved extrapolation beyond the training distribution.
    \item Neural Differential Equations (Neural ODEs/SDEs): Neural ODEs or SDEs embed neural networks into differential equations, allowing the modelling of dynamic processes through continuous-time representations. Such integration directly captures temporal dynamics, offering more flexibility and interpretability compared to classical discrete-time deep learning methods.
\end{itemize}

Despite their promising advantages, hybrid and ensemble models have inherent challenges:
% Despite notable advancement in the scientific community, there is a conceptual gap, between the raw unstructured livestock data that can be captured and the strong theoretical knowledge we have of the complex underlying infectious mechanisms. This basically allows to ground the theoretical knowledge we have on a complex mechanisms to the context-specific observations, hence harness the best of both world. 

\textit{\textbf{Keywords:}} Ensemble learning, Physics-informed neural networks, Neural differential equations


Each ensemble strategy presents specific strengths and limitations, related to computational complexity, interpretability, and predictive power. 
research gaps:
\begin{itemize}
    \item Generalizability: The majority of integrated models focus on diseases like COVID-19, limiting applicability to other infectious diseases with indirect or complex transmission routes.
    \item Expansion beyond COVID-19 to other diseases with indirect transmission mechanisms.
    \item Data Limitations: Many studies depend on synthetic or limited real-world data, highlighting the need for richer, real-time data integration (e.g., satellite imagery, social media data). - we have gathered in this thesis a real-wolrd multi-modal dataset (...)
\end{itemize}

\subsubsection{Maintainability}

    \begin{itemize}
        \item strongly coupled methods requires the builder to be deeply understand the problematic of domain in order to implement the models. (full stack vs frontend or backend). So retraining for example would require the whole system to be trained and in the long-term, it would also require sort of full-stack builders. 
        
        \item strongly coupled methods would still require a  certain quantity of data since in order for example for the deep learning model to fine-tune the mechanistic model (neural differential equations). Usually ensembled model (so more parameters) would require more dimensions in order to fit all the parameters (curse of dimensionality)
    \end{itemize}

% le domaine est très large, mais nous on va se focaliser uniquement sur le couplage deep et modèle méchaniste de façon générale. 

% sachant que nous c'est une forme de boosting, sans trop de redondances,) et surtout découplé (faiblement couplé) - ceci sont des couplages forts mais comme un site web à egalement ces limites. comme un développeur full stack - parler rapidement des NDE, des pinns qui sont juste une sous familles des NDE.

\textit{\textbf{Keywords:}} ...



%----------------------------------------------------------------------------------------
%	SECTION 
%----------------------------------------------------------------------------------------
\clearpage 

\section{Thesis objective and outline}

\subsection{Exploring the complementarities between deep learning and mechanistic epidemiological models}

Precision agriculture provides powerful tools enabling automation of real-world observations through various sensors. Throughout this thesis, we have leveraged such tools to acquire contextual observational data essential for studying an infectious disease in livestock farming. However, sensor-based observations, although rich and increasingly accessible, represent only partial information regarding complex and unpredictable disease dynamics, aptly summarized by Yoan Bourhis (2017): "Nos observations ne révèlent que la partie émergée d’un iceberg au comportement complexe et peu prévisible."

Thus, a central question guiding this thesis is: How can sensor observations be effectively employed to study infectious diseases and support informed decision-making?

Our main hypothesis is that the most scientifically robust approach to contemporary quantitative questions in animal health, particularly regarding livestock infectious diseases, lies in combining complementary artificial intelligence methods, specifically deep learning and mechanistic epidemiological models. Such integration leverages deep learning’s capabilities for processing and extracting short-term insights from unstructured (video, image or text) observations  and the extrapolative capacities of stochastic epidemiological models grounded in explicit theoretical knowledge for long-term insights. This combination aims to better link real-world observations obtained through sensors with our theoretical knowledge in order make make relevant evidence-based recommendations at a larger temporal scales.

This naturally raises another foundational question addressed in this thesis:
In what ways can deep learning complement mechanistic epidemiological models in epidemiology ?

Complex animal health problems, including the study and control of infectious diseases, require distinct yet interconnected types of expertise: diagnosing diseases from immediate observational data, making reliable prognoses about future disease dynamics, and ultimately providing actionable recommendations. Diagnosis relies predominantly on processing unstructured field observations from sensors—thus favouring deep learning. Robust prognosis, however, relies on explicit theoretical knowledge and interpretability—domains inherently suited to mechanistic models. Finally, the quality of actionable recommendations is critically dependent on effectively bridging these two forms of expertise.

This thesis proposes a loosely coupled methodology inspired by the statistical principle known as the "Mixture of Experts" (MoE). This modular integration allows each expert to specialize explicitly in its distinct role (diagnosis and prognosis), enhancing accuracy, resource efficiency, interpretability, and scalability.

Addressing these considerations, the scientific questions explored throughout this thesis are:

\begin{enumerate}
    \item To what extent can deep learning reliably automate short-term diagnosis using limited, noisy, and context-specific observational data from sensors, such as lung ultrasounds ? 
    
    \item How can mechanistic epidemiological models be reliably parametrized using empirical veterinary observations to provide accurate long-term prognosis for infectious diseases ?
    
    % \item Given multiple mechanistic epidemiological models validly representing different expertise symptomatic dynamics of infectious diseases, how can observational data alone reliably guide the selection of the most appropriate mechanistic model to enable pathogen-specific disease management ?
    
    % \item How can observational data alone guide the selection of the best mechanistic prognosis expert when they are multiple epidemiological models expert for expliciting different mechanisms of the infectious disease.
    
    \item Given multiple mechanistic epidemiological models representing different but valid expertise in prognosing an outbreak, how can observational data alone reliably guide the selection of the most appropriate mechanistic model to enable pathogen-specific disease management.

    \item How can deep learning and mechanistic models be effectively integrated into a hybrid diagnostic-to-prognostic pipeline that leverages their complementary strengths to improve livestock disease management ?
    
    \item How can uncertainties inherent in sensor-based observations be explicitly accounted for within a hybrid modelling approach, and how does this influence diagnostic and prognostic reliability ?
\end{enumerate}


\textit{\textbf{Keywords:}} methodological synergy, complementary expertise, deep learning, mechanistic epidemiological models, diagnostic accuracy, prognostic reliability, uncertainty quantification, Mixture of Experts, modular architecture, scalability.

\subsection{Application to study Bovine Respiratory Diseases}
% In this subsection, I want to show that BRD are a good example for the application of our methodology

Bovine Respiratory Diseases (BRD) refer to a group of complex, multifactorial infectious disorders predominantly affecting young cattle in fattening farms. They are characterized by inflammation of the respiratory tract, causing symptoms such as cough, nasal discharge, fever, reduced feed intake, impaired growth, and occasionally, death. Although several pathogens (viruses, bacteria, mycoplasma) contribute to BRD, their clinical presentation is frequently non-specific, complicating accurate and timely diagnosis.

Diagnosing and managing BRD effectively remains notoriously challenging for several reasons:
\begin{itemize}
    \item Non-specific Clinical Signs: Clinical manifestations of BRD (e.g., cough, fever) are highly unspecific and overlap significantly with other diseases. Consequently, visual appraisal by farmers and veterinarians often results in misdiagnoses or delayed diagnoses, leading to suboptimal treatment strategies.
    \item Limitations of Biological Diagnostic Methods: Laboratory methods (e.g., Polymerase Chain Reaction (PCR), serology) provide increased specificity and accuracy compared to clinical appraisal alone. However, these tests are invasive, expensive, and time-consuming, delaying actionable results and increasing animal stress and discomfort. Moreover, logistical issues frequently limit their practicality, especially in large-scale operations.
    \item False Positives and Diagnostic Uncertainty: Due to the multifactorial nature of BRD (co-infections, pathogen interactions, host susceptibility variability), diagnostic and prognostic accuracy remain challenging. These difficulties result in inappropriate usage of antimicrobials, contributing to the rising threat of antimicrobial resistance and negatively impacting animal welfare.
\end{itemize}

(Complexity of BRD Etiology) BRD arises from complex interactions between intrinsic and extrinsic factors:
\begin{itemize}
    \item Pathogen Diversity and Interactions: Multiple pathogens (e.g., Mannheimia haemolytica, Pasteurella multocida, Bovine Respiratory Syncytial Virus, etc.) are frequently involved, potentially interacting in complex and poorly understood ways. Current veterinary research continues to explore these interactions to better characterize clinical markers useful for early detection, prognosis, and improved control measures (as exemplified in recent doctoral works, e.g., Maud’s research).
    \item Influence of Environmental and Management Practices: External factors such as farm management, biosecurity measures, herd density, transportation stress, and climatic conditions profoundly influence the occurrence and severity of BRD outbreaks. This intrinsic and extrinsic complexity significantly complicates disease modelling, prognosis, and control efforts.
\end{itemize}


(Socioeconomic Impact of BRD in Livestock Farming) Bovine Respiratory Diseases represent a major health and economic burden for farmers, veterinarians, and the broader livestock industry:
\begin{itemize}
    \item Economic Costs and Mortality Rates: BRD accounts for substantial economic losses in terms of reduced growth performance, increased mortality rates, and heightened veterinary and medicinal expenses. Particularly in French beef fattening farms, BRD is considered one of the most prevalent and economically significant animal health problems.
    \item Antimicrobial Usage and Ethical Concerns: Frequent misdiagnosis or delayed interventions lead to inappropriate use of antibiotics, fostering antimicrobial resistance. This concern raises ethical, public health, and animal welfare issues and highlights the urgent need for improved diagnostics and targeted therapeutic approaches.
\end{itemize}

(Existing Technological Approaches and Limitations) Recent research has applied sensor technologies (e.g., intra-ruminal temperature sensors, accelerometers, audio and video analytics) coupled with traditional machine learning and deep learning methods to improve BRD detection. Although promising, these data-driven approaches often:
\begin{itemize}
    \item Exhibit high false-positive rates due to limited specificity in clinical signs or ambiguous sensor outputs.
    \item Require substantial volumes of training data to achieve reliable performance, a constraint given practical difficulties in generating extensive labelled datasets.
    \item Struggle to predict disease progression or forecast epidemiological outcomes accurately over extended periods, thus limiting their use in proactive disease management and intervention strategies.
    \item Lack the capability to explore unobservable scenarios, such as hypothetical outbreaks or unrecorded infections, limiting their utility for scenario-based disease control planning.
\end{itemize}

There are also been mechanistic models developed before and throughout this thesis to model and study BRD (see Originality of this thesis). They have never applied to real-world observations. In this thesis, we employed these models to assess our methodology.

By bridging sophisticated deep learning feature extraction with robust, interpretable mechanistic models, this hybrid approach could significantly advance the ability to manage BRD effectively—improving animal health, welfare, farm economics, and sustainability and ecological issues. 

This thesis leverages BRD as a scientifically significant case study to validate a hybrid deep-mechanistic methodology, explicitly addressing the limitations noted above, with the aim of substantially improving diagnosis, prognosis, and disease management strategies in livestock farming.

\textit{\textbf{Keywords:}} Bovine Respiratory Disease, infectious disease dynamics, antimicrobial resistance, multi-modal data integration, predictive analytics, animal welfare.


\subsection{Originality of this thesis}


\paragraph{interdisciplinarity: synergy of diverse domain expertise }
% In this subsection, I want to explain the thesis CIFRE, with the mixture of domain expertise: epidemiological mechanistic modelling, statistical inference approaches, computer vision and deep learning, hardware and software engineering Mais également la collaboration avec les vétos. Préciser que c'est une thèse cifre (ce que peut apporter/ et les gains en retours pour adventiel: les côté applicatif, igepp (deep), Dynamo (mécaniste, inférence...). It is original to have as many different domain experts come together to work on one subject right ?

Uniquely structured via a CIFRE agreement, this thesis integrates expertise from diverse domains:  
\begin{itemize}
    \item Adventiel: Providing strong expertise in software and hardware engineering for precision agriculture, particularly focusing on practical applications, technical robustness, and user-friendly decision support tools.
    \item BIOEPAR-dynamo: Offering significant theoretical and applied expertise in mechanistic epidemiological modelling and statistical inference methods, including parameter inference and calibration techniques, tailored specifically to livestock disease dynamics. Emulsion (generic simulation engine for epidemiological mechanistic models)
    \item IGEPP-demecologie: Contributing substantial expertise in statistical inference, deep learning, and computer vision methodologies.
    \item Collaborations established through multi-partner projects such as SEPTIME and MULTIPAST, involving key contributors (e.g., Baptiste-Sorin), enhance the thesis’s capacity to integrate different forms of scientific expertise.
\end{itemize}
Such interdisciplinary collaboration enhances methodological robustness and practical relevance, facilitating broader acceptance among farmers and veterinary stakeholders.


\paragraph{Data collection: enriching empirical knowledge}

A significant originality of this thesis is the comprehensive observational dataset collected specifically to study BRD. This dataset, comprising multi-modal sensor data, lung ultrasound videos, and expert veterinary annotations, simultaneously addresses fundamental scientific questions and practical agricultural needs, potentially informing innovative and practical decision-support tools.

\begin{itemize}
    \item descrire ici la mise en place du protocol experimental avec la collecte de données pour répondre à des questions de biologiques sur le diagnostique et le prognostique de BRD (thèse maud) mais également des questions de méthodo modélisation (deep et méca)
    \item One major originality is the comprehensive and detailed collection of observational data from real livestock farms, specifically tailored to study Bovine Respiratory Diseases (BRD). The thesis provides explicit descriptions of this extensive dataset, composed of multi-modal sensor data, video recordings, and expert veterinary annotations
    \item The collected dataset enables exploration of both fundamental scientific questions and applied research inquiries, potentially leading to the development of innovative, practical decision support tools applicable directly within the livestock industry. (citer la thèse de maud, car elle utilise ces données afin d'accroître la connaissance sur l'identification de biomarqueurs des BRD) 
\end{itemize}


\paragraph{Methodology: diagnosis and prognosis expertise}

This thesis proposes an original methodological framework that combines deep learning and mechanistic epidemiological modelling, with articulated contributions:
\begin{itemize}
    \item Automated Diagnosis from limited and Noisy Observational Data from a sensor: Demonstrating the feasibility and robustness of deep learning (CNN-RNN) approaches to automatically diagnose Bovine Respiratory Disease (BRD) using unstructured, context-specific sensor data (lung ultrasound videos), achieving reliable diagnostic accuracy despite limited data availability.
    \item automated prognosis from limited observations: establishing a robust methodological framework to independently parametrize and calibrate stochastic mechanistic models directly from empirical veterinary observations collected on-farm. This significantly enhances the identifiability, predictive accuracy, and practical relevance of long-term epidemiological forecasts for BRD management.
    \item Introducing clear numerical methods (Approximate Bayesian Computation with multinomial logistic regression) for reliably selecting among multiple competing mechanistic epidemiological models based solely on symptomatic observational data. This enables accurate pathogen-specific model identification, substantially reducing antibiotic misuse and improving farm economic outcomes.
    \item Structured Deep Mechanistic Modelling for Adaptive Knowledge Integration: proposing and validating a structured hybrid modelling pipeline (Bayesian Deep Mechanistic approach) explicitly linking deep learning-generated diagnostic information to mechanistic epidemiological prognosis. This novel approach grounds theoretical epidemiological knowledge directly within realistic, unstructured sensor observations, thereby providing a comprehensive, adaptive methodological baseline.
    \item Proxy Robustness and Explicit Uncertainty Quantification: Enhancing hybrid model reliability by explicitly quantifying and incorporating uncertainties inherent in noisy sensor observations (through Bayesian methods). This methodological improvement significantly reduces diagnostic and prognostic errors, thereby mitigating negative impacts arising from observational uncertainty.
    \item Modularity and Methodological Flexibility: Emphasizing methodological modularity, this thesis demonstrates how domain experts (veterinarians, deep learning specialists, mechanistic modellers) can independently develop, maintain, retrain, and adapt each modelling component. Such modularity contrasts favourably with tightly integrated approaches (e.g., Neural Differential Equations or Physics-Informed Neural Networks), offering significant advantages in interpretability, scalability, ease of use, reduced data requirements, and enhanced generalizability across diverse epidemiological contexts.
\end{itemize}


\textit{\textbf{Keywords:}} Hybrid modelling, Deep learning, Mechanistic epidemiological modelling, Automated diagnosis, parametrization, Model identifiability, Model distinguishability, Bayesian inference, Observational uncertainty, Robust diagnostics, Proxy robustness, Modularity, Mixture-of-Experts, Sensor data integration, Knowledge coherence, Unstructured observational data, Model identifiability, Adaptive epidemiological forecasting, Interpretability, Methodological flexibility.


\subsection{About the methodological approach}

The thesis structure progresses methodologically across three chapters:

Chapter 2 - Foundational structures: independent diagnosis and prognosis expertise. 
This chapter assesses independently the performance of the deep learning model in automating the BRD diagnosis from lung ultrasound video data, reaching an accuracy of 72\%. It also evaluates a stochastic mechanistic epidemiological model parametrized by veterinarian-provided clinical observations, confirming its utility for robust long-term BRD prognosis, albeit with moderate calibration precision due to observational data scarcity and inherent uncertainties in observations. Demonstrating feasibility of deep learning diagnosis from limited, real-world sensor data and creating an original annotated dataset of lung ultrasound observations, forming an empirical foundation for further research. This directly addresses scientific questions 1 and 2.

Chapter 3 - Structural synergism – Selecting appropriate mechanistic prognosis experts. This chapter addresses a critical methodological gap: distinguishing among multiple valid mechanistic models, each suited to distinct pathogen-specific scenarios. Employing synthetic outbreak scenarios and a Bayesian inference framework, the chapter demonstrates how symptomatic dynamics can reliably inform pathogen-model identification. Integrating this approach with bioeconomic evaluations, we quantify the tangible benefits (improved net profits and reduced antimicrobial usage) resulting from pathogen-informed antibiotic treatment decisions. This directly addresses scientific question 3.

Chapter 4 - A deep mechanistic approach.  This chapter proposes a Bayesian deep mechanistic approach explicitly integrating observational uncertainties into both diagnostic and prognostic stages. Employing Monte Carlo Dropout (MCD) within the deep learning model, we quantify uncertainty in lung ultrasound observations and propagate it into mechanistic model calibration through uncertainty-weighted inference. This approach reduces diagnostic uncertainty (error rate reduced from 39\% to 27.2\% RRMSE), significantly enhancing model robustness and reliability for practical livestock management scenarios. This integration enhances decision-making robustness and aligns closely with real-world constraints where sensor observations are often noisy or incomplete. thus explicitly addressing scientific questions 4 and 5 by demonstrating how uncertainty-informed hybrid methodologies enhance practical livestock management reliability.

General discussion - The final section synthesizes the findings across all chapters, critically evaluating the methodological approaches, their strengths and limitations, and the broader implications of the results. Recommendations for future research and applications are also discussed, highlighting the potential for scalability and interdisciplinary adaptation.




\section{Thesis objective and outline}

\subsection{Exploring the complementarities between deep learning and mechanistic epidemiological models}

Precision agriculture provides powerful tools enabling automation of real-world observations through various sensors. Throughout this thesis, we have leveraged such tools to acquire contextual observational data essential for studying an infectious disease in livestock farming. However, sensor-based observations, although rich and increasingly accessible, represent only partial information regarding complex and unpredictable disease dynamics, aptly summarized by Yoan Bourhis (2017): "Nos observations ne révèlent que la partie émergée d’un iceberg au comportement complexe et peu prévisible."

Thus, a central question guiding this thesis is: How can sensor observations be effectively employed to study infectious diseases and support informed decision-making?

Our main hypothesis is that the most scientifically robust approach to contemporary quantitative questions in animal health, particularly regarding livestock infectious diseases, lies in combining complementary artificial intelligence methods, specifically deep learning and mechanistic epidemiological models. Such integration leverages deep learning’s capabilities for processing and extracting short-term insights from unstructured (video, image or text) observations  and the extrapolative capacities of stochastic epidemiological models grounded in explicit theoretical knowledge for long-term insights. This combination aims to better link real-world observations obtained through sensors with our theoretical knowledge in order make make relevant evidence-based recommendations at a larger temporal scales.

This naturally raises another foundational question addressed in this thesis:
In what ways can deep learning complement mechanistic epidemiological models in epidemiology ?

Complex animal health problems, including the study and control of infectious diseases, require distinct yet interconnected types of expertise: diagnosing diseases from immediate observational data, making reliable prognoses about future disease dynamics, and ultimately providing actionable recommendations. Diagnosis relies predominantly on processing unstructured field observations from sensors—thus favouring deep learning. Robust prognosis, however, relies on explicit theoretical knowledge and interpretability—domains inherently suited to mechanistic models. Finally, the quality of actionable recommendations is critically dependent on effectively bridging these two forms of expertise.

This thesis proposes a loosely coupled methodology inspired by the statistical principle known as the "Mixture of Experts" (MoE). This modular integration allows each expert to specialize explicitly in its distinct role (diagnosis and prognosis), enhancing accuracy, resource efficiency, interpretability, and scalability.

Addressing these considerations, the scientific questions explored throughout this thesis are:

\begin{enumerate}
    \item To what extent can deep learning reliably automate short-term diagnosis using limited, noisy, and context-specific observational data from sensors, such as lung ultrasounds ? 
    
    \item How can mechanistic epidemiological models be reliably parametrized using empirical veterinary observations to provide accurate long-term prognosis for infectious diseases ?
    
    % \item Given multiple mechanistic epidemiological models validly representing different expertise symptomatic dynamics of infectious diseases, how can observational data alone reliably guide the selection of the most appropriate mechanistic model to enable pathogen-specific disease management ?
    
    % \item How can observational data alone guide the selection of the best mechanistic prognosis expert when they are multiple epidemiological models expert for expliciting different mechanisms of the infectious disease.
    
    \item Given multiple mechanistic epidemiological models representing different but valid expertise in prognosing an outbreak, how can observational data alone reliably guide the selection of the most appropriate mechanistic model to enable pathogen-specific disease management.

    \item How can deep learning and mechanistic models be effectively integrated into a hybrid diagnostic-to-prognostic pipeline that leverages their complementary strengths to improve livestock disease management ?
    
    \item How can uncertainties inherent in sensor-based observations be explicitly accounted for within a hybrid modelling approach, and how does this influence diagnostic and prognostic reliability ?
\end{enumerate}


\textit{\textbf{Keywords:}} methodological synergy, complementary expertise, deep learning, mechanistic epidemiological models, diagnostic accuracy, prognostic reliability, uncertainty quantification, Mixture of Experts, modular architecture, scalability.

\subsection{Application to study Bovine Respiratory Diseases}
% In this subsection, I want to show that BRD are a good example for the application of our methodology

Bovine Respiratory Diseases (BRD) refer to a group of complex, multifactorial infectious disorders predominantly affecting young cattle in fattening farms. They are characterized by inflammation of the respiratory tract, causing symptoms such as cough, nasal discharge, fever, reduced feed intake, impaired growth, and occasionally, death. Although several pathogens (viruses, bacteria, mycoplasma) contribute to BRD, their clinical presentation is frequently non-specific, complicating accurate and timely diagnosis.

Diagnosing and managing BRD effectively remains notoriously challenging for several reasons:
\begin{itemize}
    \item Non-specific Clinical Signs: Clinical manifestations of BRD (e.g., cough, fever) are highly unspecific and overlap significantly with other diseases. Consequently, visual appraisal by farmers and veterinarians often results in misdiagnoses or delayed diagnoses, leading to suboptimal treatment strategies.
    \item Limitations of Biological Diagnostic Methods: Laboratory methods (e.g., Polymerase Chain Reaction (PCR), serology) provide increased specificity and accuracy compared to clinical appraisal alone. However, these tests are invasive, expensive, and time-consuming, delaying actionable results and increasing animal stress and discomfort. Moreover, logistical issues frequently limit their practicality, especially in large-scale operations.
    \item False Positives and Diagnostic Uncertainty: Due to the multifactorial nature of BRD (co-infections, pathogen interactions, host susceptibility variability), diagnostic and prognostic accuracy remain challenging. These difficulties result in inappropriate usage of antimicrobials, contributing to the rising threat of antimicrobial resistance and negatively impacting animal welfare.
\end{itemize}

(Complexity of BRD Etiology) BRD arises from complex interactions between intrinsic and extrinsic factors:
\begin{itemize}
    \item Pathogen Diversity and Interactions: Multiple pathogens (e.g., Mannheimia haemolytica, Pasteurella multocida, Bovine Respiratory Syncytial Virus, etc.) are frequently involved, potentially interacting in complex and poorly understood ways. Current veterinary research continues to explore these interactions to better characterize clinical markers useful for early detection, prognosis, and improved control measures (as exemplified in recent doctoral works, e.g., Maud’s research).
    \item Influence of Environmental and Management Practices: External factors such as farm management, biosecurity measures, herd density, transportation stress, and climatic conditions profoundly influence the occurrence and severity of BRD outbreaks. This intrinsic and extrinsic complexity significantly complicates disease modelling, prognosis, and control efforts.
\end{itemize}


(Socioeconomic Impact of BRD in Livestock Farming) Bovine Respiratory Diseases represent a major health and economic burden for farmers, veterinarians, and the broader livestock industry:
\begin{itemize}
    \item Economic Costs and Mortality Rates: BRD accounts for substantial economic losses in terms of reduced growth performance, increased mortality rates, and heightened veterinary and medicinal expenses. Particularly in French beef fattening farms, BRD is considered one of the most prevalent and economically significant animal health problems.
    \item Antimicrobial Usage and Ethical Concerns: Frequent misdiagnosis or delayed interventions lead to inappropriate use of antibiotics, fostering antimicrobial resistance. This concern raises ethical, public health, and animal welfare issues and highlights the urgent need for improved diagnostics and targeted therapeutic approaches.
\end{itemize}

(Existing Technological Approaches and Limitations) Recent research has applied sensor technologies (e.g., intra-ruminal temperature sensors, accelerometers, audio and video analytics) coupled with traditional machine learning and deep learning methods to improve BRD detection. Although promising, these data-driven approaches often:
\begin{itemize}
    \item Exhibit high false-positive rates due to limited specificity in clinical signs or ambiguous sensor outputs.
    \item Require substantial volumes of training data to achieve reliable performance, a constraint given practical difficulties in generating extensive labelled datasets.
    \item Struggle to predict disease progression or forecast epidemiological outcomes accurately over extended periods, thus limiting their use in proactive disease management and intervention strategies.
    \item Lack the capability to explore unobservable scenarios, such as hypothetical outbreaks or unrecorded infections, limiting their utility for scenario-based disease control planning.
\end{itemize}

There are also been mechanistic models developed before and throughout this thesis to model and study BRD (see Originality of this thesis). They have never applied to real-world observations. In this thesis, we employed these models to assess our methodology.

By bridging sophisticated deep learning feature extraction with robust, interpretable mechanistic models, this hybrid approach could significantly advance the ability to manage BRD effectively—improving animal health, welfare, farm economics, and sustainability and ecological issues. 

This thesis leverages BRD as a scientifically significant case study to validate a hybrid deep-mechanistic methodology, explicitly addressing the limitations noted above, with the aim of substantially improving diagnosis, prognosis, and disease management strategies in livestock farming.

\textit{\textbf{Keywords:}} Bovine Respiratory Disease, infectious disease dynamics, antimicrobial resistance, multi-modal data integration, predictive analytics, animal welfare.


\subsection{Originality of this thesis}


\paragraph{interdisciplinarity: synergy of diverse domain expertise }
% In this subsection, I want to explain the thesis CIFRE, with the mixture of domain expertise: epidemiological mechanistic modelling, statistical inference approaches, computer vision and deep learning, hardware and software engineering Mais également la collaboration avec les vétos. Préciser que c'est une thèse cifre (ce que peut apporter/ et les gains en retours pour adventiel: les côté applicatif, igepp (deep), Dynamo (mécaniste, inférence...). It is original to have as many different domain experts come together to work on one subject right ?

Uniquely structured via a CIFRE agreement, this thesis integrates expertise from diverse domains:  
\begin{itemize}
    \item Adventiel: Providing strong expertise in software and hardware engineering for precision agriculture, particularly focusing on practical applications, technical robustness, and user-friendly decision support tools.
    \item BIOEPAR-dynamo: Offering significant theoretical and applied expertise in mechanistic epidemiological modelling and statistical inference methods, including parameter inference and calibration techniques, tailored specifically to livestock disease dynamics. Emulsion (generic simulation engine for epidemiological mechanistic models)
    \item IGEPP-demecologie: Contributing substantial expertise in statistical inference, deep learning, and computer vision methodologies.
    \item Collaborations established through multi-partner projects such as SEPTIME and MULTIPAST, involving key contributors (e.g., Baptiste-Sorin), enhance the thesis’s capacity to integrate different forms of scientific expertise.
\end{itemize}
Such interdisciplinary collaboration enhances methodological robustness and practical relevance, facilitating broader acceptance among farmers and veterinary stakeholders.


\paragraph{Data collection: enriching empirical knowledge}

A significant originality of this thesis is the comprehensive observational dataset collected specifically to study BRD. This dataset, comprising multi-modal sensor data, lung ultrasound videos, and expert veterinary annotations, simultaneously addresses fundamental scientific questions and practical agricultural needs, potentially informing innovative and practical decision-support tools.

\begin{itemize}
    \item descrire ici la mise en place du protocol experimental avec la collecte de données pour répondre à des questions de biologiques sur le diagnostique et le prognostique de BRD (thèse maud) mais également des questions de méthodo modélisation (deep et méca)
    \item One major originality is the comprehensive and detailed collection of observational data from real livestock farms, specifically tailored to study Bovine Respiratory Diseases (BRD). The thesis provides explicit descriptions of this extensive dataset, composed of multi-modal sensor data, video recordings, and expert veterinary annotations
    \item The collected dataset enables exploration of both fundamental scientific questions and applied research inquiries, potentially leading to the development of innovative, practical decision support tools applicable directly within the livestock industry. (citer la thèse de maud, car elle utilise ces données afin d'accroître la connaissance sur l'identification de biomarqueurs des BRD) 
\end{itemize}


\paragraph{Methodology: diagnosis and prognosis expertise}

This thesis proposes an original methodological framework that combines deep learning and mechanistic epidemiological modelling, with articulated contributions:
\begin{itemize}
    \item Automated Diagnosis from limited and Noisy Observational Data from a sensor: Demonstrating the feasibility and robustness of deep learning (CNN-RNN) approaches to automatically diagnose Bovine Respiratory Disease (BRD) using unstructured, context-specific sensor data (lung ultrasound videos), achieving reliable diagnostic accuracy despite limited data availability.
    \item automated prognosis from limited observations: establishing a robust methodological framework to independently parametrize and calibrate stochastic mechanistic models directly from empirical veterinary observations collected on-farm. This significantly enhances the identifiability, predictive accuracy, and practical relevance of long-term epidemiological forecasts for BRD management.
    \item Introducing clear numerical methods (Approximate Bayesian Computation with multinomial logistic regression) for reliably selecting among multiple competing mechanistic epidemiological models based solely on symptomatic observational data. This enables accurate pathogen-specific model identification, substantially reducing antibiotic misuse and improving farm economic outcomes.
    \item Structured Deep Mechanistic Modelling for Adaptive Knowledge Integration: proposing and validating a structured hybrid modelling pipeline (Bayesian Deep Mechanistic approach) explicitly linking deep learning-generated diagnostic information to mechanistic epidemiological prognosis. This novel approach grounds theoretical epidemiological knowledge directly within realistic, unstructured sensor observations, thereby providing a comprehensive, adaptive methodological baseline.
    \item Proxy Robustness and Explicit Uncertainty Quantification: Enhancing hybrid model reliability by explicitly quantifying and incorporating uncertainties inherent in noisy sensor observations (through Bayesian methods). This methodological improvement significantly reduces diagnostic and prognostic errors, thereby mitigating negative impacts arising from observational uncertainty.
    \item Modularity and Methodological Flexibility: Emphasizing methodological modularity, this thesis demonstrates how domain experts (veterinarians, deep learning specialists, mechanistic modellers) can independently develop, maintain, retrain, and adapt each modelling component. Such modularity contrasts favourably with tightly integrated approaches (e.g., Neural Differential Equations or Physics-Informed Neural Networks), offering significant advantages in interpretability, scalability, ease of use, reduced data requirements, and enhanced generalizability across diverse epidemiological contexts.
\end{itemize}


\textit{\textbf{Keywords:}} Hybrid modelling, Deep learning, Mechanistic epidemiological modelling, Automated diagnosis, parametrization, Model identifiability, Model distinguishability, Bayesian inference, Observational uncertainty, Robust diagnostics, Proxy robustness, Modularity, Mixture-of-Experts, Sensor data integration, Knowledge coherence, Unstructured observational data, Model identifiability, Adaptive epidemiological forecasting, Interpretability, Methodological flexibility.


\subsection{About the methodological approach}

The thesis structure progresses methodologically across three chapters:

Chapter 2 - Foundational structures: independent diagnosis and prognosis expertise. 
This chapter assesses independently the performance of the deep learning model in automating the BRD diagnosis from lung ultrasound video data, reaching an accuracy of 72\%. It also evaluates a stochastic mechanistic epidemiological model parametrized by veterinarian-provided clinical observations, confirming its utility for robust long-term BRD prognosis, albeit with moderate calibration precision due to observational data scarcity and inherent uncertainties in observations. Demonstrating feasibility of deep learning diagnosis from limited, real-world sensor data and creating an original annotated dataset of lung ultrasound observations, forming an empirical foundation for further research. This directly addresses scientific questions 1 and 2.

Chapter 3 - Structural synergism – Selecting appropriate mechanistic prognosis experts. This chapter addresses a critical methodological gap: distinguishing among multiple valid mechanistic models, each suited to distinct pathogen-specific scenarios. Employing synthetic outbreak scenarios and a Bayesian inference framework, the chapter demonstrates how symptomatic dynamics can reliably inform pathogen-model identification. Integrating this approach with bioeconomic evaluations, we quantify the tangible benefits (improved net profits and reduced antimicrobial usage) resulting from pathogen-informed antibiotic treatment decisions. This directly addresses scientific question 3.

Chapter 4 - A deep mechanistic approach.  This chapter proposes a Bayesian deep mechanistic approach explicitly integrating observational uncertainties into both diagnostic and prognostic stages. Employing Monte Carlo Dropout (MCD) within the deep learning model, we quantify uncertainty in lung ultrasound observations and propagate it into mechanistic model calibration through uncertainty-weighted inference. This approach reduces diagnostic uncertainty (error rate reduced from 39\% to 27.2\% RRMSE), significantly enhancing model robustness and reliability for practical livestock management scenarios. This integration enhances decision-making robustness and aligns closely with real-world constraints where sensor observations are often noisy or incomplete. thus explicitly addressing scientific questions 4 and 5 by demonstrating how uncertainty-informed hybrid methodologies enhance practical livestock management reliability.

General discussion - The final section synthesizes the findings across all chapters, critically evaluating the methodological approaches, their strengths and limitations, and the broader implications of the results. Recommendations for future research and applications are also discussed, highlighting the potential for scalability and interdisciplinary adaptation.



\section{[In french] Résumé grand public}

La gestion des maladies infectieuses en élevage bovin s’inscrit dans un système d’une grande complexité, en raison de facteurs intrinsèques (par exemple la virulence des pathogènes et la sensibilité des animaux) et extrinsèques (pratiques d’élevage, conditions environnementales) en constante évolution. Face à cette complexité, l’intelligence artificielle (IA) émerge comme une approche prometteuse pour modéliser les dynamiques épidémiques et anticiper leur évolution via des simulations, fournissant ainsi des outils d’aide à la décision aux éleveurs, vétérinaires et autres acteurs. Parallèlement, l’essor de l’agriculture de précision se traduit par le déploiement de capteurs capables de surveiller en continu des variables physiologiques individuelles et des conditions environnementales, et de générer des alertes rapides (en quelques heures ou jours) sur des événements critiques tels que le vêlage, les chaleurs, le bien-être ou la santé des animaux.

Néanmoins, s’agissant des maladies infectieuses en élevage, ces alertes issues de capteurs souffrent d’une faible spécificité et génèrent un taux élevé de faux positifs. Ces fausses alarmes imposent une charge mentale importante aux éleveurs, qui finissent soit par les ignorer faute de pertinence, soit par réaliser des interventions inutiles et coûteuses. Ainsi, concevoir des méthodologies innovantes capables de transformer ces signaux peu spécifiques en recommandations précises et actionnables sur des horizons de temps plus longs (plusieurs jours à plusieurs semaines) reste un défi de recherche ouvert. C’est précisément ce défi que cette thèse entreprend de relever, en posant la question centrale: comment exploiter efficacement les observations issues de capteurs pour étudier les maladies infectieuses et appuyer des décisions éclairées en élevage bovin ?

Pour y répondre, cette thèse postule qu’il est optimal d’intégrer des approches d’IA complémentaires, en l’occurrence les réseaux de neurones profonds et les modèles épidémiologiques mécanistes. La stratégie proposée capitalise d’une part sur la capacité des modèles mécanistes à représenter explicitement les processus épidémiques aux différentes échelles de temps en mobilisant les connaissances vétérinaires et biologiques, et d’autre part sur la puissance du deep learning pour extraire automatiquement des descripteurs pertinents à partir de données massives et hétérogènes issues des capteurs (images, sons, etc.). Le couplage de ces deux approches doit permettre une meilleure intégration des observations réelles (issues des capteurs, mais aussi des retours d’éleveurs et de vétérinaires) dans les prédictions épidémiques à court terme comme à moyen et long terme. 

La problématique choisie pour appliquer cette méthodologie est celle des maladies respiratoires bovines (en anglais *Bovine Respiratory Disease*, BRD) chez les jeunes bovins de boucherie. La BRD constitue en effet le principal problème de santé dans les ateliers d’engraissement, avec des conséquences sanitaires et économiques majeures. Elle ralentit la croissance et la productivité des animaux, induit des frais vétérinaires et médicamenteux importants, et provoque une mortalité non négligeable (environ 3\% en moyenne). C’est également la première cause d’usage d’antibiotiques en élevage bovin, avec près de 20\% des bovins à l’engraissement recevant un traitement contre la BRD. 

L’étiologie de la BRD est multifactorielle, résultant d’interactions complexes entre de nombreux facteurs. Côté animal, la susceptibilité dépend de la race, de l’état immunitaire et de la co-infection par divers agents pathogènes (notamment les bactéries *Mannheimia haemolytica* et *Pasteurella multocida*, ou des virus comme le virus respiratoire syncytial bovin), dont les interactions restent encore mal élucidées. Côté élevage et environnement, des facteurs de risque tels que le stress du transport, la densité des animaux, la gestion de l’alimentation, les conditions de logement, les protocoles de biosécurité ou le climat influencent fortement l’apparition et la gravité de la maladie. La conjonction de ces facteurs rend la prédiction et le contrôle de la BRD particulièrement hasardeux sur le terrain et complique sa modélisation épidémiologique. 

En outre, la détection précoce de la BRD s’avère particulièrement ardue. Ses manifestations cliniques initiales — toux, écoulement nasal, fièvre, anorexie, léthargie, retard de croissance — sont peu spécifiques et peuvent facilement passer inaperçues, d’autant que les bovins ont tendance à dissimuler les signes de maladie aux premiers stades (comportement de proie). Les méthodes traditionnelles de surveillance visuelle présentent ainsi une sensibilité et une spécificité limitées (de l’ordre de 60 à 65\% seulement), ce qui conduit à de fréquentes erreurs de diagnostic: des cas infectés peuvent ne pas être détectés à temps (faux négatifs), et inversement des animaux sains sont parfois traités à tort (faux positifs). Ces difficultés sont exacerbées par la pénurie de vétérinaires en zones rurales, qui restreint la possibilité d’une surveillance rapprochée et régulière des troupeaux.

La détection de la BRD pourrait néanmoins bénéficier des avancées récentes en élevage de précision, grâce au suivi continu de la santé des animaux par divers capteurs. Des accéléromètres, microphones, thermomètres connectés ou caméras permettent de mesurer des signaux physiologiques et comportementaux associés à la maladie, tels que la température corporelle, les patterns d’activité ou les sons respiratoires. Par exemple, environ 73\% des épisodes de fièvre (hyperthermie) chez des veaux à l’engraissement coïncident avec une BRD, ce qui suggère qu’une surveillance de la température peut être un indicateur utile. Plus récemment, un modèle statistique (régression logistique) exploitant des données de collier d’activité, de podomètre et de bolus intra-ruminal a atteint environ 75\% de sensibilité et 76\% de spécificité pour prédire l’apparition de signes cliniques de BRD jusqu’à 24 heures à l’avance. De même, des capteurs accélérométriques fixés sur les oreilles ont permis de détecter des changements de comportement (activité, rumination) distinguant clairement des veaux malades et sains, illustrant le potentiel des capteurs pour une détection plus précoce des maladies respiratoires bovines.

Pourtant, ces approches purement basées sur les capteurs demeurent limitées par l’utilisation de modèles d’apprentissage automatique traditionnels, peu aptes à extrapoler au-delà des situations déjà observées. Leur capacité à prévoir l’évolution d’une épidémie dans des contextes différents ou sur le long terme est réduite, ce qui limite leur utilité pour orienter les décisions de gestion sanitaire à moyen ou long échéance. Par ailleurs, il est éthiquement et pratiquement impossible de rassembler des données exhaustives couvrant tous les scénarios de BRD (en particulier les cas sévères) dans les conditions réelles d’élevage, ce qui freine inévitablement les méthodes purement empiriques. 

En complément des capteurs, la modélisation épidémiologique mécaniste apporte une solution prometteuse pour dépasser ces limites observationnelles. Par exemple, une étude *in silico* récente a identifié des stratégies optimales de gestion de la BRD fondées sur des alertes capteurs. Cependant, les modèles mécanistes employés étaient calibrés à partir de données de la littérature vétérinaire plutôt que de données empiriques issues du terrain, introduisant des incertitudes quant à leur validité pratique. De plus, certaines solutions actuelles de détection reposent sur des dispositifs invasifs ou coûteux, d’où l’importance d’explorer des alternatives non invasives et abordables (analyse automatique d’images, d’enregistrements audio, etc.) pour une adoption à large échelle. 

La stratégie de la thèse s’inscrit dans cette perspective et mise sur une forte interdisciplinarité en couplant apprentissage profond et modélisation mécaniste pour améliorer la détection et la prédiction de la BRD. Ce travail a été mené dans le cadre d’une convention CIFRE, en partenariat étroit entre la société Adventiel et l’organisme de recherche INRAE, ce qui a favorisé le lien entre recherche académique et application industrielle. Adventiel, entreprise française spécialisée dans les solutions numériques pour l’agriculture, a apporté son expertise en intelligence artificielle appliquée (vision par ordinateur, analyse de signaux) ainsi que son infrastructure technologique (serveurs de calcul, stockage de données) pour la collecte et le traitement des observations issues des capteurs. 

Du côté de l’INRAE, l’unité de recherche BIOEPAR (équipe DYNAMO) a fourni le cadre de modélisation mécaniste avec l’outil EMULSION — une plateforme à base de systèmes multi-agents et d’un langage dédié facilitant le développement de modèles épidémiologiques — et a partagé un premier modèle mécaniste de la BRD servant de base à cette étude. L’expertise vétérinaire et épidémiologique de BIOEPAR sur la BRD (connaissances cliniques, immunologiques et socio-économiques) a largement orienté la conception du modèle et l’interprétation des données. En parallèle, l’équipe Démécologie (unité IGEPP, INRAE) a contribué des méthodes statistiques avancées pour l’estimation des paramètres, la quantification des incertitudes et l’inférence bayésienne, afin d’aborder les défis liés à l’ajustement des modèles sur les données réelles. La synergie de ces compétences variées – modélisation, deep learning, statistique, expertise vétérinaire de terrain – confère à cette recherche un caractère original et illustre la convergence de multiples expertises autour d’un même objectif. 

D’un point de vue empirique, une composante importante de la thèse a été la constitution d’un jeu de données multimodal inédit sur la BRD en conditions d’élevage réel. Cette collecte a eu lieu dans le cadre du projet collaboratif SEPTIME (Carnot «France Futur Élevage»), impliquant l’INRAE (BIOEPAR) et l’Institut de l’Élevage (Idele). Elle s’est déroulée sur neuf exploitations d’engraissement bovin réparties dans différentes régions, lors de deux campagnes correspondant aux arrivées typiques de jeunes bovins: de janvier à juin 2023, puis d’octobre 2023 à janvier 2024. Ces périodes ont été choisies car les premières semaines suivant l’introduction d’animaux dans un nouveau troupeau sont connues pour présenter un risque élevé de BRD. Sur chaque élevage, un à trois lots de 5 à 12 bovins ont été suivis pendant 30 jours dès leur arrivée. Environ 78\% des animaux étaient de race Charolaise, ce choix facilitant l’observation visuelle de certains symptômes grâce à la robe claire de ces bovins.

Le protocole de suivi mis en place combinait plusieurs capteurs et des examens vétérinaires réguliers. Chaque ferme était équipée d’une caméra vidéo fixe enregistrant un extrait de 5 minutes chaque heure en journée (de 9h à 18sh), d’un microphone synchronisé capturant les sons en parallèle de la vidéo (notamment la toux) et d’un capteur environnemental mesurant en continu la température ambiante, l’humidité, le $CO_2$ et l’ammoniac ($NH_3$). Parallèlement, les animaux ont été examinés par un vétérinaire environ tous les deux jours durant le mois suivant leur arrivée. Lors de ces visites, un examen clinique était réalisé (observation du comportement, détection de signes respiratoires tels que fatigue, écoulements oculaires ou nasaux, prise de la température rectale) et des prélèvements biologiques étaient effectués: analyses sanguines et écouvillonnages nasaux pour détecter les pathogènes respiratoires par PCR. En complément, des échographies pulmonaires ont été pratiquées à des jours prédéfinis (le jour de l’arrivée, puis les jours 5, 14, 21 et 28) afin d’évaluer visuellement l’état des poumons et de confirmer d’éventuelles lésions. L’ensemble des données collectées (vidéos, audio, mesures environnementales, observations cliniques, résultats de laboratoire et imagerie médicale) a été automatiquement transmis et stocké de façon sécurisée sur les serveurs d’Adventiel, constituant une base empirique riche et synchronisée pour les analyses ultérieures. Ce jeu de données unique, alliant signaux de capteurs et diagnostics vétérinaires détaillés, fournit un fondement solide pour répondre aux questions scientifiques tout en restant ancré dans les besoins concrets de l’élevage.

La démarche méthodologique de la thèse se décline en trois étapes complémentaires, correspondant aux chapitres principaux, afin d’apporter successivement des éléments de réponse à la question posée. (1) Tout d’abord (Chapitre 2), chaque approche a été évaluée séparément pour en établir la faisabilité et les limites propres. D’un côté, nous explorons dans quelle mesure un modèle d’apprentissage profond peut automatiser le diagnostic de la BRD à court terme à partir de données de capteurs limitées et spécifiques (en particulier l’analyse de vidéos d’échographie pulmonaire). De l’autre, nous examinons si des observations vétérinaires de terrain peuvent servir à paramétrer un modèle épidémiologique mécaniste afin de fournir un pronostic de la maladie sur le long terme. Les résultats obtenus confirment l’intérêt de chaque approche. Un réseau de neurones profond entraîné sur les séquences d’échographies pulmonaires parvient à détecter des lésions de BRD avec environ 72\% de précision, malgré les conditions d’acquisition variées sur le terrain. Parallèlement, un modèle épidémiologique mécaniste calibré à partir des observations cliniques réussit à reproduire les tendances de l’infection sur plusieurs semaines, et ce malgré le caractère parcellaire des données disponibles. Ce double constat fournit un socle empirique pour l’analyse intégrée et s’accompagne de contributions notables, telle que la constitution d’un jeu de données original d’échographies pulmonaires annotées pour la BRD – un atout précieux pour de futurs travaux en diagnostic vétérinaire assisté par IA. Il met en lumière la complémentarité des approches: les tâches de diagnostic instantané sont confiées au deep learning, excellent pour extraire des caractéristiques complexes d’images ou de sons, tandis que le pronostic à long terme est dévolu au modèle mécaniste, qui s’appuie sur les bases théoriques épidémiologiques pour simuler l’évolution de la maladie dans le temps.

(2)La deuxième étape (Chapitre 3) aborde la variabilité des agents pathogènes pouvant être en cause dans la BRD et l’impact de cette variabilité sur le pronostic. Plusieurs modèles mécanistes spécifiques peuvent en effet être envisagés selon le pathogène prédominant (par exemple un modèle calibré pour le virus BRSV, et d’autres pour les bactéries *M. haemolytica* ou *P. multocida*). Il devient alors crucial de déterminer, à partir des seuls symptômes observés, quel agent prédomine afin de choisir le modèle de prévision adéquat, et de vérifier si cette identification améliore les décisions sanitaires. Par analogie, tout comme un médecin généraliste oriente un patient vers un spécialiste approprié en fonction de ses symptômes, notre système doit pouvoir sélectionner le «modèle spécialiste» (viral ou bactérien) correspondant à la situation réelle afin d’affiner ses prédictions. 

Pour ce faire, nous avons mis en œuvre une approche bayésienne originale combinant une méthode d’inférence par ABC (*Approximate Bayesian Computation*) et une régression logistique multinomiale. Cette méthode permet de différencier avec environ 93\% d’exactitude entre plusieurs scénarios simulés de BRD, en identifiant lequel des pathogènes principaux (virus BRSV, *M. haemolytica*, *M. bovis*, etc.) correspond le mieux à la trajectoire de symptômes observée. Surtout, le fait d’intégrer la reconnaissance de l’agent causal dans le modèle conduit à des recommandations de gestion plus efficaces. À l’aide d’un modèle bio-économique simulant le fonctionnement de l’élevage, nous montrons qu’en adaptant les interventions au pathogène identifié, il est possible de réduire d’environ 44\% le recours aux traitements antibiotiques, tout en améliorant légèrement la performance économique de l’atelier d’engraissement. Ce résultat illustre concrètement l’intérêt de lier étroitement la modélisation épidémiologique aux décisions de terrain: un pronostic mieux ciblé permet des actions plus pertinentes, bénéfiques à la fois pour la santé des animaux (moins de traitements inutiles) et pour la rentabilité de l’élevage.

(3) La troisième et dernière étape (Chapitre 4) réalise l’intégration effective des deux approches (diagnostic automatique et modélisation mécaniste) dans un cadre unifié, tout en gérant explicitement les incertitudes inhérentes aux données de capteurs. L’enjeu est double : utiliser les diagnostics automatisés à court terme issus des capteurs pour informer le modèle mécaniste en vue d’un pronostic à long terme, et tenir compte de l’incertitude de ces diagnostics pour garantir la fiabilité des prévisions. Pour cela, nous avons développé une approche hybride nommée *Bayesian Deep Mechanistic* (BDM), qui intègre les prédictions d’un modèle de deep learning (appliqué notamment aux vidéos d’échographie pulmonaire) au sein d’un modèle épidémiologique de manière probabiliste. Concrètement, chaque prédiction de l’IA est associée à un degré de confiance, estimé par la technique du *Monte Carlo dropout* afin de quantifier l’incertitude du réseau de neurones. Ces diagnostics «probabilisés» alimentent ensuite le modèle mécaniste de deux manières: soit en filtrant ou pondérant les observations en fonction de leur niveau d’incertitude (de façon à ne conserver que les informations jugées fiables), soit en intégrant directement cette incertitude dans l’inférence des paramètres du modèle. Un tel dispositif améliore nettement la précision et la robustesse du système global. Par exemple, l’erreur de prévision à long terme est réduite d’environ un tiers (de 39\% à 27\% d’erreur quadratique moyenne relative) lorsque l’on tient compte de l’incertitude des données capteurs dans le modèle. Ainsi, le cadre BDM rapproche les performances d’un pronostic automatisé de celles d’un expert humain, en combinant les atouts du deep learning et des modèles mécanistes tout en atténuant leurs faiblesses respectives grâce à une gestion rigoureuse des incertitudes. Il en résulte un outil d’aide à la décision fiable, transparent et adaptable pour le suivi des maladies infectieuses en élevage.

En synthèse, les travaux menés dans cette thèse démontrent l’intérêt d’approches d’IA hybrides pour l’étude et le contrôle des maladies infectieuses en élevage, en particulier dans le contexte de l’élevage de précision. L’étude de cas sur la BRD illustre comment le croisement du deep learning et de la modélisation mécaniste permet de dépasser les limites actuelles des capteurs en santé animale : on passe de simples alertes ponctuelles et peu spécifiques à un système intégré de diagnostic automatisé, de pronostic robuste à long terme et d’aide à la décision, le tout étayé par une quantification explicite de l’incertitude. Les résultats obtenus se positionnent par rapport à la littérature actuelle en apportant tout à la fois des avancées méthodologiques (intégration bayésienne innovante, différenciation de modèles en fonction des pathogènes) et des bénéfices concrets pour l’élevage (réduction de l’usage inutile d’antibiotiques, amélioration du bien-être animal et optimisation économique). Bien que développée sur la BRD, l’approche proposée revêt un caractère générique et pourrait être transposée à d’autres maladies infectieuses en élevage (voire en santé des plantes), dès lors que des données de capteurs sont disponibles pour alimenter les modèles. Ce travail, fruit d’une collaboration étroite entre acteurs académiques et industriels, ouvre ainsi de nouvelles perspectives pour des systèmes de santé prédictive en élevage, combinant intelligence artificielle et expertise métier afin d’aider les éleveurs et les vétérinaires à prendre des décisions éclairées.

% \chapter*{\Huge Big Title Here}
% \addcontentsline{toc}{chapter}{Big Title Here}  % Add to TOC if needed

\chapter{Foundational structures: diagnosis and prognosis experts} % Main chapter title
%---------------------------------------------------------------------------------------
%	SECTION 
%----------------------------------------------------------------------------------------

\section{Introduction}
%-----------%-----------
%	SOUS-SECTION 
%-----------%-----------
\subsection{Contextual background}

Ultrasonography, a common diagnostic modality in both human and veterinary medicine, is particularly valued for its non-invasiveness, portability, and rapid assessment capabilities. Specifically, in veterinary medicine, thoracic ultrasonography (TUS) is extensively employed as a quick, accurate, and practical method for diagnosing lung lesions associated with respiratory diseases such as Bovine Respiratory Disease (BRD). It provides an immediate, real-time assessment of lung pathology without the limitations of other imaging techniques such as radiography or computed tomography, which involve high costs, radiation exposure, and require specialized facilities and sedation or anesthesia \cite{ollivett_-farm_2016}. Pulmonary ultrasound videos provide a direct view of the lung tissue, making them highly relevant for assessing the severity of respiratory diseases like BRD. TUS can accurately identify critical pathological features associated with disease severity and prognosis, including consolidated lung tissue, lobar pneumonia, abscess formation, and pleural effusion. In feedlot cattle, Timsit et al. (2019) \cite{timsit_association_2019} demonstrated that the maximal depth and area of lung consolidation measured at the time of bronchopneumonia diagnosis using TUS were significantly associated with an increased risk of disease relapse and negatively impacted animal growth performance. \cite{timsit_association_2019} Interpreting ultrasound images poses challenges even for skilled experts due to inherent limitations of ultrasonography. Ultrasound images typically appear as noisy, black-and-white visuals that provide limited detail, capturing only two-dimensional representations of tissue shapes and echo patterns. The presence of artifacts such as comet tails or reverberations, as well as the subjective nature of differentiating between subtle variations in lung tissue echogenicity, further complicates the interpretation of lung ultrasounds. Additionally, lung lesions must reach the pleural surface to become visible ultrasonographically, limiting the detection of deeper pulmonary pathology. Despite these limitations, thoracic ultrasonography remains highly valuable due to its real-time diagnostic capability, portability, and practical utility in large-animal field settings, offering immediate insights into disease severity and prognosis when assessing BRD outbreaks on-farm.


%une des limitations aussi de ce type d'observations et qui motive également à aller vers du processing multimodal d'observations est que les lésions ou artéfacts vue dans les poumons sont souvent indicateurs d'évenements tardifs ? Ultrasound imaging can only detect lung lesions that extend to the pleural surface, meaning lesions deeper within the lung parenchyma or isolated deep within the lung cannot be visualized effectively. Thus, lesions entirely surrounded by aerated lung tissue, such as deep lung abscesses or lesions not reaching the pleural surface, may go undetected using TUS (cf ollivett) Not all lung lesions visualized by TUS correlate equally with clinical outcomes. While maximal depth and area of lung consolidation detected by ultrasound strongly predict negative outcomes, certain observed features, like comet-tail artifacts or minimal pleural effusion, may have limited or no prognostic relevance, reducing the comprehensive clinical utility of ultrasound alone in certain contexts (timsit, 2019).

A stochastic mechanistic model, developed by Picault et al. (2022) \cite{picault_modelling_2022}, was created to study BRD propagation and evaluate the impact of farming practices, including pen size, risk levels of cattle, and antimicrobial treatments (individual versus collective metaphylaxis). The motivation behind this model was the need to explore optimal BRD management practices in different farming conditions, specifically assessing the balance between disease control effectiveness and antimicrobial usage. Twelve scenarios, reflecting various fattening systems characterized by differences in pen size (small pens of 10 animals vs. large pens of 100 animals), risk levels (low, high, and high risk mitigated by preventive antimicrobial treatment), and treatment protocols (individual or collective metaphylaxis), were evaluated. Model parameters were calibrated using existing empirical data and relevant veterinary literature. Results demonstrated that BRD occurrence, severity, and mortality were predominantly influenced by risk level and pen size. Large pens and higher risk levels consistently resulted in increased severity and higher mortality rates. The model also emphasized the effectiveness of collective antimicrobial treatments during fattening periods, particularly in large pens with high-risk scenarios, by significantly reducing disease severity and mortality despite the associated increase in antimicrobial usage. Conversely, implementing measures to reduce risk at pen formation provided the best overall outcomes, effectively minimizing both antimicrobial usage and cumulative disease duration. However, this mechanistic model had not previously been calibrated directly using empirical veterinary observations, leading to uncertainties regarding its practical and explanatory utility with real-world veterinary data and highlighting the importance of future validation with empirical datasets to enhance its practical applicability.

\subsection{Originality and objective of this Work}

\subsubsection*{diagnosis and prognosis expertise}

The SEPTIME project has enabled the collection of empirical data addressing critical knowledge gaps in Bovine Respiratory Disease (BRD), providing diverse datasets captured at varying temporal frequencies. Lung ultrasound videos (LUS) were recorded and employed due to its rapid, non-invasive capability for diagnosing respiratory conditions. Throughout the project's duration, lung ultrasound videos were recorded. Although the dataset assembled within this project is extensive, the analysis presented in this chapter utilizes only a subset due to the limited number of observations available. The labelled observations from these LUS videos represent a valuable foundation for evaluating and benchmarking various modelling methods. However, the primary challenge addressed in this work pertains to the limited quantity of observations available at the time of analysis, which constrains the ability to establish robust long-term prognostic insights.

\begin{figure}
  \includegraphics[width=\linewidth]{figures/chap2/annotated_ultrasound.jpg}
  \caption{Pulmonary ultrasound}
  \label{fig:ultrasound}
\end{figure}
\newpage

Achieving accurate and consistent results requires significant operator skill, training, and a systematic examination technique based on anatomical and ultrasonographic landmarks (fig \ref{fig:ultrasound}). Variations in operator skill levels can result in differing degrees of diagnostic accuracy and reliability, making standardized training and experience crucial for correct implementation


% Achieving accurate and consistent results requires significant operator skill, training, and a systematic examination technique based on anatomical and ultrasonographic landmarks. Variations in operator skill levels can result in differing degrees of diagnostic accuracy and reliability, making standardized training and experience crucial for correct implementation

This study therefore addresses two scientifically complementary research questions, directly inspired by contextual gaps and illustrated by our methodological contributions:

\paragraph{To what extent can deep learning reliably automate short-term diagnosis using limited, and context-specific observational data from sensors, such as lung ultrasounds ?} Unlike previous methods relying on manually extracted lesion characteristics \cite{timsit_association_2019}, our objective is to evaluate the extent to which deep learning architectures can autonomously derive high-level semantic representations from lung ultrasound videos.  We could then use this diagnosis expert to perform occasional diagnosis at different observation dates, this handles the need of a lot of data and could still provide first hand description of the health status (fig \ref{fig:chap2-question1}). We hypothesize that capturing the spatio-temporal patterns present in ultrasound videos through deep learning architectures can significantly enhance diagnostic robustness, particularly in field conditions where traditional manual lesion characterization is limited by subjectivity and variability. It might be lesion or it could another artefacts that the human eye wouldn't easily detect. This "sensor-to-diagnosis" automation could support veterinarians by offering immediate and objective assessments of animal health, providing valuable insight into the clinical state without extensive manual feature extraction or prolonged observational periods. Importantly, this approach could be practically deployed as a rapid and non-invasive technique to perform regular health monitoring, allowing veterinarians to obtain explicit, short-term descriptions of animal health status.

\begin{figure}
  \includegraphics[width=\linewidth]{figures/chap2/chap1-question1.jpg}
  \caption{Can we use deep learning to automated the diagnosis at occasional date points using the limited available observations from sensors ?}
  \label{fig:chap2-question1}
\end{figure}
    
\paragraph{Can mechanistic epidemiological models be parametrized using empirical veterinary observations to provide accurate explanatory long-term predictions for BRD ?} The previously established stochastic mechanistic model for BRD propagation \cite{picault_modelling_2022} was calibrated using theoretical assumptions rather than empirical veterinary observations, limiting its practical validation. In this work, we investigate whether empirical veterinary assessments considered ground truth, collected at limited temporal intervals and sparse frequency, can effectively parametrize a mechanistic epidemiological model to reliably predict BRD dynamics. Since daily health observations can be logistically challenging and costly in practice, we aim to assess whether accurate modelling can still be achieved by fitting the model to sparse empirical observations, allowing it to explicitly reconstruct disease dynamics and clinical states even on unobserved days (fig \ref{fig:chap2-question2}). Anything can happen in between the points, we could rely on the theoretical knowledge embedded in mechanistic models to explicit and give us evidence-based insights.


\begin{figure}[h]
  \includegraphics[width=\linewidth]{figures/chap2/chap1-question2.jpg}
  \caption{Can a mechanistic model for BRD be fitted to real-wolrd observations from veterinarians to predict at a larger temporal scale a explicit all the dynamics of BRD ?}
  \label{fig:chap2-question2}
\end{figure}
\newpage    


%-----------%-----------
%	SOUS-SECTION 
%-----------%-----------
\subsection{Main contributions and perspectives}
\label{chap:contributions}
This research provided three core contributions:

\begin{enumerate}
    \item Creation of an original dataset comprising pulmonary ultrasound videos with corresponding veterinary clinical annotations, providing a valuable resource for future BRD diagnostic and epidemiological research. Data collection occurred from January to June 2023, spanning a 30-day period post-arrival of calves on nine farms, each managing multiple batches of animals. Clinical annotations classified animals as symptomatic or asymptomatic according to veterinary-defined criteria (rectal temperature above 39.7°C and presence of at least one clinical symptoms such as cough or nasal discharge). Practical constraints, including animal immobilization, shaving the scan area, and veterinary availability, limited data quantity to approximately thirty annotated ultrasound videos. [\textcolor{red}{be more factual on the quantity of observation per class}]

    \item Veterinary clinical assessments enabled the parametrization of the BRD mechanistic epidemiological model \cite{picault_modelling_2022}. The sensitivity analysis conducted by Picault in 2022 revealed three parameters as the most influential in controlling BRD dynamics, antimicrobial usage, and mortality rates. These parameters are the \textbf{ pathogen transmission rate}, which describes how rapidly an infectious agent spreads between animals; the \textbf{mean duration in the infectious state}, representing how long an infected animal can transmit the disease; and the \textbf{mean duration in the pre-infectious state}, indicating how long an animal remains exposed before becoming actively infectious. Accurate estimation of these parameters is critical because they substantially influence infection spread, disease severity, and effectiveness of control strategies, including antimicrobial use. This parametrization (fig \ref{tab:parameter_comparison}) allowed predictions of BRD dynamics over a 30-day period, achieving forecasting accuracy with a root mean squared error (RMSE) below 5\% in certain farms. However, the general applicability of the average pathogen model across all scenarios warrants caution. We hypothesize that this simplification could limit its accuracy in predicting outbreaks driven primarily by viral infections. Indeed, Picault himself stated, "our assumption that the same 'average' pathogen could be used for all scenarios is indeed questionable. BRD is intrinsically a multi-pathogen disease, and the exact prevalence of each pathogen, their possible interactions, as well as the diversity of strains, can be expected to change the dynamics of infection and disease severity \cite{Kudirkiene2021, Becker2020}. In this study we assumed an average pathogen to keep the model as simple as possible. However, in further study, model parameters reflecting microbiological characteristics (e.g., the mean duration of infectiousness and the pathogen transmission rate) could be made pathogen-specific to allow for comparisons between various pathogens."

    \begin{table}[h]
        \centering
        \renewcommand{\arraystretch}{1.2}
        \begin{tabular}{l ccc ccc c}
            \toprule
            \multirow{2}{*}{\textbf{Parameter name}} & \multicolumn{3}{c}{\textbf{Farm 1}} & \multicolumn{3}{c}{\textbf{Farm 2}} & \multirow{2}{*}{\textbf{Nominal values}} \\
            
            \cmidrule(lr){2-4} \cmidrule(lr){5-7} 
            & Median & Q1 & Q3 & Median & Q1 & Q3 & Calibrated \\
            \midrule
            Pathogen Transmission rate & 0.009 & 0.006 & 0.012 & 0.019 & 0.014 & 0.023 & 0.008 \\
            Mean duration in infectious & 150 & 118 & 193 & 123 & 100 & 156 & 120 \\
            Mean duration in pre-infectious & 87 & 68 & 115 & 76 & 58 & 100 & 72 \\
            \bottomrule
        \end{tabular}
        \caption{Inferred values of parameters vs nominal value of parameters}
        \label{tab:parameter_comparison}
    \end{table}

    \begin{figure}[H]
      \includegraphics[width=\linewidth]{figures/chap2/prognosis-chap1.jpg}
      \caption{Asymptomatic forecast: ground truth vs predictions of an average pathogen mechanistic model}
      \label{fig:prognosis-chap1}
    \end{figure}

    \item A deep learning model based on a spatio-temporal CNN-RNN architecture was trained to classify animals’ clinical health status (symptomatic or asymptomatic) using pulmonary ultrasound videos. This model achieved an accuracy of approximately 72\% (table \ref{tab:feature_extractor_performance}). Specifically, the deep learning architecture combined convolutional neural networks (CNNs), serving as spatial feature extractors, and recurrent neural networks (RNN), which captured temporal information. Throughout the experiments, only the feature-extractor component—the CNN architecture—was varied, while the RNN architecture remained fixed across all tested models. The CNN's goal was to identify and extract relevant spatial features within individual ultrasound frames, such as lesions, pleura lines, or other anatomical details indicative of BRD. Conversely, the RNN aimed to leverage temporal sequences of these extracted features to model their evolution over the video duration. Different CNN architectures were evaluated, including EfficientNetB7, InceptionResnetV2, InceptionV3, VGG16, and an ensemble model combining InceptionV3 with InceptionResnetV2. Among these architectures, InceptionV3 achieved the highest performance, with a weighted F1-score of 70\%, precision of 72\%, recall of 69\%, and an overall accuracy of 69\%. In contrast, VGG16 demonstrated significantly lower performance, yielding an F1-score of only 14\%. The performance of the final selected deep learning model was assessed on two separate farms to verify its robustness and generalizability under contrasting practical conditions.
    
    \begin{table}[h]
        \centering
        \renewcommand{\arraystretch}{1.2} % Adjust row height for better readability
        \begin{tabularx}{\linewidth}{l *{4}{>{\centering\arraybackslash}X}}
            \toprule
            Feature Extractor & \small Weighted Precision & \small Weighted Recall & \small Weighted F1-score & \small Accuracy \\
            \midrule
            
            EfficientNetB7 & 0.67 & 0.62 & 0.63 & 0.62 \\
            InceptionResnetV2 & 0.71 & 0.50 & 0.49 & 0.50 \\
            InceptionV3 & \textbf{0.72} & \textbf{0.69} & \textbf{0.70} & \textbf{0.69} \\
            VGG16 & 0.09 & 0.31 & 0.14 & 0.31 \\
            InceptionV3 + InceptionResnetV2 & 0.71 & 0.62 & 0.63 & 0.62 \\
            
            \bottomrule
        \end{tabularx}
        \caption{Diagnosis performance of different deep learning architecture}
        \label{tab:feature_extractor_performance}
    \end{table}
    % \newpage 
        
\end{enumerate}

These results demonstrate the individual feasibility of both diagnostic automation using deep learning and epidemiological prognosis via mechanistic modelling. However, the two phases—diagnosis and prognosis—were not yet integrated into a complete predictive pipeline. Results highlighted the limitations of employing an average pathogen model universally, as BRD is intrinsically a multi-pathogen disease, and pathogen-specific variations significantly influence infection dynamics and disease severity. In the next chapter, methodologies for selecting the optimal prognosis expert model from multiple alternatives using outbreak observations will be discussed. 


% Pour ta thèse comme pour chaque chapitre, tu dois répondre à plusieurs questions impéra-tivement : 

% 1.	c'est quoi le problème ? (et dans quel contexte c'est un problème ?) à quelle question scientifique tu cherches à répondre ?

% 2.	en quoi c'est dur ?

% 3.	comment ça se positionne par rapport à l'existant : d'autres travaux / littérature (y compris par rapport à tes autres travaux pour le cas de chaque chapitre)

% 4.	quelles méthodes tu as employées, pourquoi, sous quelles hypothèses ?

% 5.	quels sont les résultats principaux ? en quoi ils apportent (ou pas) des éléments de réponse à la question (ou à d'autres questions) et quelles sont les questions qui se posent ensuite ?

% 6.	dans quelle mesure tes résultats sont-ils robustes ? quels sont les points faibles ou les limitations ? comment pourrait-on les corriger / contrebalancer (et est-ce nécessaire) ?

% 7.	enfin qu'est-ce qui resterait à explorer / faire ? quelles autres questions émergent à l'issue de ton travail ?

\subsection{[In French] Résumé grand public}

%-----------------------------------
%	SECTION 
%-----------------------------------

La maladie respiratoire bovine (BRD) est une pathologie fréquente et complexe qui représente un défi majeur en élevage, entraînant des pertes économiques importantes liées aux traitements, à la diminution des performances zootechniques et à une mortalité accrue. Afin de diagnostiquer cette maladie, les vétérinaires ont parfois recours à l’échographie pulmonaire, une méthode rapide, non invasive et réalisable directement à la ferme grâce à des appareils portables. Ces appareils, utilisés habituellement pour l'échographie reproductive chez les bovins, permettent d’examiner rapidement les poumons des animaux sans exposition à la radiation, contrairement à la radiographie ou au scanner. L’échographie pulmonaire peut identifier précisément et rapidement certaines lésions pulmonaires associées à la BRD, telles que la consolidation des lobes pulmonaires, les abcès ou encore les épanchements pleuraux. Ces lésions échographiques sont d’autant plus importantes qu’elles peuvent indiquer la gravité de la maladie, prédire les rechutes et renseigner sur les performances futures des animaux atteints.

Le projet SEPTIME, dans lequel s’inscrit ce travail de thèse, a permis la collecte de nombreuses données empiriques issues d’échographies pulmonaires enregistrées directement dans des élevages bovins durant les années 2023 et 2024. À partir de ces vidéos échographiques annotées par des vétérinaires experts, nous avons exploré deux approches complémentaires afin d’améliorer le diagnostic et le pronostic de la BRD :


\begin{enumerate} 
    \item L’utilisation de modèles d’intelligence artificielle (apprentissage profond) permettant d’automatiser rapidement la détection des cas cliniques de BRD à partir d’échographies pulmonaires. Ces modèles montrent un bon potentiel avec une précision atteignant environ 72\% dans la reconnaissance automatique de la maladie.

    \item Le paramétrage et l’évaluation d’un modèle mécaniste épidémiologique, qui simule la propagation à long terme de la maladie à partir d'observations réelles issues du terrain. Ce modèle mécaniste, une fois calibré, permet de prédire efficacement l’évolution de la maladie sur plusieurs semaines et pourrait ainsi aider les éleveurs à anticiper les épidémies et à adapter leurs stratégies de gestion sanitaire.

\end{enumerate}


Ces résultats démontrent la faisabilité individuelle de l'automatisation du diagnostic et du pronostic à l'aide de l'apprentissage profond et du pronostic épidémiologique via la modélisation mécaniste. Le chapitre suivant aborde les méthodologies de sélection du modèle expert de pronostic parmi de multiples alternatives en utilisant des observations cliniques.

\section{Proceedings published in Society for Veterinary Epidemiology and Preventive Medicine, 2024}

\includepdf[pages=-]{articles/SVEPM.pdf}  % Replace with your actual filename

% \chapter*{\Huge Big Title Here}
% \addcontentsline{toc}{chapter}{Big Title Here}  % Add to TOC if needed

\chapter{Structural synergism in bovine respiratory disease modelling} % chapter title

\section{Introduction}
\subsection{Contextual background}

%-----------
%	SOUS-SOUS-SECTION 
%------------

Mathematical modelling has emerged as a cornerstone for evidence-based decision-making in animal health, increasingly recognized as indispensable by policy makers and veterinary epidemiologists  \cite{Picault2024, Ezanno2022}. Modelling frameworks offer a structured approach to understanding disease dynamics, quantifying transmission risks, forecasting outbreaks, and evaluating intervention strategies. In veterinary epidemiology, and particularly in the management of complex multi-pathogen diseases like Bovine Respiratory Disease (BRD), mathematical models serve as powerful tools that integrate knowledge across multiple disciplines, including epidemiology, veterinary science, agricultural management, and economics.

BRD is inherently complex due to its multifactorial nature, characterized by interactions between various pathogens, host susceptibility, environmental stressors, and management practices. Numerous mechanistic epidemiological models have been proposed to capture this complexity \cite{picault_modelling_2022, sorindupont_modeling_2023}. However, the mechanisms governing pathogen interactions, co-infections, and the clinical manifestations of BRD remain incompletely understood. Consequently, existing models often differ significantly in their structure, underlying assumptions, and levels of biological detail. These structural differences create uncertainties in predictions, complicating the practical task of identifying a model that accurately reflects real-world outbreak dynamics.

Recent developments emphasize the advantages of using pathogen-specific mechanistic models instead of generalized "average pathogen" approaches. Pathogens such as Orthopneumovirus bovis (BRSV), Mannheimia haemolytica (Mh), and Mycoplasmopsis bovis (Mb) exhibit distinct epidemiological characteristics, clinical progressions, and treatment responses, each requiring targeted intervention strategies. Recognizing this diversity, pathogen-specific models have been advocated for their enhanced realism and predictive accuracy in simulating BRD outbreaks. Beyond parameterization challenges, the comparative assessment of multiple valid mechanistic models necessitates addressing critical methodological dimensions. Model distinguishability, assesses whether competing models produce sufficiently distinct predictions, thereby enabling pathogen-model identification from clinical observations. Secondly, and most practically relevant, is the concept of decision impact assessment, evaluating whether the selection of a given model actually translates into measurable, economically viable, and actionable improvements on the farm. These two interconnected methodological challenges underscore the need not only for rigorously validated mechanistic models but also for frameworks capable of linking theoretical model selection directly with improved decision-making outcomes.




\subsection{Originality and objective of this work}
%-----------%-----------
%	SOUS-SECTION 
%-----------%-----------

this chapter focuses on two scientifically complementary research questions, driven by critical contextual and methodological challenges highlighted above:

\paragraph{To what extent can we reliably differentiate between multiple pathogen-specific mechanistic models of BRD, solely based on symptomatic observations ?} In this chapter, we introduce a numerical approach aimed at distinguishing among competing BRD mechanistic epidemiological models \cite{sorindupont_modeling_2023} based exclusively on observed symptomatic trajectories. Specifically, we show a general theoretical framework for pathogen-model identification, a critical issue across epidemiological contexts where accurate differentiation between pathogens is challenged by symptom overlap. The relevance of this work extends beyond Bovine Respiratory Disease (BRD), offering a broadly applicable solution for any epidemiological scenario involving multiple plausible mechanistic hypotheses or co-existing pathogens that necessitate distinct pathogen-specific management strategies. Orthopneumovirus bovis (BRSV) model captures rapid, airborne viral transmission dynamics, characterized by acute and intense infection episodes. It explicitly incorporates compartments for partial immunity, primary infection, potential reinfection (with reduced infectiousness), and rapid progression from mild to severe clinical signs, making the outbreaks swift but relatively short-lived. Mannheimia haemolytica (Mh) is modelled as an opportunistic infection primarily triggered by host immunosuppression or environmental stressors. Unlike the BRSV model, the Mh model does not account for reinfection states but includes a clear transition from asymptomatic carriage to active infection states, which can escalate into severe clinical manifestations. The infection dynamics are less explosive compared to BRSV, emphasizing progression triggered by stress-induced susceptibility. The mycoplasmosis bovis (Mb) model structurally mirrors the Mh model regarding compartments (asymptomatic carriers transitioning to symptomatic stages). However, it notably differs by emphasizing chronicity and persistence. This pathogen exhibits slower transmission, prolonged infection durations, and intermittent clinical symptom manifestation, making early detection and timely treatment more challenging and resulting in prolonged circulation within cattle populations. The probability that a treated animal recovers (for 1 dose) is set to 71\% for  Mh and 60\% for Mb. The models where calibrated using from the litterature: probabilities of recovery with antibiotic treatment (single dose) are set at 71\% for Mh and 60\% for Mb. the probability of detecting symptomatic animals (with mild clinical signs) is ... and the probability of detecting a animal displaying severe clinical signs is ...


\paragraph{Does the distinction and identification of the most likely pathogen-specific mechanistic model significantly improve practical decision-making outcomes ?} In typical cattle farming practices (conventional treatment decisions), antibiotic treatments for Bovine Respiratory Disease (BRD) are usually administered empirically based solely on observable clinical signs, without reliable pathogen identification. Under these conditions, farmers treat all symptomatic animals with antibiotics, regardless of whether the underlying infectious agent is bacterial or viral. Since antibiotics are effective only against bacteria and not viruses, this approach frequently leads to inappropriate use of antibiotics, which has two main drawbacks: if the infectious agent is viral, antibiotic treatments are unnecessary, ineffective, and economically wasteful. Such misuse increases antimicrobial resistance risks without any animal health benefit. If the infectious agent is bacterial, antibiotics are warranted and beneficial; failure to correctly administer them could lead to severe economic losses and compromised animal welfare.

To address this issue, we explicitly incorporate an economic dimension into our analysis by integrating pathogen-specific model predictions into a bio-economic framework. This enables us to quantify practical benefits, such as reductions in antibiotic usage, improved animal health outcomes, and enhanced profitability directly resulting from pathogen-informed decisions. By explicitly evaluating these consequences, our work connects theoretical epidemiological modelling with tangible, farm-level decision-making, reinforcing the practical relevance of modelling for effective livestock management.


% cite to articles discussing about cholera model distinguishability. [explain the difference between model selection and model distinguishability.]


\subsection{Main contributions and perspectives}

\subsubsection*{prognostic expert identification via model distinguishability}

Our primary methodological contribution involves numerically distinguishing three mechanistic BRD models tailored for Orthopneumovirus bovis (BRSV), Mannheimia haemolytica (Mh), and Mycoplasmopsis bovis (Mb) \cite{sorindupont_modeling_2023}. First, We constructed synthetic outbreak scenarios representative of French beef cattle farms. Three discrete-time, stochastic agent-based, pathogen-specific model  were utilized to generate outbreak trajectories over 277 days, capturing symptomatic dynamics every 12 hours for a batch of 12 calves. Variations across scenarios reflected realistic risk compositions, yielding a dataset comprising 13,650 individual simulations. Secondly, Three pathogen-specific stochastic, compartmental models \cite{sorindupont_modeling_2023} were compared, each encapsulating different transmission dynamics. Given the stochastic complexity of models and intractable likelihood functions, we employed Approximate Bayesian Computation (ABC) combined with multinomial logistic regression to identify the most likely pathogen-specific model. The ABC approach quantitatively assessed model distinguishability using summary statistics (detected symptomatic trajectories), thus allowing informed selection of the pathogen responsible for observed outbreaks.Given the stochastic complexity of models and intractable likelihood functions, we employed Approximate Bayesian Computation (ABC) combined with multinomial logistic regression to identify the most likely pathogen-specific model. The ABC approach quantitatively assessed model distinguishability using summary statistics (detected symptomatic trajectories), thus allowing informed selection of the pathogen responsible for observed outbreaks. Using synthetic symptomatic data generated under realistic farm conditions, we successfully identify the most likely pathogen-specific model with an average accuracy of approximately 93\% (fig \ref{fig:chap3-confusion-matrix}). Key performance metrics (true positive rates: BRSV=96\%, Mh=90\%, Mb=87\%) clearly indicate the feasibility and reliability of pathogen identification based on early symptomatic trajectories.

\begin{figure}[H]
  \includegraphics[width=\linewidth]{figures/chap3/chap3-confusionmatrix.png}
  \caption{Confusion matrix. Classification performance for BRSV, Mh and Mb. The diagonal represents correctly classified instances, while off-diagonal values indicate misclassification between classes.}
  \label{fig:chap3-confusion-matrix}
\end{figure}


\subsubsection*{Decision-intelligence: measuring and improving the impact on decision-making}

The second major contribution involves integrating pathogen-specific epidemiological models into a detailed economic evaluation framework. This integration enables us to quantitatively assess the economic impact of adopting pathogen-informed treatment decisions versus conventional empirical treatment strategies. Our economic model accounts explicitly for weight gain, carcass quality, feed, antibiotic treatment costs, and veterinarian interventions, calculating expected net profits and antibiotic usage for simulated cattle batches. We showed that pathogen-informed decisions substantially reduce antimicrobial use by approximately 44\% (fig \ref{fig:chap3-expectation}) in these conditions, directly addressing critical issues like antimicrobial resistance and public health safety. Simultaneously, these pathogen-informed strategies lead to a modest yet consistent increase (around 1\%) in net profitability, highlighting that economic viability can be maintained or improved even when reducing antibiotic treatments.

By leveraging the high accuracy (~93\%) of pathogen-model distinguishability, we maximize the likelihood that the correct treatment decision, either antibiotic administration for bacterial infections or withholding antibiotics for viral infections—is consistently taken.This accuracy directly improves decision-making outcomes by: Increasing the frequency of correct recommendations, thus ensuring targeted treatments that align with actual disease aetiology. And reducing incorrect or harmful decisions (false positives and false negatives), significantly decreasing inappropriate antibiotic use and associated economic and public health costs.

\begin{figure}[H]
  \includegraphics[width=\linewidth]{figures/chap3/expectations.jpg}
  \caption{Pathogen informed treatment decisions versus conventional treatment decisions.}
  \label{fig:chap3-expectation}
\end{figure}

\subsubsection*{Perspectives}

This work bridges theoretical epidemiological modelling with practical decision-making, emphasizing the importance of model distinguishability not only theoretically but as a practical tool for veterinary epidemiology and livestock management. Our approach is broadly applicable to scenarios where pathogen differentiation based solely on observations remains challenging yet essential. In the next chapter, we further explore integrating a real-world observation (sensor-based) for immediate diagnostic insights (deep learning-driven) with longer-term, prognosis-oriented mechanistic epidemiological models, combining these two forms of expertise to generate actionable disease management recommendations.

\subsection{[In French] Résumé grand public}
La gestion efficace des maladies respiratoires bovines (BRD) constitue un enjeu crucial pour la santé animale et pour l’économie des élevages bovins. Ces maladies sont complexes car elles impliquent souvent plusieurs pathogènes différents, notamment des virus comme Orthopneumovirus bovis (BRSV), et des bactéries telles que Mannheimia haemolytica (Mh) ou Mycoplasmopsis bovis (Mb). Chaque pathogène possède ses propres particularités en matière de transmission, de symptômes et de réponse au traitement, rendant leur identification précise essentielle pour une gestion optimale.

Dans ce chapitre, nous avons proposé une approche basée sur la modélisation mécaniste afin d’identifier quel pathogène est à l’origine d’une épidémie, simplement à partir des symptômes observés chez les animaux. Notre méthode utilise des modèles mécanistes spécifiques à chaque pathogène, pour déterminer avec fiabilité l’agent responsable d’une épidémie sur une exploitation. Grâce à des simulations numériques réalistes, nous avons montré que cette identification est possible avec une précision élevée (environ 93\% en moyenne).

Cette identification précise présente des avantages pratiques majeurs pour les éleveurs. Aujourd'hui, faute d’identification fiable, les traitements antibiotiques sont souvent administrés à tous les animaux symptomatiques sans distinction, même si certains souffrent d’infections virales pour lesquelles les antibiotiques sont inefficaces. Cette pratique entraîne une utilisation excessive et inutile des antibiotiques, augmentant le risque de résistance bactérienne, tout en générant des coûts économiques inutiles. Notre étude démontre que l’intégration de ces modèles spécifiques dans une démarche décisionnelle permet une réduction d'environ 44\% de la consommation d’antibiotiques tout en maintenant, voire en augmentant légèrement (~1\%), la rentabilité des exploitations.

Ainsi, ce travail met en évidence l’intérêt concret des modèles mécanistes pour améliorer les décisions pratiques en élevage, réduisant à la fois les coûts économiques et les risques sanitaires liés à une mauvaise utilisation des traitements. Cette approche pourrait s’appliquer à d’autres contextes épidémiologiques où identifier précisément le pathogène à partir de simples observations cliniques reste un défi majeur.


%-----------------------------------
%	SECTION 
%-----------------------------------
\section{Peer-reviewed preprint in bioarxiv, 2025}


    % \input{chapters/chap1-article} # pas besoin de mettre un file appart sauf si j'ai des choses spécifiques à rajouter pour cette partie
    \includepdf[pages=-]{articles/article3.pdf}  % Replace with your actual filename
    \includepdf[pages=-]{articles/supp-mat3.pdf}

% \chapter*{\Huge Big Title Here}
% \addcontentsline{toc}{chapter}{Big Title Here}  % Add to TOC if needed

% \chapter{Components integration - a deep mechanistic approach} % propositions de titre
\chapter{A deep mechanistic model: Grounded mechanistic model for adaptive knowledge}
%----------------------------------------------------------------------------------------
%	SECTION 
%----------------------------------------------------------------------------------------
\section{Introduction}
%-----------%-----------
%	SOUS-SECTION 
%-----------%-----------

 % pour justifier la précision obtenue: Moreover, for BRD and similar conditions there is no universally accepted “gold standard” diagnostic: clinical thresholds vary by practitioner and farm context, and standard laboratory tests (e.g., bacterial culture) can take days to return, further compounding uncertainty in disease classification

\subsection{Contextual background}
Modern sensor modalities (e.g., lung ultrasound, video and audio surveillance, etc) offer high‑frequency, objective measurements that overcome many limitations of human observation (inter‑observer variability, delayed reporting). However, these measurements represent only the observable manifestations of underlying pathophysiological processes and thus provide an incomplete, noisy “snapshot” of disease state—what Yoan Bourhis (2017) aptly describes as the “tip of the iceberg” .

Aleatoric uncertainty refers to the inherent noise and variability present in the data itself. In the context of sensor‑based diagnostics—such as lung ultrasound videos of cattle, aleatoric uncertainty arises from factors beyond the model’s control: image acquisition artifacts, animal movement, inconsistent probe positioning, and intrinsic biological variability in disease presentation. Because aleatoric uncertainty reflects randomness in the observation process, it cannot be reduced simply by collecting more data; instead, it must be explicitly modelled so that downstream predictions correctly reflect the limits of information contained in each measurement. Epistemic uncertainty, by contrast, captures the model’s lack of knowledge about the correct mapping from inputs to outputs. This form of uncertainty is highest when the model encounters examples that are far from the distribution of its training data—rare clinical presentations, novel farm environments, or unforeseen sensor configurations. Unlike aleatoric uncertainty, epistemic uncertainty can be reduced through the acquisition of additional, representative training examples. Crucially, a model can output a high softmax score (traditional logits interpreted as probabilities) even when its epistemic uncertainty is large, leading to overconfident and potentially erroneous predictions \cite{gal_dropout_2016}.

Epidemiological forecasting relies on mechanistic models, which codify known biological relationships (transmission rates, immune response kinetics, within‑host pathogen dynamics) into mathematically tractable systems of equations. While these models capture long‑term disease trajectories and can simulate the effects of interventions at the population level, they are poorly informed by sensor data when observations are unstructured, sparse and noisy.


% Standard deep‑learning pipelines commonly equate the softmax output of the final layer with predictive confidence, yet this conflation masks both aleatoric and epistemic uncertainties and gives no indication of when a prediction should be trusted. In high‑stakes veterinary diagnostics, such overconfidence can result in inappropriate treatments, delayed intervention, and significant economic and welfare costs.


%-----------%-----------
%	SOUS-SECTION 
%-----------%-----------
\subsection{Article Originality and objectives}

\paragraph{To what extent can automated short-term diagnostics derived from limited sensor observations effectively inform and specify a mechanistic epidemiological model for long-term disease prognosis?} In this work, we propose and explore a novel hybrid methodology explicitly aimed at integrating deep learning, sensor-based diagnostics with a mechanistic epidemiological model (Fig \ref{fig:chap4-question1}). Specifically, we develop an approach that leverages short-term, sensor-derived diagnostics obtained from limited Lung Ultrasound (LUS) video data to inform parameter calibration in epidemiological models. We employ a Bayesian Deep Mechanistic approach to bridge sensor-based diagnostics and mechanistic forecasting. This integration aims to enhance the interpretability, and practical applicability of epidemiological prognosis, enabling accurate long-term disease management strategies from inherently limited and uncertain short-term observations. 


\begin{figure}[H]
  \includegraphics[width=\linewidth]{figures/chap4/chap4-question1.jpg}
  \caption{Sketch workflow of a ...}
  \label{fig:chap4-question1}
\end{figure}

\paragraph{How can intrinsic uncertainties inherent in sensor-derived diagnostic data, especially noisy observations such as Lung Ultrasound (LUS) videos, be explicitly quantified and incorporated into mechanistic models to ensure a more robust and trustworthy prognostic prediction ?}

Given the intrinsic uncertainty and noisiness of real-world LUS videos, exacerbated by animal movement and image acquisition limitations, this work explicitly addresses the quantification and integration of these uncertainties into the diagnostic and prognostic pipeline. We variational methods in Bayesian deep learning model to quantify the uncertainty associated with sensor-based diagnostic predictions. Bayesian probability theory offers us mathematically grounded tools to reason about model uncertainty. These quantified uncertainties are subsequently managed through two complementary approaches: either by filtering out high-uncertainty, unreliable observations (Out-of-distribution detection) to ensure safe prognosis and prioritize them for detailed veterinary reassessment, or by directly propagating uncertainties into mechanistic model calibration through uncertainty-weighted parameter inference. This dual strategy enhances the robustness, reliability, and practical applicability of long-term prognostic predictions.


%-----------%-----------
%	SOUS-SECTION 
%-----------%-----------
\subsection{Main contributions}
This chapter introduces the Bayesian Deep Mechanistic (BDM) model, a novel integrative approach that leverages both data-driven diagnostics derived from sensor technologies and knowledge-driven epidemiological modeling. This integration addresses critical limitations inherent in existing diagnostic and prognostic methodologies, specifically for Bovine Respiratory Disease (BRD), by explicitly quantifying and managing uncertainty from limited and noisy sensor observations. Three main contributions emerge from the approach presented in this chapter:


\begin{enumerate}
    \item \textbf{Coupling punctual diagnostics with mechanistic forecasting (fig \ref{fig:chap4-method1}).}   We designed and trained a hybrid deep learning pipeline, a spatio-temporal CNN‑RNN—that ingests raw lung ultrasound (LUS) video clips and outputs a binary infectious status (infectious vs. non‑infectious) for individual bovine. Our dataset comprised 265 LUS videos collected from 163 animals across nine French farms, capturing real‑world variability in lesion appearance, probe positioning, and animal movement.  Crucially, these “punctual” diagnostic predictions served not only as standalone classifiers but also as the empirical anchor for calibrating a mechanistic epidemiological model via Approximate Bayesian Computation (ABC). In practice, we treat each deep learning prediction as a point estimate of batch‑level infection prevalence at discrete observation times; ABC then infers the three key parameters—initial prevalence, average infectious duration, and transmission rate—that best reconcile the mechanistic model’s simulated infection trajectories with these sensor‑derived snapshots. This coupling bridges short‑term, deep learning diagnostics and long‑term, mechanistic forecasts. 

    Although fully automated and practical for on‑farm use, our diagnostic model achieved a relative root‑mean‑square error (RRMSE) of 39\% against veterinarian‑confirmed labels, reflecting the inherent noise and variability of LUS data. When these predictions alone were used to parametrize the mechanistic model, the resulting 30‑day forecast exhibited an RRMSE of 38.4\%, substantially higher than the 23\% error attained when using veterinarian ground‑truth diagnostics. This gap underscores both the promise of automated sensing for scalable disease surveillance and the imperative of explicitly quantifying and managing uncertainty in order to approach the accuracy of expert‑driven prognostics.

    \begin{figure}[H]
      \includegraphics[width=\linewidth]{figures/chap4/method 1.jpg}
      \caption{coupling punctual diagnostics with mechanistic forecasting}
      \label{fig:chap4-method1}
    \end{figure}

    \item \textbf{Sensor observation cleansing through Uncertainty-Based Filtering (fig \ref{fig:chap4-method2}).} Lung ultrasound (LUS) videos are full with sources of noise, animal motion, suboptimal probe positioning, uninterpretable image artifacts—that can lead a deterministic classifier to make confidently wrong predictions (In chapter 1 \ref{fig:chap2-question1},  we achieved only a 72\% accuracy, for a binary classifier, that is low). In order to control this risk in observations used, we had to explicitly capture the epistemic uncertainty (i.e., the model’s lack of knowledge). For that we converted our CNN‑RNN DL model into a Bayesian deep learning model using Monte Carlo Dropout (MCD). During inference, MCD performs dozens of stochastic forward passes with dropout active, producing a distribution of softmax probability vectors rather than a single point estimate. We then quantify uncertainty by computing the Shannon entropy of each video’s averaged class probability distribution—a well‑established acquisition function in active learning that measures the spread (disorder) of predictive beliefs. High entropy indicates low confidence and signals inputs for which the model’s knowledge is insufficient.

    By ranking predictions by entropy, we implemented an uncertainty‑based filter: low‑entropy (high‑confidence) cases are accepted as automatic diagnoses, while high‑entropy (low‑confidence) cases are flagged for veterinarian review. This selective acceptance dramatically improved diagnostic accuracy, lowering the relative root‑mean‑square error (RRMSE) from 39\% (deterministic predictions) to 32\% against veterinarian ground truth. Crucially, when only filtered (high‑confidence) diagnoses were used to calibrate our epidemiological model, the 30‑day forecast error fell to 27.2\% RRMSE—much closer to the 23\% error achieved using full veterinarian diagnoses
    
        \begin{figure}[H]
          \includegraphics[width=\linewidth]{figures/chap4/method 2.jpg}
          \caption{Sensor observation cleansing through Uncertainty-Based Filtering}
          \label{fig:chap4-method2}
        \end{figure}

    \item \textbf{Prognosis robustness through Uncertainty Propagation (fig \ref{fig:chap4-method3}).} Rather than discarding low‑confidence diagnoses, we leveraged each prediction’s quantified uncertainty to inform the parametrization of a stochastic mechanistic epidemiological model. After obtaining a posterior distribution (uncertainties) over batch‑level infectious counts via Monte Carlo Dropout, we extracted both the mean (as the point estimate of infected prevalence) and its variance (as a measure of diagnostic confidence). During Approximate Bayesian Computation (ABC) parameter inference, we replaced the standard Euclidean distance with a weighted version in which each observation’s contribution was inversely proportional to its uncertainty (i.e., higher variance → lower weight). This uncertainty‑weighted inference ensures that reliable, low‑variance diagnostics exert greater influence on parameter estimation (pathogen transmission rate, mean duration in the infectious state, mean duration in the pre-infectious state), while noisy, high‑variance observations contribute less, effectively down-weighting potentially misleading observations rather than excluding it outright. The result is a more robust posterior over epidemiological parameters and, consequently, more accurate long‑term forecasts: the uncertainty‑propagated model achieved a 30‑day forecast RRMSE of 27.2\% nearly matching the 23\% error obtained when calibrating solely on veterinarian‑confirmed diagnoses. This method also reduces diagnostic error from the deterministic model’s 39\% RRMSE to 32\% RRMSE, matching the improvement seen in our uncertainty‑filtering approach

    These results demonstrate that explicitly propagating diagnostic uncertainty not only improves automated classification but also closes most of the remaining gap between sensor‑based forecasts and expert‑driven prognostics, unlocking robust long‑term predictions from inherently noisy, limited observations. By embedding predictive uncertainty directly into the model calibration process, our approach preserves information from all sensor‑based observations maximizing data utility, while mitigating the impact of noise.

    \begin{figure}[H]
      \includegraphics[width=\linewidth]{figures/chap4/method 3.jpg}
      \caption{Prognosis robustness through Uncertainty Propagation}
      \label{fig:chap4-method3}
    \end{figure}
    
\end{enumerate}

 
\subsection{[In French] Résumé grand public} 

La santé animale et la prévention des maladies infectieuses reposent de plus en plus sur des technologies de diagnostic automatisées, comme l’échographie pulmonaire chez les bovins. Toutefois, ces données issues de capteurs sont souvent bruyantes, incomplètes et sujettes à des erreurs d’interprétation. Notre étude propose une nouvelle approche hybride qui combine l’intelligence artificielle (IA) et les modèles mathématiques traditionnels (appelés «mécanistiques») pour améliorer la détection précoce et la prévision à long terme de la maladie respiratoire bovine.

Nous avons d’abord développé un modèle d’apprentissage profond capable d’analyser automatiquement de courtes vidéos d’échographie pulmonaire et de prédire si un animal est infectieux ou non. Pour tenir compte de l’incertitude inhérente à ces diagnostics automatisés (problèmes de qualité d’image, mouvements de l’animal…), nous utilisons une technique bayésienne qui mesure la confiance de chaque prédiction. Les cas où le modèle est peu sûr sont soit signalés pour un examen vétérinaire, soit pondérés de façon moindre dans les étapes suivantes.

Ensuite, ces diagnostics ponctuels, enrichis de leur degré de confiance, servent à calibrer un modèle épidémiologique mathématique qui simule la propagation de la maladie dans un troupeau sur plusieurs semaines. Cette intégration («fusion») permet de produire des prévisions de l’évolution de l’infection plus fiables que celles basées uniquement sur l’IA ou uniquement sur les modèles traditionnels. Concrètement, notre méthode réduit l’erreur de prédiction à 27\% sur 30 jours — un résultat proche de la précision obtenue lorsqu’on utilise exclusivement des diagnostics vétérinaires, mais avec un processus entièrement automatisé et évolutif pour une utilisation directe en élevage.

En résumé, cette étude démontre qu’il est possible d’exploiter efficacement des données de capteurs imparfaites grâce à une modélisation hybride : l’intelligence artificielle apporte une analyse rapide et à grande échelle, tandis que le modèle mathématique garantit une prévision robuste à long terme. Cette approche ouvre la voie à une surveillance de la santé animale plus précise, proactive et accessible, contribuant à améliorer le bien‑être des animaux et à réduire les pertes économiques liées aux maladies infectieuses.


\section{Article published in Preventive Veterinary Medicine (Elsevier), 2024}


    % \input{chapters/chap1-article} # pas besoin de mettre un file appart sauf si j'ai des choses spécifiques à rajouter pour cette partie
    \includepdf[pages=-]{articles/PVM.pdf}  % Replace with your actual filename



%\part*{Discussion et conclusion générale}
\fancyhead{} % clear all header fields
\fancyhead[OL]{\textsc{General discussion}}
% \chapter*{\Huge Big Title Here}
% \addcontentsline{toc}{chapter}{Big Title Here}  % Add to TOC if needed

\chapter{General discussion} % Main chapter title

% 


%----------------------------------------------------------------------------------------
%	SECTION 
%----------------------------------------------------------------------------------------
\section{Main contributions}
%-----------%-----------
%	SOUS-SECTION 
%-----------%-----------


\paragraph{Integration of sensor-based and mechanistic models} Central methodological contribution of this thesis is the integration of deep learning diagnostic tools with mechanistic epidemiological models to improve disease management, specifically for bovine respiratory disease (BRD). This hybrid AI-epidemiological approach explicitly tackles the challenge of reconciling short-term, sensor-driven diagnostic accuracy with long-term, model-based epidemiological forecasting. We drew inspiration from the Mixture of Experts paradigm by loosely coupling two specialized components – one for diagnosis and one for prognosis – and assigning each to its domain of expertise. This design allowed deep learning to focus on immediate, data-driven classification of disease status, while the mechanistic model provided reliable long-term projections based on epidemiological principles. The approach demonstrated two key strengths. First, the deep learning module excelled at rapid, accurate diagnosis from noisy sensor data (lung ultrasound imagery), even with limited training examples. For instance, the model achieved about 72\% classification accuracy using fewer than 30 training ultrasound video samples, highlighting its practicality in data-scarce veterinary settings. Notably, when we enhanced this diagnostic model with a Bayesian architecture to quantify prediction confidence, the effective accuracy rose to roughly 88\% by filtering out highly uncertain cases – an improvement that underscores the value of uncertainty-aware AI in automating complex clinical assessments \cite{gal2016dropout}
. Second, the mechanistic epidemiological module excelled at prognosis: it extended the short-term diagnostic insights into accurate long-term forecasts of disease dynamics. After parametrising the model with empirical BRD outbreak observations, it produced robust epidemic trajectory predictions with a forecast error (RMSE) below 10\%. This level of accuracy demonstrates the mechanistic model’s value for strategic decision-making over longer time horizons, complementing the immediacy of the deep learning diagnoses. By systematically separating diagnostic and prognostic tasks, our hybrid framework capitalizes on the respective strengths of data-driven AI and mechanistic modelling, an approach aligned with recent calls to combine these paradigms for epidemic prediction Overall, this integration of sensor-based AI and mechanistic modelling is an innovative step toward decision-support tools that operate across temporal scales of disease management \cite{chen2024hybrid}.



\paragraph{Explicit handling of uncertainty}  A second major contribution of this work is the explicit quantification and propagation of uncertainty within the hybrid diagnostic-prognostic pipeline. We addressed the question of how to manage the significant uncertainty inherent in BRD sensor data (noisy ultrasound observations of pathological lung changes) through a Bayesian Deep Mechanistic approach. By employing Monte Carlo Dropout for variational inference, the deep learning model generated probabilistic predictions along with measures of confidence \cite{gal_dropout_2016}. We then filtered out the most uncertain diagnoses – essentially having the system “know what it doesn’t know” – which yielded a notable reduction in diagnostic error rates (from an initial relative RMSE of 39\% down to 32\%). These uncertainty-filtered predictions were subsequently used in the mechanistic model’s calibration via a weighted approximate Bayesian computation scheme, so that less certain inputs were given diminished influence. Propagating the diagnostic uncertainty in this manner improved the reliability of long-term forecasts: the hybrid model’s projection error dropped to a relative RMSE of 27.2\%, approaching the 23\% error of a baseline model informed by expert veterinary diagnoses. In other words, our framework better matched expert-driven forecasts by embracing uncertainty rather than ignoring it. This result is significant because it shows that a principled treatment of uncertainty can enhance both immediate and future predictions in disease monitoring systems. The outcome supports recent observations that accounting for prediction confidence improves epidemiological forecasts. Importantly, this contribution tackles real-world complexity: livestock health data are often limited or imperfect, and by embedding uncertainty into the decision pipeline, our approach maintained robust performance even when data quality was compromised. This methodological advance – integrating Bayesian deep learning with epidemiological simulation – directly confronts the need for reliability in AI-driven agriculture, ensuring the system remains cautious and reliable under data ambiguity.

% [remark on Explicit integration and propagation of uncertainty: à développer, notamment en expliquant plus en détail "pourquoi ça marche" et l'articulation entre prédiction ML et prédiction méca". ]

\paragraph{Pathogen identification via model-based distinguishability} Another key innovation of this thesis is a method to identify the likely causative pathogen of BRD from clinical observations, using mechanistic models and simulation-based inference. We introduced a pathogen-specific modeling framework in which distinct mechanistic epidemic models were formulated for different BRD etiological agents – specifically, Orthopneumovirus (BRSV), Mannheimia haemolytica, and Mycoplasmopsis bovis. Using approximate Bayesian computation combined with multinomial logistic regression as a model selection tool, we were able to discriminate among these candidate pathogen models based solely on early symptomatic trajectories. This approach achieved high identification accuracy (93\% on average, with individual pathogen identification rates of about 87–96\%). Such performance demonstrates that even when different pathogens cause clinically overlapping respiratory syndromes, their dynamical “fingerprints” in outbreak data can be teased apart with the right analytical approach. This contribution is particularly noteworthy because coinfections and similar clinical presentations are common in BRD \cite{Gaudino2022}, making targeted interventions difficult in practice. By focusing on model distinguishability – ensuring each pathogen’s model produces sufficiently unique patterns – our method provides a data-driven way to infer the likely infection cause. This is an important step toward pathogen-specific decision support. Unlike traditional diagnostic tests which might require lab work or specific assays for each pathogen, our approach uses routine observational data (e.g. clinical scores over time) to probabilistically identify the pathogen. This opens the door for earlier and more tailored treatments. In summary, we demonstrated a novel use of simulation-based inference for epidemiological model selection in an animal health context, aligning with emerging applications of ABC in infectious disease modelling \cite{beaumont2019abc}.

\paragraph{Coupling with bio-economic modelling and decision support} We extended our framework beyond biological predictions by integrating economic analysis, thereby linking epidemiological outcomes to tangible farm management metrics. Specifically, we coupled the outputs of our mechanistic BRD models (such as predicted number of cases under different interventions) with a farm profitability model that accounts for treatment costs, animal performance, and other economic factors. Through this integration, we evaluated the real-world impact of using pathogen-informed strategies versus conventional blanket treatments. The results indicated that tailoring interventions to the identified pathogen could substantially reduce antimicrobial usage – by approximately 44\% in our simulations – without sacrificing economic performance. In fact, the optimized, information-driven strategy slightly increased net profit (~1\% higher) compared to traditional empirical treatment regimens. These findings carry practical significance. They suggest that better diagnostic-prognostic information can enable win-win scenarios in livestock health: improving animal welfare and public health (through judicious antibiotic use) while maintaining or even enhancing farm profitability. This bio-economic coupling illustrates the concrete benefits of our hybrid methodology. It moves the contribution from a purely methodological realm into one that resonates with industry and societal goals, like combating antimicrobial resistance (AMR) in agriculture \cite{lhermie2019antibiotic}. By quantitatively showing that informed decisions can reduce antibiotic use with minimal economic penalty, our work provides evidence in favour of precision medicine approaches in veterinary practice. Moreover, incorporating economic considerations forces the model to focus on outcomes that matter to farmers and stakeholders, enhancing the relevance of our research for real-world adoption. This interdisciplinary integration of epidemiology and economics is still uncommon; thus, our thesis contributes a template for how to merge disease modelling with cost-benefit analysis to guide actionable recommendations.

\paragraph{Structured modularity and methodological scalability} Finally, an essential contribution of this thesis is the modular design of the hybrid modelling framework, which emphasizes clear separation between components and hence greater interpretability and flexibility. We deliberately maintained independent modules for (i) sensor-based diagnosis via deep learning, (ii) disease progression and prognosis via mechanistic models, and (iii) outcome evaluation via economic modelling. This weak modularity (weakly coupled) means that each module can be developed, fine-tuned, and validated by domain experts relatively independently – for example, veterinarians and epidemiologists can focus on improving the mechanistic model or its parameters, while computer scientists can refine the deep learning model, without constantly retraining a monolithic system. This is in contrast to end-to-end integrated approaches like EAAMs, which entangle data-driven and mechanistic components into a single architecture. The modular design also eases adaptation to new contexts or updates. For instance, if a new diagnostic sensor becomes available or a new pathogen emerges, one can update or swap out the relevant module (diagnostic or mechanistic) without overhauling the entire system. This feature is especially attractive in agriculture settings where conditions vary widely: the framework could be reconfigured for a different species or management system by exchanging modules while preserving the overall architecture. This emphasis on modular, plug-and-play components aligns with software engineering best practices and is conducive to multi-disciplinary collaboration. Different teams (data scientists, veterinarians, economists) can work in parallel on their piece of the puzzle, which is crucial in an interdisciplinary project. In summary, the thesis not only delivered specific models and results, but also a methodological template for hybrid modelling that is interpretable, extensible, and generalizable. This approach could help bridge the gap between experimental AI models and practical decision-support tools in agriculture, where stakeholder trust and adaptability are paramount \cite{wolfert2017big}.

% [remark: explain precisely how our approach supports interpretability],
% [remark: give examples of adaptation to diverse contexts. Answer: conduite en bande porcine (réf à vianney sicard et expliquer rapidement ses travaux de modélisation multi-agent multi-echelle en epidemiologie: abstract de sa thèse ?)]

% \subsection{Practical contributions}

%-----------
%	SOUS-SOUS-SECTION 
%------------
% \paragraph{Data collection}. The collection of data... 




\section{Limitations}

Despite the above contributions, several limitations of our current work must be acknowledged. These limitations point to areas where further research and development are needed, and they temper the interpretation of our results. We discuss the main bottlenecks in turn, focusing on data inputs, model coupling, validation, and practical deployment challenges. It should also be noted that some of these limitations arise inherently from the interdisciplinary nature of the project – spanning animal health, machine learning, and farm management – which requires balancing competing considerations.

\paragraph{Data input limitations (sensor and observations} A fundamental limitation lies in the reliance on thoracic ultrasound (TUS) as the primary sensor input for the diagnostic module. While lung ultrasound imaging was chosen for its practical relevance (it provides a direct non-invasive view of lung lesions and consolidation in BRD cases), it only captures one aspect of the animal’s health state. Pulmonary ultrasound, as valuable as it is, offers an incomplete picture of respiratory disease. For example, severe lung lesions can sometimes be missed if they do not contact the pleura or if they occur in lung regions not accessible to ultrasound scanning. Studies have shown that traditional clinical examinations like auscultation often fail to detect such lesions altogether, and TUS is much more sensitive in that regard \cite{buczinski_comparison_2014}. In feedlot cattle, 
Timsit \cite{timsit_association_2019} demonstrated that the maximal depth and area of lung consolidation visible on ultrasound at the time of diagnosis are significantly associated with increased risk of BRD relapse and with reduced weight gain. This evidence underpins our use of ultrasound as a prognostic indicator. Likewise, on-farm studies \cite{ollivett_-farm_2016} have advocated TUS as a useful tool to identify poor prognostic signs such as extensive lung lobe consolidation or abscessation in calves, which can guide culling or intensified treatment decisions. However, focusing on ultrasound alone means our diagnostic system could overlook clinical signals of BRD that manifest in other modalities (e.g. fever, coughing, nasal discharge, or behavioral changes). For instance, a Scottish study in sheep found that relying on auscultation alone missed many cases of pneumonia that ultrasound or necropsy would catch, indicating that each modality has blind spots. In our case, the limitation is that using only ultrasound-based features might lead to false negatives (disease not detected if lesions are not visible on the pleural surface) or false positives (lesions due to past infection or other causes). This in turn would affect the accuracy of both the diagnosis and the downstream prognostic recommendations. In a real veterinary setting, a clinician examines the animal holistically – looking at physical demeanour, nasal/ocular secretions, listening for coughs or abnormal lung sounds, measuring temperature, etc. Our current approach does not yet incorporate these additional data streams. Therefore, a more multimodal sensing strategy is warranted. Combining multiple sensors (visual, acoustic, thermal, etc.) could give a more complete view of the disease state and mitigate the reliance on any single observation type. The need for multimodal data integration is underscored by evidence that certain BRD cases present predominantly with behavioural changes (e.g. feed intake reduction) or audible symptoms (frequent coughing) that might precede ultrasound-detectable lesions \cite{carpentier2018automatic}. In summary, the limitation is not the ultrasound modality per se – which is in fact quite informative – but the exclusivity of its use. Expanding the observational input to include, for example, automatic cough monitors, could improve detection sensitivity and specificity. Our current dataset was also relatively limited in size and scope (few farms and conditions), which might limit the generalizability of the trained diagnostic model; larger and more diverse datasets are needed to ensure the model’s robustness across different herd management conditions.

\paragraph{Proxy-based coupling and uncertainty modelling} Another methodological limitation concerns how the deep learning outputs are integrated into the mechanistic model – what we termed a proxy-based hybrid approach. We used the deep learning model to generate a proxy indicator (the probability of infection in the group) which then feeds into the mechanistic simulation. While intuitive, this coupling can be fragile. It assumes that the learned proxy is a reliable summary of the complex infection state, and any error or bias in the proxy will propagate to the prognosis. We partially addressed this by incorporating the model’s uncertainty (via variational Bayesian methods) into the coupling: uncertain predictions were down-weighted during mechanistic calibration. However, the uncertainty quantification method itself has limitations. We relied on Monte Carlo dropout to approximate Bayesian uncertainty in the deep network \cite{gal_dropout_2016}. This method provides an estimate of model uncertainty and has the advantage of easy implementation, but it optimizes for average-case performance and does not guarantee calibrated uncertainty intervals \cite{matiz2020conformal}. In practice, we observed that our Bayesian neural network sometimes remained over-confident or under-confident in certain scenarios. For example, some lung ultrasound videos classified as positive (diseased) with high confidence turned out to be false alarms, partly because the ground-truth labeling by veterinarians is subjective and can be ambiguous (there is no perfect gold standard test for subclinical BRD). The Bayesian neural network’s predictive intervals did not always capture these ambiguities – a limitation because it means that simply having a high model confidence isn’t a foolproof indicator of correctness \cite{kendall2017uncertainties}. Moreover, the Monte Carlo dropout approach can become computationally expensive and may not scale well to more complex deep learning architectures. Recent advances like Transformers \cite{vaswani2017attention} have shown superior performance in many pattern recognition tasks, including medical imaging, but applying Monte Carlo sampling to such large models would be costly and may still yield poorly calibrated uncertainties. In summary, while our incorporation of uncertainty is a strength of the thesis, it is not the final word on the matter. The limitation is that our current uncertainty modelling may not fully guarantee that the “right” decisions (e.g., whether to trust a particular model prediction) are always made. In future iterations, alternative uncertainty quantification techniques (such as conformal prediction to generate guaranteed coverage prediction sets, or Bayesian neural networks with better priors) should be explored to overcome this limitation. Additionally, our hybrid model currently treats the deep learning output as a static proxy; a tighter integration (for instance, a joint inference over parameters of both models) could potentially improve coherence between diagnosis and prognosis, though this comes at the cost of a much more complex inference procedure.

\paragraph{Validation using simulated vs real field data} Some of our findings, particularly those related to optimal control strategies and pathogen-specific interventions, were derived from simulated outbreak scenarios rather than extensive field trials. This reliance on simulation is a practical necessity – it would be infeasible to experimentally trial different pathogen-specific interventions on real farms within the PhD timeline – but it constitutes a limitation in terms of validation. We showed theoretically (in Chapter 3) that the mechanistic model, when given early infection data, can differentiate between pathogens and inform decisions like whether to use a virus-specific treatment or a bacterium-specific antibiotic. These simulations included realistic stochastic variability and indicated significant potential benefits (less antibiotic use, maintained performance). However, real-world BRD outbreaks can be more complex. Multiple pathogens often circulate simultaneously or sequentially in the same group of animals \cite{Gaudino2022}, and subclinical infections can go undetected. The interactions between co-infecting agents (viral and bacterial) may alter disease dynamics in ways not fully captured by our set of discrete pathogen-specific models. For instance, concurrent infections could lead to atypical progression or different treatment responses that the model, which assumes a single dominant pathogen at a time, might not predict. Furthermore, farmer interventions (such as metaphylactic antibiotic treatment or vaccination) in real settings are not as controlled as in our simulations, potentially introducing deviations from model assumptions. In short, there is a gap between simulated performance and field performance of the system. This thesis did not include a longitudinal field trial to empirically confirm that using our hybrid diagnosis-prognosis system leads to better outcomes than status quo decisions. The lack of field validation means that conclusions about management benefits should be interpreted cautiously. For example, the predicted 44\% reduction in antibiotic use assumes perfect adherence to model recommendations and accurate pathogen identification by the model. In practice, there may be cases where the model’s recommendation is not followed or is misinformed by unusual data, which could reduce the realized benefit. This limitation points to the need for future empirical studies: deploying the system on farms to measure its impact on decision-making, disease outcomes, and economic returns. Until such validation is done, our results remain promising indicators rather than proven outcomes. We have taken steps to ensure realism in simulations (e.g., including variability and noise), but empirical calibration against real outbreak data is needed to fine-tune model parameters and to build confidence in the system’s recommendations under practical conditions.

\paragraph{Interdisciplinary and deployment challenges} A further set of limitations arises from the interdisciplinary scope of the project, which brings together expertise from veterinary science, computer science, and agricultural engineering. One issue was the definition of ground truth for training and evaluation. In the absence of a single definitive diagnostic test for BRD, we relied on veterinary clinical assessments (symptom scoring, etc.) as proxies for ground truth labels (infected vs. healthy). However, even experienced veterinarians can disagree on borderline cases, and as noted, BRD is a syndrome with no unique biomarker. This label noise likely impacted the training of our diagnostic model – a limitation common in medical AI applications where labels are imperfect \cite{roy2019weak}). Disagreements over what constitutes “disease presence” introduced uncertainty not only in the model but also among the team members interpreting results. Better approaches to handle ambiguous labels (such as probabilistic labels or consensus labelling) were not fully implemented in this thesis. Another challenge was meeting the diverse expectations of stakeholders. Farmers ideally want a tool that is easy to use, provides clear recommendations, and improves their bottom line. Veterinarians want the tool to be trustworthy, aligning with their clinical intuition and not missing critical cases. Industry partners (e.g., ag-tech companies) are concerned with feasibility: is the system fast and reliable enough, cost-effective, and integrable into farm workflows? Our prototype system, while scientifically promising, is still a proof-of-concept. Usability and integration limitations include the need for continuous data connectivity (ultrasound data had to be uploaded to a server for analysis), the time taken to run analyses (which currently may not be real-time in field conditions), and the requirement for relatively sophisticated hardware (ultrasound machines, GPU servers for the AI model, etc.). In a practical deployment, decisions such as on-device (edge) vs. cloud computing must be addressed to ensure timely feedback to the farmer. We partially addressed this by setting up a basic communication pipeline: for example, audio recordings were transmitted from on-farm sensors to a central server for processing. However, this is only an initial step. We did not fully optimize the system for latency or energy consumption – important factors if devices are battery-powered or connectivity is intermittent. Additionally, the current system would require a technician or vet to perform ultrasounds on calves, which is an extra labour step. Automating or simplifying data collection (perhaps using fixed sensors or self-service kiosks for animals) is another practical hurdle. In summary, the limitation here is that significant work remains to turn the research prototype into a deployable product. This includes improving the user interface, ensuring the analysis can run with minimal user intervention, establishing reliability and fail-safes (what if a sensor fails or gives implausible data?), and conducting training for end-users. The interdisciplinary nature of the project, while a strength, also meant we had to navigate different terminologies and priorities, which occasionally slowed progress or led to compromises in design. These challenges emphasize that technical innovation alone is not sufficient – understanding the context of use is crucial. The thesis lays the groundwork, but a concerted effort with input from farmers, veterinarians, and engineers will be needed to refine the system. Only by doing so can we overcome the last-mile limitations and ensure the tool is accepted and effective in real decision-making scenarios.



% Phil cott PR, Collie D, McGorum B, Sargison N. Relationship between thoracic auscultation and lung pathology detected by ultrasonography in sheep. Vet J 2010;186:53–57
%-----------------------------------
%	SECTION 
%-----------------------------------
\section{Perspectives}

Building on the contributions and acknowledging the limitations discussed, several avenues for future work emerge. These perspectives encompass methodological enhancements and broader explorations to increase the impact and applicability of our hybrid AI-epidemiological framework. We outline key future directions in terms of model improvements, uncertainty handling, inference techniques, decision-making integration, comprehensive validation, and domain transfer. Each of these is aimed at addressing current limitations and pushing the boundary of what such hybrid systems can achieve in animal health management.


\paragraph{Integration of multimodal deep Bayesian mechanistic models} A natural extension of this work is to incorporate multiple sensor modalities (e.g. visual and auditory data) into a unified diagnostic-prognostic framework. In current practice, veterinarians assess BRD using a combination of visual cues (signs of fatigue, nasal discharge, posture) and auditory cues (frequency and nature of coughing, lung sounds) alongside ultrasound findings. Our system could be expanded to emulate this holistic assessment by fusing data from cameras and microphones in addition to ultrasound. Recent developments in deep learning provide methods for audio-visual learning that could be leveraged to this end (Zhu et al., 2020, Deep Audio-Visual Learning survey). For example, an audio analysis model could continuously monitor cough sounds in the barn, which are a strong indicator of respiratory distress \cite{10.1371/journal.pone.0123111}. By aligning cough event data with visual health indicators and ultrasound results, we might improve early detection sensitivity – catching cases that ultrasound-alone diagnostics could miss. However, integrating heterogeneous data streams poses significant challenges: synchronization of signals (timing coughs to specific animal observations), dealing with noise (barn acoustics can be poor, and visuals can be affected by lighting or occlusion), and learning an effective joint representation. Advanced techniques like audio-visual attention mechanisms or representation learning could be applied so that the model learns cross-modal features (e.g., linking an increase in cough frequency with subtle changes in animal posture or ultrasound anomalies). Audio-visual separation methods might help isolate meaningful sounds (coughs vs. background noise) in realistic farm environments 
\cite{carpentier2018automatic}. Future research may implement a multimodal Bayesian mechanistic model, wherein each modality contributes to an overall belief about the herd’s health state, and uncertainties from each sensor are combined. Importantly, incorporating new modalities will require new data – potentially a large labeled dataset of concurrent audio, video, and ultrasound recordings of calves. Obtaining such data is non-trivial, and data annotation becomes a bottleneck. Here, techniques like weakly-supervised learning and active learning could prove invaluable. Rather than exhaustively labeling every instance (which is labor-intensive and prone to human error, one could use active learning to have the model query a human expert for labels on only the most informative or uncertain cases \cite{gal2017deep}. Semi-supervised learning could further allow the model to learn from the abundance of unlabeled sensor data available on farms (e.g., long audio recordings) combined with the limited labeled examples. By coupling these strategies, future systems might build robust multimodal classifiers with far less manual labeling than traditionally required. In summary, expanding to a multimodal, sensor-fusion approach is promising for enhancing diagnostic accuracy and early detection of BRD. This research direction requires methodological advances in multi-sensor data fusion and pragmatic solutions for data collection and annotation on farms, but the payoff would be a more sensitive and veterinarian-like AI system that captures the full spectrum of disease indicators.

\paragraph{Advanced uncertainty quantification and explainability} This thesis took a first step toward uncertainty-aware AI in agriculture using variational Bayesian methods. A future direction is to explore complementary or alternative approaches for uncertainty estimation that provide stronger guarantees and interpretability. One such approach is conformal prediction, a framework that can wrap around any model to produce prediction sets with a guaranteed coverage probability \cite{Angelopoulos2021}. Unlike Monte Carlo dropout, conformal prediction does not rely on Bayesian assumptions; instead, it uses past prediction errors to determine confidence sets for new predictions with rigorous statistical coverage (e.g., “with 90\% probability, the true outcome lies in this set”). Integrating conformal prediction into our deep learning diagnostic could yield more actionable uncertainty estimates – for instance, instead of outputting a single label, the system might output a set of likely diagnoses or a range for the number of infected animals, with an associated confidence level. Matiz and Barner \cite{matiz2020conformal} highlighted that conformal methods can complement Bayesian neural networks by providing calibrated uncertainty measures alongside the model’s point estimates. For our hybrid model, a potential research avenue is a hybrid uncertainty approach: use Bayesian methods (like dropout) to maintain high average accuracy and sharpness of predictions, but apply conformal wrapping to ensure the uncertainty intervals are reliable (e.g., capturing the true outbreak size 95\% of the time). This could result in, for example, prediction intervals for the future number of BRD cases that farmers and vets can trust to a specified probability. The practical challenge will be computational: conformal prediction typically requires an additional calibration step and may need plenty of past data for validation. Moreover, applying conformal prediction in a streaming data context (where the model is used continuously on new farms or new seasons) is an open area of research. Nonetheless, the benefit would be decision-theoretic robustness – users of the system could be presented with worst-case and best-case scenarios within a confidence bound, which might encourage more cautious and risk-aware decisions. Another aspect for future work is explainability of the model’s predictions. In high-stakes domains like animal health, users are more likely to trust and adopt AI if it can explain its reasoning. Techniques such as feature attribution (e.g., highlighting which part of an ultrasound image led to a positive diagnosis) or case-based reasoning (e.g., “this farm’s data closely resembles past outbreak X”) could be integrated so that the system not only predicts but also justifies its predictions. The Bayesian nature of our approach could be leveraged to produce explanations like “the model is only 50\% confident because the inputs are unlike anything seen before,” which itself is useful information. In summary, future research should aim to enhance the trustworthiness of the hybrid model through better uncertainty quantification (possibly combining Bayesian and conformal methods) and improved explainability. This will ensure that as the model’s capabilities grow (e.g., multimodal input), its outputs remain transparent and calibrated – qualities that are essential for real-world deployment and user acceptance.

\paragraph{Robust simulation-based inference and model parametrisation} We identified that our use of approximate Bayesian computation (ABC) for pathogen model selection is promising, but there is room to improve the efficiency and robustness of the inference. One path is to employ Sequential Monte Carlo ABC (ABC-SMC) algorithms or other advanced simulation-based inference techniques. ABC-SMC iteratively focuses simulation effort on parameter regions with higher posterior likelihood, which can greatly improve efficiency over the basic ABC rejection approach we used. By adopting an ABC-SMC approach, we could better explore complex parameter spaces, especially if we integrate more parameters or more complex mechanistic models (for example, models capturing coinfection dynamics). More robust inference could lead to finer discrimination between similar pathogen models or more precise parameter estimates for each model, thus improving the fidelity of forecasts. Additionally, recent developments like using machine learning surrogates within ABC (e.g., regression adjustments or neural density estimators) could be leveraged. For instance, replacing the simple multinomial logistic regression post-ABC with a trained classifier or using distance-learning approaches
\cite{jagalur2021abc} might improve the power to distinguish models using high-dimensional summary data. Another consideration is joint parameter and model inference. In our work, we first identified the most likely pathogen model and then used that model’s best-fit parameters for forecasting. A more rigorous Bayesian approach would be to treat the pathogen identity as just another parameter to infer – effectively averaging predictions over all possible models weighted by their posterior probability (a form of Bayesian model averaging). This could potentially account for uncertainty in pathogen identification in the forecasts (e.g., if two pathogens are similarly likely, the forecast might combine both possibilities). The downside is computational complexity, but ABC-SMC methods are well-suited to approximate this kind of joint inference \cite{beaumont2019abc}. We also note that alternative inference paradigms, like synthetic likelihood or Hamiltonian Monte Carlo for simulator-based models, are emerging and could be tested on our problem to see if they offer gains in speed or accuracy. In sum, the perspective here is to stress-test and refine the inference engine of our hybrid model. By exploring more advanced ABC variants or other likelihood-free inference techniques, future work can ensure that the model calibration and pathogen identification remain robust even as model complexity grows or as we move to more challenging datasets. Such improvements would strengthen the foundation of the entire hybrid approach, since accurate inference is critical to everything from generating trustworthy predictions to learning from new data.

\paragraph{Coupling to robust decision-making frameworks} While we incorporated a basic economic analysis, future research can deepen the integration between epidemiological predictions and decision optimization. For example, instead of outputting a single “optimal” intervention strategy based on average outcomes, the system could use the uncertainty in its predictions to suggest strategies that are robust to worst-case scenarios. This aligns with concepts in decision theory where one seeks solutions that perform acceptably under a range of possible futures, not just the most likely future. In practice, this could mean using the posterior distribution of the mechanistic model parameters (or the predictive distribution of future cases) to evaluate interventions: e.g., choosing a treatment plan that maximizes expected profit and minimizes the risk of catastrophic loss in a bad outbreak. Techniques like Value of Information analysis could also be employed to determine if gathering more data (say, doing an extra diagnostic test) is worth the effort before making a treatment decision. Our current analysis already hinted at interventions (like selective antibiotic metaphylaxis or enhanced biosecurity) and their outcomes, but an explicit decision model would allow one to simulate policies over an entire season or production cycle. Importantly, any such decision-support extension should be evaluated not just on model outputs but on how it impacts real objectives (antibiotic use, cost, and animal welfare). Future collaboration with economists and ethicists might also consider incorporating externality costs (e.g., the societal cost of antibiotic resistance) into the decision-making objective, potentially guiding farmers towards choices that are globally optimal, not just farm-optimal. In summary, the perspective is to evolve our system from a predictive tool into a prescriptive tool – one that can recommend actions under uncertainty. Doing so will likely involve robust optimization techniques and further interdisciplinary work, but it directly addresses the end-goal of this research: not only to predict disease, but to improve disease control outcomes in practice.

\paragraph{Closing the diagnostic-prognostic loop with real-world trials} As noted in the limitations, a critical next step is to validate and refine the integrated system through field studies and deployment pilots. One future research avenue is to implement the full pipeline – from sensor data acquisition to diagnosis to pathogen identification to recommended intervention – on a set of commercial farms, in close collaboration with veterinarians, and monitor outcomes. This would effectively test the “deep mechanistic model with pathogen-specific expert selection” in a real-world setting. Concretely, we envision using the lung ultrasound video dataset (and potentially other sensors) collected in Chapter 2 as the foundation to develop a system that, for each new batch of calves, automatically analyzes incoming sensor data, produces a probabilistic diagnosis for each calf (with uncertainty), and then uses that to infer the most likely pathogen-specific scenario via numerical solvers (e.g., ABC-SMC as discussed). This inference could trigger specific control recommendations (for example, “outbreak likely viral – consider anti-viral and avoid antibiotics unless secondary infection signs appear”). Validation of this approach would involve biological ground truthing: for instance, collecting nasal swabs or blood samples from calves to identify the actual pathogen(s) via PCR or culture, and comparing those to the model’s inferred pathogen. Additionally, one would track metrics like antibiotic usage, illness recurrence, and weight gain in groups managed with model support versus control groups managed by standard practice. Key performance indicators would be whether the model-informed groups use significantly less medication while maintaining health and performance. Any discrepancies or failures observed during such trials would provide invaluable feedback to improve the model (e.g., if the model systematically misses a particular scenario, that model structure might need extension). We should also explore the system’s user experience during these trials: how easily can farm staff and vets interact with it? do they trust the recommendations? By iterating with user feedback, the model and interface can be adjusted (perhaps simplifying outputs or adding explanation features as discussed). Ultimately, such applied research will help transition our framework from a concept to a tangible tool. A successful field demonstration would not only prove out the efficacy of our approach but also possibly reveal new research questions (for example, how to rapidly adapt the model to a farm experiencing an atypical outbreak, or how to incorporate farmer intuition into the AI feedback loop). This “last mile” research is often where interdisciplinary projects either flourish or flounder, so careful experimental design and stakeholder engagement will be paramount. The knowledge gained from these real-world deployments will also inform any necessary regulatory approvals or guidelines for AI in veterinary practice, an emerging area that we have not yet touched but will be important for widespread adoption.

\paragraph{Adaptation to other domains and scalability} Finally, the modular and interdisciplinary nature of our methodology lends itself to transfer and generalization to other infectious disease management problems in agriculture. Future work could test the adaptability of the hybrid model in different contexts, thereby evaluating its generality. For example, one could apply a similar deep learning + mechanistic modelling approach to swine respiratory disease in farrow-to-finish pig operations. Efforts have already been made in modelling porcine infectious diseases with multi-scale agent-based models \cite{Sicard2022},and integrating sensor data (such as cough monitors for pigs or thermal cameras for fever detection) with those models could improve early outbreak detection in swine just as we aimed to do in cattle. Another potential application is in dairy herd health monitoring beyond BRD – for instance, combining sensor-based lameness detection with a mechanistic model of disease spread in a barn to forecast and control a foot-and-mouth disease outbreak. The crop farming sector might also benefit: one could envision using imaging sensors (drones or satellites detecting crop stress) feeding into mechanistic models of pest or disease spread in fields, thereby informing integrated pest management strategies. Coupling mechanistic models in crop protection with ML that interprets sensor images of crop canopies could parallel our work in the plant domain. The challenge in transferring the methodology will lie in customizing each module to the new domain while preserving the overall architecture. The diagnostic AI would need retraining on the new sensor data, the mechanistic model would need to capture the relevant epidemiology (or pest ecology), and the economic module would change to whatever metrics matter (e.g., crop yield or market value). However, none of these require fundamentally new algorithmic development – they are matters of implementation and training, which speaks to the scalability of the approach. We anticipate that as long as the disease system has (1) some form of sensor that provides early indicators, and (2) a mechanistic understanding that can be modelled, our hybrid approach can be applied. One lesson from our work that will be valuable in other domains is the importance of modularity: keeping the components decoupled means a new team of experts can replace or modify one part (say, the pig disease model) without needing to rewrite the entire pipeline. In pursuing these new applications, collaboration with domain experts (swine veterinarians, plant pathologists, etc.) will be crucial to ensure the models are biologically sound. Additionally, computing infrastructure and data management need to scale – a successful deployment in one sector could mean data coming from hundreds of farms, requiring robust cloud support and perhaps automated model updating as more data flows in. These are engineering challenges but foreseeable ones. 
In conclusion, by validating and refining our approach in other livestock or agricultural health contexts, future research can test the universality of the hybrid AI-epidemiological modelling paradigm. If successful, it would mark a significant advance in digital agriculture, providing a general blueprint for smart disease surveillance and control across different farming systems. This would amplify the impact of our initial research, contributing not only to cattle health management but broadly to the sustainability and efficiency of animal and crop health interventions in the era of precision agriculture.


% pour faire références à la thèse d'Hassan, on peut également regarder tout les questions autours du comment disposer les capteurs pour optimiser la collecte d'informations riche et miniser la consommation d'energie (regarder la problèmatique de Hassane). 

% \section{Conclusion}

% Il reste à constuire une bdd unifiés pour faciliter la publication et l'utilisation des données multimodales

% In conclusion, the methodological progress of this thesis demonstrates how to integrating deep learning with mechanistic epidemiological models—grounded in explicit uncertainty quantification—significantly improves both diagnostic accuracy and epidemiological prognosis from sensors observations. 


% By explicitly addressing the complexities and uncertainties inherent in sensor-based disease diagnostics, our approach provides practical, actionable solutions for more effective and sustainable livestock disease management strategies.

% remarks [le but est d'avoir construit de nouvelles connaissances et ouvert de nouvelles questions, pas juste d'avoir une sorte de réponse binaire]

\newpage\thispagestyle{empty}

\fancyhead{} % clear all header fields
\fancyhead[OL]{\textsc{Conclusion}}


%----------------------------------------------------------------------------------------
%	BIBLIOGRAPHY
%----------------------------------------------------------------------------------------
%\printbibliography %Prints bibliography
\fancyhead{} % clear all header fields
\fancyhead[OL]{\textsc{Bibliographie}}
\printbibliography[heading=bibintoc,title={Bibliography}]

%----------------------------------------------------------------------------------------

%: ----------------------- glossary ------------------------
%\printindex
%\fancyhead{} % clear all header fields
%\fancyhead[OL]{\textsc{Glossaire}}
%\printglossary[title=Glossaire,toctitle=Glossaire]

%\input{auxilliaires/glossaire} 

% \chapter*{Abstract}
\addcontentsline{toc}{chapter}{Abstract} 

\includepdf[pages=1]{figures/ED_requirements/Abstract.pdf}

	
% \newpage
% \thispagestyle{empty}


% \mbox{}
% \newpage %résumé

\end{document}  
